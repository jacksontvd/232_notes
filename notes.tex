\documentclass{booc}

\usepackage{feynman}
\usepackage{master}
\usepackage{dynkin}
\hfuzz = 2762pt
\usepackage[margin=1.5in]{geometry}

\begin{document}

\title{Quantum field theory II}
\author{Lectures by: Professor Petr Ho\v{r}ava}
\thanks{Notes by: Jackson Van Dyke; All errors introduced are my own.}
\date{Spring 2018}

\maketitle
\tableofcontents
\pagebreak
\section{Introduction}

QFT is a calculus for describing systems with many interacting and fluctuating
degrees of freedom.
This covers high energy particle physics, 
where we have fluctuations due to $\hbar$, as well as
condensed matter (where $\h$ also plays a role), and classical systems
in condensed matter where the temperature plays this role.
In particular, QFT is most useful for describing local systems
in the sense that the constituents interact with only their local counterparts.
More succinctly, QFT is a machinery which takes a chosen set of symmetries 
and produces predictive results.
Our goals will fall in the two categories:
\begin{enumerate}
\item Techniques and calculations of QFT
\item Big picture, looking towards the horizons of the field
\end{enumerate}
In particular, this semester is meant to approach open questions, rather than solving
problems solved by many people in the past.
We will first focus on Gauge theories, and then move to
effective field theories (EFT). 
In terms of methods, we will concern ourselves over methods of renormalization
and the renormalization group (RG).
The other main camp of methods consists of the so-called \emph{non-perturbative}
techniques.
This is not entirely equivalent, but it is usually correlated with 
the role of topology and geometry in physics.
There are of course overlaps between these four topics, but this is roughly what will be covered:
\begin{enumerate}
\item Gauge theories:
QED, Ward identities, 
non-Abelian Yang-Mills (YM) theories (mostly in four relativistic dimensions),
quantization of non-Abelian YM,
Faddeev-Popov ghosts, BRST (as a generic technique for quantizing arbitrary gauge theories),
BRST-BV (involving good aspects of cohomology theories),
supersymmetry,
topological QFT, topological YM gauge theory (as an example of BRST quantization),
Chern-Simons gauge theory (mostly in $2+1$ dimensions), 
(fractional) quantum Hall effect(s), 
anomalies, spontaneous symmetry breaking, Higgs mechanism.
\item RG:
Confinement of QED in $1+1$ dimensions,
bosonization in 2D (relates free fields describing bosons, to free fields describing fermions),
unitarity of $S$-matrix (optical theorem),
Wilsonian renormalization, critical exponents, 
nonrelativistic novelties (apply the idea of RG to non-rel. field theories)(gravity)
\item Nonperturbative/topology and geometry:
cohomology, dualities (between weak and strong coupling of two different theories),
anomalies (alg. topology, diff. geometry) (local and global),
vortices, monopoles, solitons, instantons,
K-theory (fancy cohomology), topological insulators, 
stable Fermi surfaces, 
large $N$ expansion 
(taking large limit of DOF, provides duality for ADS-CFT correspondence)\footnote{
Whenever you have a chance to find a new duality, do so.
For example if no one ever came up with local Gauge symmetries, 
we can write one down with no gauge symmetries, 
then you can consider a large number of degrees of freedom, 
and discover the local gauge symmetries this way. 
So one should always ask what the dual of some particular symmetry is.}
\item EFT:
Effective actions, low energy QCD, chiral Lagrangian, 
effective field theory of gravity, 
effective field theory of cosmological inflation
(Keldysh-Schwinger).
\end{enumerate}

\chapter{Bosonization}

This is a good opportunity to review some concepts from the first half of this course.
We can review these in the context of free field theories. 
Bosonization itself, is a beautiful equivalence 
between a class of bosonic field QFTs,
and a class of fermionic QFTs.
If you are studying a theory, this theory is not determined by the Lagrangian,
but rather the available observables (operators on your Hilbert space). 
As such, if someone else is studying another
theory, then if there is a mathematical isomorphism between such observables, 
these theories are somehow the same. 
This notion is called \emph{duality}.

We will see this in 
$1+1$ relativistic dimensions
and add in self interactions later.
Here we are truly entering a non-perturbative regime, 
because we are not depending only on Feynman
diagrams and weak coupling expansions.
In particular, using techniques from before, 
we can start with no coupling, turn on small self-coupling, 
and then using this strong-coupling duality, we can switch
to the dual description, 
and use the same weak coupling tools in this strongly couples regime.
When we turn on this strongly coupling duality for a free field theory,
we get the so called sine-Gordon/Thirring model.\footnote{
This theory is treated more thoroughly in \cite{coleman_symmetry}.}

\section{History of bosonization}

The first real introduction of bosonization was from D.C. Mattis and E. Lieb
in 1965 
\cite{mattis_lieb}.
Another landmark paper in the field came from
A. Luther and I. Peschel published in 1975 \cite{luther_peschel}.
This was around the same time that S. Coleman and S. Mandelstam introduced
the sine-Gordon model in \cite{mandelstam_1975}.
also in 1975.
Since then, this has developed into a very powerful tool. 
The most famous reference for the applications in condensed matter
is Haldane's 
``Luttinger liquids''
\cite{haldane}.
There are also many applications of bosonization to string theory.

\section{Bosonic field theory}

Take a scalar field $\phi \in \RR$.
Start with an action
\begin{equation}
S = \int \d{^D x} \left\{ 
\frac{1}{2} \left( \p \phi \right)^2
+ \cancel{V\left( \phi \right)}
\right\}
\end{equation}
where we are Wick rotating to Euclidean space.
In other words, we are integrating 
$\cZ = \int \cD \phi\left( x \right) e^{-S}$ 
rather than 
$\cZ = \int \cD \phi\left( x \right) e^{iS}$.
It is tempting to think this simple theory is boring, but exactly
the opposite is true.
We already know that the naive classical ``engineering'' dimension of the field is 
\begin{equation}
\left[ \phi \right]_\cl = \frac{D-2}{2} 
\end{equation}
Note this may require quantum corrections from loops and the like.
This means $D = 2$ is special. 
To see this, let's calculate the two point function:
\begin{equation}
\lr{\phi\left( x \right) \phi\left( x' \right)} = 
\frac{
\Gamma\left( \frac{D - 2}{2} \right)
}{
4\pi^{D /2}
}
\left( 
\frac{1}{x^2}
\right)^{\frac{D-2}{2}}
\end{equation}
so for $D \fromto 2$, this result becomes questionable.
This motivates the notion that we need a UV regulator in this theory. 
We also see that we need to step back and introduce an IR regulator as well.
These regulators lead to interesting results.

In momentum space our propagator can be written:
\begin{eqnarray}
\int 
\frac{\d{^D k}}{\left( 2\pi \right)^D}
\frac{e^{ikx}}{k^2}
&=&  
\int_0^\infty \d{s}
\int \frac{\d{^D k}}{\left( 2\pi \right)^D}
\exp\left( 
-sk^2 + ikx
\right)
\\&=& 
\int_0^\infty \d{s}
\int 
\left( 
\sqrt{\frac{\pi}{s}}
\right)^D
\frac{e^{-\frac{x^2}{4s}}}{\left( 2\pi \right)^D}
\\&=& 
\frac{1}{2^D\pi^{D /2}}
\int_0^\infty
\d{s} s^{\frac{D}{2} - 1}
e^{-\frac{sx^2}{4}}
\end{eqnarray}
So we need to do something more sophisticated due to the issues at $D = 2$. 
In particular, we need to add an infrared regulating mass $m_0 \neq 0$.
So the propagator in the case of $D = 2$,
$G\left( m , m_0^2 \right)$, depends on the square of this new mass.\footnote{
Note when you list coupling constants as arguments of the propagator, 
it is typical to list them
in the order in which they appear in the action.
This can play a roll in considerations of naturalness etc.}
Now let's back up, and not be so exclusive with this theory. 
To generalize as much as possible, we go back and introduce this new mass to our action. 
This becomes:
\begin{equation}
S\left( \phi \right) = 
\int \d{^2 x}
\left\{ \frac{1}{2}\left( \p \phi \right)^2
+ \frac{1}{2}m_0^2 \phi^2\right\}
\ .
\end{equation}
So now $\psi \in S^1$
and we have a new geometric parameter, $R$,
the radius of $S^1$.
We will write $g\equiv R^{-1}$ to remind us this is a coupling constant.
The change of coordinates here is somewhat up to us. 
We are on $S^1$, 
so we have two obvious choices:
\begin{equation}
\phi = \phi + 2\pi
\qquad\qquad
\psi = \psi + 2\pi R
\end{equation}
We choose the first one.
So where does the radius show up?
As a coupling constant in front of the whole thing:
\begin{equation}
S\left( \phi \right) = 
\frac{1}{g^2}
\int \d{^2 x}
\left\{ \frac{1}{2}\left( \p \phi \right)^2
+ \frac{1}{2}m_0^2 \phi^2\right\}
\end{equation}
we cannot absorb this into the action, so indeed $g$
is a physical coupling constant\footnote{
this is bad terminology, since it doesn't reflect
coupling directly, but it is a parameter on which the system depends} 
An example of where this periodicity shows up is in
composite operators. 
So for $V_\al \ceqq e^{i\al \phi}$, for which values of $\al$
is this operator physical?
Of course this is when $\al$ takes integer values.
So our propagator is now
\begin{eqnarray}
G\left( x , m_0^2 \right) &=&  
g^2 \int
\frac{\d{^2k}}{\left( 2\pi \right)^2}
\frac{e^{ikx}}{k^2 + m_0^2}
\\
&\approx & g^2 \left( -\frac{1}{2\pi}
\log m_0 \abs{x} + \text{const.}
+ \cO\left( m_0 \abs{x} \right)
\right)
\label{eqn:free_scalar_propagator}
\end{eqnarray}
so the obstacle as $m_0 \fromto 0$ is logarithmic.
How does one deal with this?
Before we answer this we consider some properties of this propagator.

The first conclusion from this propagator, is something people often find shocking. 
This is however a physical result with measurable consequences. 
The massless scalar field $\phi$, 
in $1+1$ dimensions doesn't exist as a propagator. 
Any such interpretation doesn't make physical sense, because
the endpoint functions depend on $m_0$. 
An immediate important consequence, is that there is no
spontaneous symmetry breaking ever in global internal continuous symmetries
in $1+1$ relativistic dimensions.
We will see this again in a more formal way.
This statement is actually known as the Coleman\footnote{
This theorem is know by this name by high energy physicists, 
or by some, the Coleman-Mermin-Wagner theorem.
There is also pressure from condensed matter physicists to call it
the CHMW theorem (H for Hohenberg).}
theorem.
How do we prove this?
We just rely on Goldstone's theorem:

\begin{thm}
In a relativistic QFT, 
if a continuous global internal symmetry
is broken, this automatically implies a massless excitation $\phi$
in the system. 
If this does not exist in some particular dimension, 
then it cannot every exist. 
So there couldn't have ever been a breaking of such a symmetry.
\end{thm}

So this theorem immediately implies that $\phi$ doesn't make sense as a physical object.
Note here continuous means symmetries generated by infinitesimal generators. 
So not discrete symmetries.
For example, a reflection of the circle is discrete, and Goldstone has nothing to say about this. 
Global means not gauge-symmetry. 
Internal means as opposed to spacetime symmetries. 
We might wonder if the same story applies to spacetime symmetries.
Actually yes, but the theorem is actually slightly stronger, 
because there is a one to one correspondence between the number of phis, 
and the number of broken symmetries. 
So internal is one implication, spacetime is biconditional.

So if $\phi$ doesn't exist, how can we claim this is a good theory?
Instead, there is indeed a story for such a theory. 
This is ok because we don't care so
much about the Lagrangian, 
we just want a list of observables (operators in the Hilbert space)
with corresponding correlation functions.
We can list some of them as composite operators.
In particular, $\p_\mu \phi$
is a meaningful operator.
It is actually very important. 
A shift symmetry by a  constant (rotation) is a good classical symmetry, 
and cannot be broken, since we have no such spontaneous symmetry breaking. 
We will see quite explicitly, that this cannot be broken by calculating 
correlation functions directly.
This is important because the conserved current of such a $\U\left( 1 \right)$
symmetry is:
\begin{eqnarray}
\p_\mu \phi \sim
j_\mu 
\qquad\qquad
\e^\mu \p_\nu \phi \sim j'_\mu
\end{eqnarray}
The naive engineering dimension is $\left[ \p_\mu \phi \right]_\cl = 1$. 
This shows us that $\p_\mu \phi$ does indeed have meaningful correlation functions. 

The more interesting composite operators are the previously mentioned exponentials. 
So we choose the two simplest ones:
\begin{eqnarray}
V_{\pm} = e^{\pm i\phi}
\end{eqnarray}
These are known in the math and string theory literature as
``vertex operators.''
These show up all over various operator algebras in mathematics.
Now we are interesting in 
\begin{equation}
\lr{
V_+\left( x_1 \right)
\cdots
V_+\left( x_n \right)
V_-\left( y_1 \right)
\cdots
V_-\left( y_m \right)
}
\end{equation}
So what would make this meaningful?
Can we take $m_0 \fromto 0$?
In the presence of the IR regulator, we will also need a UV regulator, 
which we will call $\Lam$. 
We want to trade this UV regulator for some renormalization group scale, called 
$\mu_\RG \equiv \mu$.
So the two things we want to do are send $m_0\fromto 0$ and send
$\Lam \fromto \mu$.
These will both make sense if the correlation functions are meaningful. 
So how do we do this?
We will be able to calculate these correlation functions because they simplify to Gaussian integrals. 
Let us first simplify to the following case:
\begin{eqnarray}
\lr{V_+\left( x_1 \right)V_-\left( x_1 \right)} &=& 
\lr{
\exp\left( i\int \d{^2 x} J\left( x \right) \phi\left( x \right) \right)
}_S
\\&=&  \exp\left( 
-\frac{1}{2}
\int \d{^2 x}\d{^2 y}
J\left( x \right) G\left( x - y , m_0 \right) J\left( y \right)
\right)
\end{eqnarray}
So $V_+\left( x_1 \right)$ 
is a special case of 
\begin{equation}
\exp\left( i\int \d{^2 x} J\left( x \right) \phi\left( x \right) \right)
\end{equation}
when $J = \dd^2\left( x_1 \right) - \dd^2\left( x_2 \right)$.
So we can plug this in to get
\begin{equation}
\lr{V_+\left( x_1 \right)V_-\left( x_1 \right)}=
\exp\left( 
-\frac{1}{2} 2G\left( 0, m_0^2 \right)
+ \frac{1}{2} 2 G\left( x_1 - x_2 , m_0^2 \right)
\right)
\end{equation}
But we have an expression for $G$ from
\eqref{eqn:free_scalar_propagator}.
Now if we send $m_0\fromto 0$
and we get a UV divergence. Now to regulate this, we have to introduce a cutoff. 
The simplest way is to put the system on a two-dimensional lattice with spacing $a$. 
So call $a = 1/ \Lam$ where $\Lam$ has the dimension of energy. 
So now whenever there is an $m_0 \abs{0}$ replace with $m_0 a = m_0 / \Lam$. 
This could be more sophisticated, but we take this more pragmatic route. 
So now
\begin{eqnarray}
\lr{V_+\left( x_1 \right)V_-\left( x_1 \right)}&=& 
\exp\left( 
\frac{g^2}{2\pi}
\left( 
\log\frac{m_0}{\Lam}
 - \log m_0 \abs{x_1 - x_2}
\right)
\right)
\\ &=& 
\left( 
\frac{1}{\abs{x_1- x_2} \Lam}
\right)^{g^2 / \left( 2\pi \right)}
\end{eqnarray}
This is somewhat nice since 
the IR regulators drop out,
but the bad news is that this UV regulator doesn't.
But in reality, this is a good thing, 
because we want a good result in terms of the renormalization scale $\mu_\RG$.
Now the classical dimensions are
$\left[ \phi \right]_\cl = 0$
$\left[ e^{i\al \phi} \right]_\cl = 0$
which develops an anomolous dimension $\gamma$. 
Recall the $\b\left( g \right)$ function determines the 
effects of the coupling constants, 
and $\gamma$ determines the effect given by difference in anomolous dimension.
In any case, $\Lam$ is still in there, 
but in a very nice place, where we can introduce a simple tratment given by
the introduction of $\mu_\RG$.
This is the scale at which we measure the system.
The operators that are being measured, are rescaled versions of these operators. 
So we should call these
$V_{+0}$ or $V_{+B}$ to get bare or renormalized oeprators, but we won't overdo notation. 
Naively we get
\begin{equation}
\lr{V_+\left( x_1 \right)V_-\left( x_1 \right)}=
\left( 
\frac{1}{\mu\abs{x_1 - x_2}}
\right)^{g^2 / \left( 2\pi \right)}
\left( \frac{\mu}{\Lam} \right)^{g^2 / \left( 2\pi \right)}
\end{equation}
So we propose that
\begin{equation}
\sqrt{Z} V_{\pm}^R = V_\pm
\end{equation}
where
\begin{equation}
\sqrt{Z} = 
\left( \frac{\mu}{\Lam} \right)^{g^2 / \left( 2\pi \right)}
\end{equation}
so now
\begin{equation}
\lr{V_+^RV_-^R}=
\left( \frac{1}{\mu\abs{x_1 - x_2}} \right)^{g^2 / \left( 2\pi \right)}
\qquad\qquad
\left[ V_{\pm}^R \right] = g^2 / 4\pi
\end{equation}
even though $\left[ V_{\pm} \right] = 0$.
So we get a non-trivial scaling dimension as expected.
Now recalling $\left[ \p_\mu \phi \right] = 1$,
we now have a list of observable operators: $\p_\mu \phi , V_{\pm}$.

Now we've made our renormalization, so this better work for 
the extended calculation.
First recall:
\begin{equation}
\lr{V_+\left( x \right) V_-\left( 0 \right)} = 
\exp\left( 
G\left( 1/ \Lam \right) - G\left( \abs{x} \right)
\right)
\end{equation}
Now we can finally calculate:
\begin{eqnarray}
\lr{V_+\left( x_1 \right)\cdots V_+\left( x_n \right)
V_-\left( y_1 \right) \cdots V_-\left( y_m \right)}
=
G_{\text{bare}}^{\left( n,m \right)}\left( x_1, \cdots , y_m \right) \\
=
m_0^{\frac{\left( n-m \right)^2g^2 }{ 2\pi}}
\left( 
\frac{1}{\Lam}
\right)^{g^2\frac{n+m }{4\pi}}
\left( 
\frac{
\prod_{i\neq j}^n \abs{x_i - x_j}
\prod_{i'\neq j'}^m \abs{y_{i'} - y_{j'}}
}{
\prod_{i,j}\abs{x_i - y_j}
}
\right)^{g^2 / \left( 2\pi \right)}
\end{eqnarray}
where we got the factor $n-m+ n\left( n-1 \right) + m\left( m+1 \right) - 2nm = \left( n-m \right)^2$
from
\begin{equation}
J\left( x \right) = 
\sum_{i = 1}^n \dd\left( x_i \right) - 
\sum_{j =1}^m \dd\left( y_j \right)
\end{equation}
So this gives us that for
$n\neq m$, we get $G^{\left( n,m \right)} \equiv 0$.
Now this expression should suggest we have some symmetry. 
We know what this symmetry is. 
It is a global $\U\left( 1 \right)$ continuous symmetry, 
where $\phi\left( x \right)\fromto \phi\left( x \right) + c$
where $c$ is a spacetime independent constant.
Even independent of the Goldstone considerations, we would have found 
this symmetry from this correlation function calculation.
So now $n = m$ is forced on us, so we just need to calculate:
\begin{eqnarray}
G^{n,n}_{\left( R \right)}\left( \cdots \right)  
=
\left( 
\frac{
\prod^n_{i\neq j} \mu\abs{x_i - x_j} \prod^n_{i\neq j} \mu\abs{y_i - y_j}
}{
\prod_{i,j} \mu\abs{x_i - y_j}
}
\right)^{g^2 / \left( 2\pi \right)}
\end{eqnarray}

Now the RG equation\footnote{
This is also a Callan-Symanzik equation.}
can be written:
\begin{eqnarray}
0 = \mu \frac{d}{d\mu}\lr{\ubr{V_+ \cdots V_i}{2n}}
= Z^n\left( 
\mu \frac{\p}{\p\mu} + 2n\mu
\frac{\p \log\sqrt{Z}}{\p\mu} \lr{V_+^{\left( R \right)} \cdots V_-^{\left( R \right)}}
\right)
\\ \implies
\left( \mu \frac{\p}{\p\mu} + 2n \gamma \left( \frac{g^2}{4\pi} \right) \right)
G_{\left( R \right)}^{\left( n,n \right)}
\end{eqnarray}
Now we can compare this to the standard RG equation:
\begin{equation}
\left( \mu \frac{\p}{\p\mu} + 
\b\left( g_R \right)
\frac{\p}{\p g_R} + 
n \gamma_+ \left( g_R \right) + 
n\gamma_- \left( g_R \right)
\right)
G_{\left( R \right)}^{n,n}
\left( x  ,y ; g_R ;\mu \right)
\end{equation}
where $\b$ is the beta function, and $\gamma_{\pm}\left( g_R \right)$ 
are the anomalous dimensions of $V_{\pm}$.
Notice $\b\left( g_R \right) = 0$ for this theory.
When $\b\left( g_R \right)$ then we are sitting on the line of exact CFTs.
Systems where you have families of exact CFTs are important for various reasons.
There is a full class of CFTs of this type, with $c = 1$
where $c$ is the central charge.
We will return to this, since it shows off these boson fermion equivalences we are trying to derive.
Note that the scalar theory has a $\U\left( 1 \right)$ symmetry, 
but in the massless case there are even more. 
These will be easier to find in comparison with the fermion case.

\section{Fermionic field theory}

We now consider free fermions in Euclidean, Wick rotated $\RR^{1, 1}$ Lorentz spacetime. 
This theory can be treated in a significantly simpler manner.
We choose a Dirac fermion $\psi$ (i.e. complex valued). 
We can compare this to having two Majorana fermions rather than one dirac fermion. 
On an infinite plane, these look the same, but if we are working on a 
more complicated underlying spacetime, then these things become very different
when we take on the periodicity of non-contractible cycles.
We start with an action of the form
\begin{equation}
S\left( \psi , \bar\psi \right) = 
\int \d{^2 x} \bar \psi \fsl{\p} \psi
\end{equation}
where $\bar \psi$ is the dirac conjugate.
Now recall we have 
\begin{equation}
\psi = 
\begin{pmatrix}
\psi_+ \\
\psi_-
\end{pmatrix}
\end{equation}
In the component notation, the action can be written
\begin{equation}
S = \int \d{z} \d{\bar z}
\left( 
\bar \psi_+ \p_{\bar z} \psi_+ + 
\bar \psi_- \p_z \psi_-
\right)
\end{equation}
Notice we have a classical $\U\left( 1 \right) \times \U\left( 1 \right)$
symmetry, where
\begin{eqnarray}
\psi &\fromto & e^{i\theta + i\theta_5 \gamma_5}
\psi
\\
\bar \psi & \fromto & e^{-i \theta + i\theta_5 \gamma_5} \bar \psi
\end{eqnarray}
So this already predicts, that if this is supposedly 
equivalent to a free scalar, then we better have another
copy of $\U\left( 1 \right)$ symmetry laying around in that theory.

The conserved current for each one of these $\U\left( 1 \right)$ symmetries is
\begin{eqnarray}
J_\mu &=& \bar \psi \gamma_\mu \psi\\
J_\mu^5 &=&  i \gamma_5 \gamma_\mu \psi
\end{eqnarray}
Now writing this in terms of the components, we get
\begin{eqnarray}
J_1 &=&  \bar \psi \gamma_1 \psi = \bar \psi_+ \psi_+ + \bar \psi_- \psi_-\\
J_2 &=&  i \left( \bar \psi_+ \psi_+ - \bar \psi_- \psi_- \right)
\end{eqnarray}
Now since we have the identity:
$i \gamma_5 \gamma_\mu = - \e_{\mu\nu} \gamma^\nu$,
we see that these two conventions are related to one another from the $\e$ symbol:
$J_\mu^5 = -\e_{\mu\nu} J^\nu$.
So now we have the theory and its symmetries, and we want to 
calculate some correlation functions. 
In particular, we need the propagator:
\begin{eqnarray}
\Delta_+ &\ceqq&
\lr{\bar \psi_+ \left( z, \bar z \right) \psi_+\left( 0 \right)}
=  - \frac{1}{2\pi z}
\\
\Delta_- &\ceqq & \lr{\bar \psi_-\left( z , \bar z \right)
\psi_-\left( 0 \right)} = -\frac{1}{2\pi \bar{z}}
\end{eqnarray}
And this is all we really need to calculate correlation functions.
For example, 
\begin{eqnarray}
\lr{\prod_1^n \bar \psi_+\left( z_i \right)
\psi_+\left( z_i ' \right)}
&=& \left( -\frac{1}{2\pi} \right)^n
\det\left( 
\frac{1}{z_i - z_i'}
\right)
\\
\lr{\prod_1^n \bar \psi_-\left( \bar z_i \right)
\psi_-\left( \bar z_i ' \right)}
&=& \left( -\frac{1}{2\pi} \right)^n
\det\left( 
\frac{1}{\bar z_i - \bar z_i'}
\right)
\label{eqn:fermi_endpoint_functions}
\end{eqnarray}


\section{Equivalence}

Now we move on to the equivalence of these two theories. 
This really means showing that the endpoint functions are the same.
To do this, we need to massage the functions calculated in \eqref{eqn:fermi_endpoint_functions}.
We have the following identity to help us:
\begin{equation}
\left( -1 \right)^{n+1}
\det\left( 
\frac{1}{z_i - z_i'}
\right)
=
\frac{
\prod_{i<j} \left( z_i - z_j \right)
\prod_{i' < j'} \left( z'_{i'} -  \right)
}{
\prod_{i,j} \left( z_i  - z_j' \right)
}
\end{equation}
\begin{proof}
Proceed by induction.
For $n= 2$,
we can calculate:
\begin{eqnarray}
\det\left( \frac{1}{x_i - y_i} \right) &=&  
\frac{1}{\left( x_1 - y_1 \right)\left( x_2 - y_2 \right)}
- 
\frac{1}{\left( x_1 - y_2 \right)\left( x_2 - y_1 \right)}
\\ &=& 
\frac{\left( x_1 - y_2 \right)\left( x_2 - y_1 \right) - 
\left( x_1 - y_2 \right)\left( x_1 - y_2 \right)}{\prod_{i,j}\left( x_i - y_j \right)}
\end{eqnarray}
and expand. 
\end{proof}

Now we introduce
\begin{equation}
\sigma_+ = \bar \psi_- \psi_+
\sigma_- = \bar \psi_+ \psi_-
\end{equation}
So we have
\begin{eqnarray}
\lr{
\prod_{1}^n \sigma_+\left( x_i \right)\sigma_-\left( y_i \right)
}
&=& \left( -1 \right)^n \lr{\prod^n \bar \psi_+\left( y_i \right) \psi_+\left( x_i \right)}
\lr{\prod \bar \psi_- \left( x_i \right) \psi_-\left( y_i \right)} 
\\
&=& 
\left( \frac{1}{2\pi} \right)^{2n}
\left( 
\frac{
\prod_{i<j} \abs{x_i - x_j}^2
\abs{y_i - y_j}^2
}{
\prod_{i,j} \abs{x_i - y_j}^2
}
\right)
^{g^2 / \left( 4\pi \right)}
\end{eqnarray}
so when $g^2 = 4\pi$, this exponent is just one.
Now we get the equivalence between the correlation function from the bosonic theory:
\begin{equation}
\left( \frac{1}{2\pi} \right)
\lr{
\prod_{i = 1}^n
\mu^2 V_+^{\left( R \right)}\left( x_i \right)
V_-^{\left( R \right)}\left( y_i \right)
}
\end{equation}
So our bosonization expression is
\begin{equation}
2\pi \sigma_{\pm} = 
\mu V_{\pm}^{\left( R \right)}
\end{equation}
Then with the expression
\begin{equation}
V_{\pm}^{\left( R \right)} = 
\left( \frac{\Lam}{\mu} \right)^{g^2 / \left( 4\pi \right)}
V_{\pm}
\end{equation}
we get
\begin{equation}
\bar \psi_-\psi_+ = 
\frac{\Lam}{2\pi}e^{i\phi}
\qquad\qquad
\bar\psi_+ \psi_i = 
\frac{\Lam}{2\pi} e^{-i\phi}
\end{equation}
Now we notice that the Lagrangians formally match as well,
so we can finish off our ``dictionary:''
\begin{eqnarray}
\bar \psi \fsl{\p} \psi 
\leftrightarrow 
\frac{1}{2}\left( \p_\mu \phi \right)^2
\qquad\qquad
j^\mu = \bar \psi \gamma^\mu \psi 
\leftrightarrow 
\frac{\e^{\mu\nu}}{\sqrt{\pi}} \p_\nu \phi
\\
\bar\psi \psi 
\leftrightarrow
-\Lam \cos \sqrt{4\pi} \phi
\qquad\qquad
\bar \psi \gamma_* \psi
\leftrightarrow
\Lam \sin\sqrt{4\pi} \phi
\end{eqnarray}
Now call the current corresponding to the scalar $\U\left( 1 \right)$ shift symmetry:
$\cJ^\mu = \p_\mu \phi$. 
Notice we aren't getting the same one here, we're getting a dual version:
\begin{equation}
\frac{\e^{\mu\nu}}{\sqrt{\pi}} \p_\nu \phi
\equiv j^\mu\sim \e^{\mu\nu} \cJ_\nu
\end{equation}
Later on we will call this an electric-magnetic duality.
At this point, we might consider proposing another duality:
the scalar field $\phi$ should also be equivalent to another scalar, say, $\theta$, 
related by
\begin{equation}
\p_\mu \phi \equiv \e_{\mu\nu}\p^\mu \theta
\end{equation}
So we can write this in terms of $\phi$ or $\theta$. 
This is known as T-duality in the string theory literature.
This seems strange, but in some sense, this is quite clear in $1+1$ dimensions. 

Let us now return to a Minkowski signature. 
We introduce light-cone coordinates: $x^{\pm} = t \pm s$.
Then the KG equation is given by
\begin{eqnarray}
0&=& \p_+ \p_- \phi 
\\
\phi &=&  \psi_+\left( x^+ \right) + \phi_-\left( x^- \right)
\end{eqnarray}
so we get a left and right moving piece. 
Now we can just define:
\begin{equation}
\theta = \phi_+\left( x^+ \right) - \phi_-\left( x^- \right)
\end{equation}
so in either case, we are describing the same theory with two different scalar fields. 
It turns out this type of duality symmetry related systems at large and small $R$ 
on the RG scale.
This duality symmetry takes $R\fromto \something /R$.
This is a refinement of the moduli space of these scalar fields in 1+1 dimensions with
one free scalar field component.
So our claim is that the dirac theory is somewhere on this scale at $R_D$, 
which is something like $4\pi$. 
Then if we seek the location of the pairs of Majorana fermions,
there is another radius further up, which has a branch moving off of this exact CFT line, which
corresponds to some scalar field theory of $\tilde \phi$
which takes values in $S^1 / \ZZ_2\equiv I$
which is known as an orbifold. 
Now, amazingly, there is a point on this other branch, which we call $R_{\text{majorana}}$
which is equivalent to a pair of majorana fermions. 
So we are fillig in this space of theories with one scalar degree of freedom.
So how do we evaluate how many degrees of freedom we have in some particular theory,
if the fields aren't quantum objects?
This is related to central charge and conformal anomaly, which we will be studying later. 
But there is a good set of correlation functions in these dimensions, 
and there is a good invariant, the central charge, which gives the number of DOF in a unitary CFT. 
Amazingly enough, all such theories have been classified for $c\leq 1$.





\section{Massless Schwinger model}

We now consider the application of bosonization to the massless
Schwinger model.
The action for this theory takes the form:
\begin{equation}
S_\QED = \int \d{^2 x}
\left\{ 
\bar \psi \fsl{\p} \psi + 
A_\mu j^\mu
- \frac{1}{4}F_{\mu\nu}F^{\mu\nu}
\right\}
\end{equation}
where $j^\mu$ is the electromagnetic current associated to the Dirac field.
Note that when dealing with non-abelian Yang-Mills theories 
one often writes $1 / \left( 4e^2 \right)$ to replace the coefficient of 
$\left( F^{\mu\nu} \right)^2$
which is just amounts to rescaling the field.
Recall $F_{\mu\nu} = \p_\mu A_\nu - \p_\nu A_\mu$.
Note that in $d + 1$ dimensions, we have $d -1$ polarizations. 
So here, we have $0$ classical polarizations. 
But as we have seen, we should not trust classical Lagrangians. 
We might not actually have a propagating fermion and $0$ polarizations. 
But even with $0$ polarizations, this isn't trivial, because we might still have
non-zero electric fields.
So we might have a long-distance field, even if we don't have propagating photons,
and we will actually see, using the tool of bosonization that we even get a degree of freedom
with non-zero mass.

Now trade these fermions $\psi , \bar \psi$ for $\phi$
using bosonization.
Now write $A_\mu= \p_\mu \al$
so when we make a scalar transformation, $\al$ just becomes 
$\al + \e_{\mu\nu} \p^\mu f\left( x , t \right)$.
Now $f$ is gauge invariant, so we can just locally go to a gauge where $\al = 0$, and essentially
forget about $\al$.
Then this becomes
$A_\mu = \e_{\mu\nu} \p^\mu f\left( x , t \right)$.
Now we have some trivial identities in Minkowski signature:
\begin{eqnarray}
\e_{01} = +1
&\qquad\qquad&
\e_{\mu\nu} \e^{\nu\sigma} = \dd_\mu^\sigma
\\
\e^{01} = -1 
&\qquad\qquad&
\eta_{\mu\nu} =
\begin{pmatrix}
+1 & 0 \\
0 & -1
\end{pmatrix}
\end{eqnarray}
Then according to our discussion above, our action becomes:
\begin{equation}
S = \in \d{^2 x}
\left\{ 
\frac{1}{2} \left( \p_\mu \phi \right)^2 + 
\frac{\p_\mu \phi}{\sqrt{\pi}}
\p^\mu f
+ \frac{1}{2e^2} \p^2 f \p^2 f
\right\}
\end{equation}
where $\p^2 = \p^\mu \p_\mu$. 
So our first result, is that this complicated fermionic theory is really just a simple bosonic
theory.
So instead of working to see we don't have any special fermionic effects, we can just
use bosonization to treat it as a Gaussian theory, where we can readily make exact calculations.
Note our source in the Gaussian integration is
$J = -\p^2 f / \sqrt{\pi}$.
So we can write:
\begin{eqnarray}
S &=&  
\int \d{^2 x} \left\{ 
-\frac{1}{2\pi} \p_\mu f \p^\mu f
+ \frac{1}{2e^2} \p^2 f \p^2 f
\right\}
\\ &=& 
\int \d{^2 x} \left\{ 
\frac{1}{2\pi} A_\mu A^\mu  +\frac{1}{2e^2} A)\mu \p^2 A^\mu
\right\}
\end{eqnarray}
Now note that in momentum space, we get a dispersion relation: 
$\om^2 = k^2 + e^2 / \pi$
so we do have one propagating degree of freedom, with a mass. 
Originally Schwinger was trying to make the point that gauge invariance does not imply
a massless theory.
Now we can calculate the two-point functions in momentum space:
\begin{eqnarray}
\lr{A_\mu A_\nu} = 
\left( 
\eta_{\mu\nu} - \frac{p_\mu p_\nu}{p^2}
\right)
\frac{1}{p^2 - e^2 / \pi}
\end{eqnarray}
Note this is not a free field approximation,
this is the exact two point function.
This has no cut, only a pole at the mass of the physical particle.
\begin{defn}
A field theory is said to exhibit \emph{confinement},
when there are no asymptotic fermionic states with non-zero charge.
\end{defn}
In other words, we have shown confinement in this theory. 

We can also show that this theory exhibits spontaneous breaking of the chiral symmetry.
This is very similar to what we expect from QCD.
We can argue this by evaluating:
\begin{eqnarray}
\lr{\sigma_{pm}} = 
\frac{\Lam}{2\pi}
\lr{
e^{+ i\theta}
}=\exp\left( 
-\frac{1}{2} \left( -\frac{1}{2\pi} \log \frac{m}{\Lam} + \cdots\right) 
4\pi
\right)=
\frac{m}{2\pi}
e^{\ldots}
\end{eqnarray}
again using the relations from bosonization.
Note the mass of this scalar $\theta$ is
$e / \sqrt{2\pi}$.
So all the $\Lam$ terms cancel in the last step.
\begin{exr}
Check that at short distances ($x\fromto 0$)
we have that:
\begin{equation}
\lr{\sigma_+\left( x \right) \sigma_-\left( 0 \right)} \sim \frac{1}{2\pi^2 x^2}
\end{equation}
\end{exr}

\section{Thirring model - sin-Gordon theory}

We will develop a duality between two theories, neither of which are free
using the following correspondences:
\begin{eqnarray}
\bar \psi \fsl{\p}\psi
&\longleftrightarrow &
\frac{1}{2}\left( \p_\mu \phi \right)^2 
\\
\bar \psi \gamma^\mu \psi 
&\longleftrightarrow &
\frac{\e^{\mu\nu}}{\sqrt{\pi}} \p_\nu \phi
\\
\bar \psi \psi 
&\longleftrightarrow &
-\Lam \cos\left( \sqrt{4\pi} \phi \right)
\\
\bar \psi \gamma_* \psi
&\longleftrightarrow &
\Lam \sin\left( \sqrt{4\pi} \phi \right)
\\
\left( \bar \psi \psi \right)^2 
&\longleftrightarrow &
\left[ -\Lam \cos\left( \sqrt{4\pi} \phi \right) \right]^2 \approx
-\frac{1}{2\pi} \left( \p_\mu \phi \right)^2
\end{eqnarray}
Recall that
\begin{equation}
\lr{\sigma_{\pm}} \sim
\Lam e^{G\left( 0 , m^2 \right)} \sim m
\end{equation}
So the UV cutoff showing up here is important because it shows us
this is not some silly technical triviality, and $\Lam$ plays a critical role
in a full understanding of these theories.

\subsection{Sin-Gordon theory}

The most practical way to discover the sine-Gordon theory, would be to start with 
some fermionic theory, and ask what theory it is dual to.
It was however actually discovered on its own, and later seen to be dual to a fermionic theory.

The action for this theory is:
\begin{equation}
S_{\text{SG}} = 
\int \d{^2x}
\left\{ 
\frac{1}{2} \p_\mu \phi \p^\mu \phi
+ \frac{m^4}{\lam}\left\{ 
\cos\left( 
\frac{\sqrt{\lam}}{m} \phi
\right) - 1
\right\}
\right\}
\end{equation}
we just subtract $1$ such that for $\phi = 0$ we get $0$ energy.
Note that we have introduced two couplings. 
The coupling $m$ will play the role of mass, and $\lam$ will play the role of self coupling.
Now notice for small values of $\phi$, we get that the first term is roughly:
\begin{equation}
\approx \frac{1}{2} \left( \p_\mu \phi \right)^2 - 
\frac{1}{2} m^2 \phi^2 + \frac{\lam}{4!} \phi^2 - 
\frac{\lam^2}{m^2} \frac{\phi^6}{6!}
\end{equation}
We will not prove it here, but we offer the fact:

\begin{prop}
The sine-Gordon theory is classically integrable.
\end{prop}

Note integrability has many different definitions.
The practical definition is a theory which can be rewritten
completely in terms of action-angle variables.
That is, taking variables $p , q \fromto I , \theta$ where $I$ is some action. 
Then the Hamiltonian becomes very simple. 
Then a theory possesses quantum integrability 
when we can solve for the $S$-matrix exactly.

We now state some elementary facts about this theory:
The equation of motion does indeed contain $\sin$:
\begin{equation}
\p^\mu \p_\mu \phi + 
\frac{m^3}{\sqrt{\lam}}\sin\left( \frac{\sqrt{\lam}}{m}\phi \right) = 0
\end{equation}
We can gather from this, that the theory also 
has an infinite number of degenerate ground states.
We can always ask for perturbations around such minima, and then using perturbation
theory to get some Gaussian integrals and such. 
But we can also have theories which jump from one to another. 
Well $\phi = 0$ is a solution, but now we want to 
impose a boundary condition for which $\phi$ is at one minimum
at $-\infty$ in space, and in the neighboring minimum at $\infty$ in space. 
To get this, we introduce a static\footnote{
We know how to boost static solutions to arbitrary
velocity, so this actually gives us a whole family of soliton solutions.}
``soliton'' solution.
A soliton is a generic name for a quantum object which does not dissipate.

We can anticipate that this will be $0$ at $-\infty$ and then
jump to the next level between there and $\infty$ on plot of $\bar\phi$ versus $\bar x$
where have introduced:
\begin{equation}
\bar x = mx
\qquad
\bar t = mt
\qquad
\bar \phi = \frac{\sqrt{\lam}}{m} \phi
\end{equation}
Now we note some symmetries of the theory.
\begin{equation}
\bar \phi \fromto -\bar \phi
\qquad\qquad
\bar \phi \fromto \phi + 2\pi n , n\in \ZZ
\end{equation}
We can now generalize and impose the boundary condition that
$\bar \phi\left( -\infty \right) = 2\pi N_1$ 
is the minimum of the potential for integer $N_1$
and $\bar \phi\left( \infty \right) = 2\pi N_2$ for integer $N_2$.
Then define $Q = N_1 - N_2$ to be the topological quantum number. Now we can actually express:
\begin{equation}
Q= \frac{1}{2\pi} \int \d{\bar x} \p_{\bar x} \bar \phi
\sim \int \d{\bar x} j^0
\qquad\qquad
j^\mu = \e^{\mu\nu} \p_\nu \bar \phi
\end{equation}
Note this current
is conserved because $\p_\mu\left( \e^{\mu\nu}\p_\nu \bar \phi \right) = 0$
identically.
Note that $\phi \fromto \phi+c$ with current $\tilde j^\mu = \p^\mu$ is not a symmetry here.

Now we can actually write down a solution.
The simplest solution has $Q = \pm 1$. 
We can check trivially that
\begin{equation}
\bar \phi\left( x \right) = 
\pm 4 \arctan\left( e^{-\bar x - \bar x_0} \right)
\end{equation}
is a solution.
The theory has many more, such as an analytically known solton anti-soliton solution.
This solution corresponds to scattering of a soliton off of an anti-soliton, 
and is therefore a time dependent solution.
\begin{equation}
4\arctan\left\{ 
\frac{
\sinh\left( 
\frac{u\bar t}{\sqrt{1 - u}}
\right)}{
u\cosh\left( \bar x / \sqrt{1 - u^2} \right)}
\right\}
\end{equation}

We also have a ``breather solution'' given by
\begin{equation}
\bar \phi = 4\arctan\left\{ 
\frac{
\sin\left( v \bar t / \sqrt{1 + v^2} \right)
}{
v\cosh\left( \bar x  / \sqrt{1 + v^2} \right)
}
\right\}
\end{equation}
This one is more of a bound state. 
It is time dependent in the sense that it oscillates out and back, 
but represents one compact object. It is sometimes called a doublet.

We might wonder why this Soliton solution is considered to be non-perturbative.
One way to see this is to notice that we 
can evaluate the energy classically by plugging this into the classical energy formula, which gives
\begin{equation}
E_{\ext{soliton},\cl} = 
\frac{8m^3}{\lam}
\end{equation}
Clearly this is non-perturbative, since turning $\lam$ off doesn't make any sense. 
This inverse scaling with couplings is very typical for these solitonic objects.
We can still handle them in this particular example since this theory is integrable.\footnote{
One often requires supersymmetry in a non-integrable theory to control our solitons.}
Quantum mechanically, the energy can be calculated as:
\begin{equation}
E_{\text{soliton}, qm} = \frac{8m^3}{\lam} + \frac{m}{\lam} + \cO\left( \lam \right)
\end{equation}

Now the sine-Gordon model, in the notation 
$\phi' = \frac{\sqrt{\lam}}{m} \phi$, can be rewritten
\begin{equation}
S_{\text{SG}} = \frac{m^2}{\lam}
\int
\frac{1}{2} \left( \p_\mu \phi' \right) + m^2 \cos \phi'
\end{equation}
the important thing to notice here, is that the coupling constant $\kappa^2 = \frac{\lam}{m^2}$
is present with a power of $-2$.
Some of the literature takes $t = \k^2$.

\subsection{Thirring model and duality}

The Thirring model Lagranian is written
\begin{equation}
S_T = \int \d{^2 x}
\left\{ 
\bar \psi\left( 
\fsl{\p} + m_F
\right) \psi - \frac{1}{2} g J_\mu J^\mu
\right\}
\end{equation}
Now we can apply the duality sequence to various terms in this action:
\begin{equation}
S =
\int \d{^2 x} \left\{ 
\left( \p_\mu \phi \right)^2 + 
\Lam \cos \phi' + \frac{g}{\pi} \left( \p_\mu \phi \right)^2
\right\}
\end{equation}
Then to make contact with the sine-Gordon model, we see that the appropriate equation is:
\begin{equation}
1 + \frac{g}{\pi} = \frac{4\pi}{\kappa^2}
\end{equation}
The main implication of this is that
$m_F = 0$. 

But where does the fermion show up in the sine-Gordon theory?
Mandelstam showed that the fermion is exactly the soliton in the sine-Gordon theory.
If we decompose the scalar into the left chiral and left chiral component, we had
\begin{equation}
\sigma_{\pm}
\bar \psi \psi \sim e^{\# \phi} = e^{\# \phi_+} e^{\# \phi_-}
\end{equation}
which suggests the above.

\section{Theory classification}

Recall the classification space of all CFTs in $1+1$ dimensions which
have the same ``number of degrees of freedom'' as a real scalar.
The horizontal axis is parameterized by the radius $R$ of $S^1$.
We have $R = 0$ on the far on left, and some critical radius $R = R_*$ 
where the theories with $0 < R < R_*$ are in correspondence with the theories
with $R > R_*$ where we decompose
\begin{equation}
\phi = \phi_+\left( x^+ \right) + \phi_-\left( x^- \right)
\theta = \phi_+\left( x^+ \right) - \phi_-\left( x^- \right)
\end{equation}
\begin{equation}
R\fromto \frac{\#}{R}
\qquad\qquad
\int \left( \p\phi \right)^2\fromto \int\left( \p\theta \right)^2
\end{equation}
and send 
The same correspondence holds for the vertical axis parameterized by
$R_I$, where $R_I$ is the radius of $S^1 / \left( \ZZ / 2\ZZ \right)$.
Amazingly enough, these two one dimensional families of free field theories, touch each other
at a certain radius, where $R_I = R_{I*}$. 
Note this is not where $R = R_*$. 
The crossing point corresponds to $R_I = R_{I*}$ and $R > R_*$.

We have proven that the Dirac fermion sits somewhere
between $R = R_*$ and the point where the $R_I$ axis crosses the $R$ axis.
we call this $R_D$. 
As it turns out, the pair of Majorana fermions
(which would be equivalent on local spacetime to a Dirac fermion)
live somewhere on the $R_I$ axis. 
Explicitly, the Majorana$^2$ theory lives at $R_I = R_M$.

Also recall the theory at $R = R_D$ 
had a $\U\left( 1 \right)\times \U\left( 1 \right)$ symmetry.
In fact, this symmetry is present along the whole $R$ axis.
Additionally, if we consider $R = R_*$, 
this symmetry gets enhanced and becomes an 
$\SU\left( 2 \right)\times \SU\left( 2 \right)$ symmetry.
So when we have the $\left( \U\left( 1 \right) \right)^2$ symmetries,
there will be currents $\p_\mu \phi$, $\e^{\mu\nu}\p_\mu \phi$, each with
dimension one.
Now for the enhanced case at $R = R_*$, 
some of the allowed $e^{\pm i \phi_{\pm}}$ will also get dimension $1$, 
so they can also be candidate conserved currents.

\begin{exr}
Show that these candidate currents do indeed satisfy the proper commutation relations.
\end{exr}

In addition to the theories which are parameterized by $R$ and $R_I$, 
we have three isolated theories living in the bottom right.
These are given by the symmetries of the dodecahedron, the octahedron, and the icosahedron.
We call these $D$,$O$, and $I$.
The reason we mention this, is because this classification of CFTs, turns out 
to be one of many mathematical classification problems
known as an ``ADE'' classification.
This means there are three sequences of objects. 
The $A$ sequence of objects, $\left\{ A_n \right\}_{n\in \NN}$,
the $D$ sequence of objects, $\left\{ D_n \right\}_{n\in \NN}$,
and the $E$ sequence $\left\{ E_6 , E_7, E_8 \right\}$.
In this problem the $R$ axis gives the $A$ sequence, 
the $R_I$ axis gives the $D$ sequence, and the $E$ sequence is given by 
these $D$,$O$, and $I$ theories.
As it turns out, these show up because of the $\SU\left( 2 \right)$ symmetry. 
As we saw, the appropriate vertex operators conspire to get this new symmetry, 
so there is an underlying $\SU\left( 2\right)$ which we can start with.
Then we can ask the following interesting question:
what are all of the possible discrete subgroups $\Gamma \subseteq \SU\left( 2 \right)$?
For each such $\Gamma$, we can then consider $\SU\left( 2 \right) / \Gamma$.
Note this is precisely mapped to an ADE classification problem.
There is an infinite $A$ sequence of subgroups which contain 
$\ZZ_n$
an infinite $D$ sequence which contain
$\ZZ_2 \ltimes \ZZ_n$
and then $E_6$,$E_7$, and $E_8$.
Now factoring out by an $A_n$ puts us on $R$ axis,
factoring out by some $D_n$ puts us on the $R_I$ axis, 
and then the $E$s give you these three loose points.

So applying what we learned in the sin-Gordon Thirring duality,
we see that we actually have extended the duality to 
the whole line using self-interacting massless theories. 
Similarly we could show the same for the Majorana theories along this vertical axis.

\chapter{Non-abelian Yang-Mills}

Recall that QED
is a $\U\left( 1 \right)$ gauge theory with field $A_\mu$.
Non-abelian Yang-Mills theories are essentially an attempt to generalize this notion
to general Gauge fields to describe
other interactions, such as the strong and weak interactions, with some
$A_\mu^I$ where $I$ is indexed by the adjoint
representation of some Lie group $G$.
This will typically be a compact Lie group 
since this will give us positive energy and a well defined
Hamiltonian and Lagrangian.
The issue is of course that other interactions have different non-Abelian symmetries. 
Unless explicitly mentioned, we will consider
relativistic Minkowski space-time with Poincare symmetries.

Yang Mills is a QFT of a ``connection'' described
by $A_\mu^a$ which is the adjoint representation of some compact Lie group $G$.
We can more compactly write this is $A_\mu^a T^a$ which we will see more about soon.

\section{Lie groups}

\begin{defn}
The group $\U\left( N \right)$ consists
of unitary $n\times n$ matrices. 
\end{defn}

This just means that $U^\dag = U^{-1}$.

\begin{prop}
A Lie group also forms a manifold. 
\end{prop}

\begin{defn}
The Lie algebra associated to a Lie group $G$
is the tangent space at the identity:
$T_e G = \fg$.
\end{defn}

\begin{defn}
The adjoint representation of a Lie group is 
the action of group on its own algebra.
\end{defn}

\begin{rmk}
Physicists are sloppy about differentiating between
Lie algebras and Lie groups. 
It is often easiest to figure out which one is being talked about
from the context. 
For example if someone is taking a tensor product $\tp$ of two
objects, they're likely dealing with groups.
If they're taking the sum $\oplus$, they're likely dealing with algebras.
\end{rmk}

Unless stated otherwise, we will be working only with compact Lie groups.


\begin{defn}
A subgroup $H \subseteq G$ is a subgroup such that $H$ is preserved by
the conjugate group action of $G$ on $H$.
\end{defn}

\begin{defn}
A Lie group is \emph{simple} iff
it is connected as a topological manifold, non-abelian,\footnote{
Ruling out $\U\left( 1 \right)$}
and does not have a non-trivial connected normal subgroup.
\end{defn}

\begin{defn}
A Lie group is \emph{semi-simple} iff
it is a product of any finite number of simple Lie groups.
\end{defn}

\begin{exm}
$\SU\left( 2 \right)$ connected and non-abelian, 
and $\U\left( 1 \right)\subseteq \SU\left( 2 \right)$, 
but this subgroup is not normal.
Therefore $\SU\left( 2 \right)$ is simple. 
$\SU\left( 3 \right)$ is as well.
$\U\left( 1 \right)$ is not simple.
\end{exm}

\begin{exm}
The standard model group
$\U\left( 1 \right)\otimes \SU\left( 2 \right) \otimes \SU\left( 3 \right)$
is not semi-simple because of the $\U\left( 1 \right)$ factor.
\end{exm}

\subsection{Killing-Cartan classsification}

This classification of Lie groups
will give us the available choices of Yang-Mills gauge theories. 
In particular, we will have a classification given $A_n$, $B_n$, $C_n$, 
$D_n$, for $n\in \ZZ^+$, 
and then the special  cases $E_{6,7,8}$, 
$F_4$, $G_2$.
This is an extended version of the ADE classification we saw before.

For any Lie group $G$, 
there is a unique Lie algebra $\fg$. 
Now if we work with the Jacobi identities, this
implies a lot of constraining structure, not only on $\fg$, 
but on a certain sub-algebra.

Vector spaces by themselves don't necessarily have a commutation relation. 
This would have to be extra data we provide.
So why does $\fg$ already have this?
Well if we take any vector $v$ in $\fg$, then we can use the group action
to ``parallel transport'' this vector $v$ to any point in the group. 
Now we can define $v\fromto V_L$
which is a left invariant vector field.
In particular, we can define the Lie commutation relation for the Lie algebra
with $\left[ X\left( g \right) , Y\left( g \right) \right]$. 

We now introduce the Cartan sub-algebra. 
For any Lie algebra we have this commutation relation. 
So we might ask whether there is some maximum commuting sub-algebra?
In other words the largest set of elements:
\begin{equation}
\left\{ t_i \in \left\{ T^a \right\} \st
\left[ t_i , t_j \right] = 0
\right\}
\end{equation}
This set, if it exists, is unique up to conjugation.
In any case, for any $\fg$, there is indeed such a sub-algebra $\fh$. 
This will correspond to some subgroup of $G$, 
which will correspond to some particular generators.
Note that this subgroup will always be Abelian by construction.
Now we can consider some coordinate system $e^{i \theta^a T^a}$,
and if we consider only those corresponding to $\fh$,
$e^{i\theta^i t^i}$, then the subgroup generated by this must be 
$\U\left( 1 \right)^{\rank\left( G \right)}$ to some power called the rank of $G$.

Now since $G$ is a finite dimensional manifold, it of course has a dimension,
and now we have defined the rank as well.
For $\SU\left( 2 \right)$, the number of such generators is just $1$. 
Here $\U\left( 1 \right) $ corresponds to the Cartan sub-algebra. 
We also have $\U\left( 1 \right)^2 \subseteq \SU\left( 3 \right)$. 
This notion of rank is important, because it gives us our classification. 

One we have the Cartan subgroup\footnote{
Also called the Cartan ``torus,'' since
$S^1$ is the topology of $\U\left( 1 \right)$.}
we go back to the cartan subalgebra $\fh\subseteq \fg$, 
and call $n = \rank\left( G \right)$.
We now have an $n$-dimensional real subspace inside the real algebra. 
We also know when we exponentiate into the full group, this is a compactification of
the full manifold into a Torus.
This means we need a discrete lattice $\Gamma\subseteq \RR^n$
which will have $n$ basis elements called roots, 
and there will be some geometric structure in place. 
The generic idea is that the structure of this lattice completely determines the groups, 
and therefore we can classify them accordingly.
We sketch a proof.

\begin{proof}[Sketch]
As it turns out, up to overall normalization, 
for all Lie groups, there are only three possible ``lengths'' of the roots.
We call these $l_1 , l_2 , l_3$.

Let's start producing elements of a graphical representation. 
We will associate a vertex with each root, and denote its interaction
with the other roots by the path we draw between them.
Recall for rank $n$ we want to associate $n$ roots.

It also turns out to be the case, 
that only two of these three lengths can show up and be present for 
any particular Lie group. 

There are two conventions we can take now. 
We can either write nodes corresponding to roots
of shorter length as $\circ$, and write the longer length roots
as $\bullet$, or we can denote the longer root with an arrow pointing along
the path between them, towards the longer root.
We take the second conventions.
We also write paths between $l_1$ nodes as a single line. 
We write paths between $l_1$ and $l_2$ as double lines, 
and we write paths between $l_1$ and $l_3$ as triple lines.
These diagrams can be seen in \cref{fig:dynkin}.
\begin{figure}[h!]
\mathtabular{
\begin{tabular}{L L}
A_n&
\dynk{A}{}\\
B_n&
\dynk{B}{}\\
C_n&
\dynk{C}{}\\
D_n&
\dynk{D}{}\\
E_6&
\dynk{E}{6}\\
E_7&
\dynk{E}{7}\\
E_8&
\dynk{E}{8}\\
F_4&
\dynk{F}{4}\\
G_2&
\dynk{G}{2}
\end{tabular}}
\caption{Dynkin diagrams}
\label{fig:dynkin}
\end{figure}
\end{proof}

These seem somewhat esoteric and useless, besides maybe giving us 
a way to classify these group.
But in reality, these can show us quite a bit.
In particular, we can find what are often called ``accidental isomorphisms''
by comparing Dynkin diagrams:
\begin{equation}
B_2 = C_2
\qquad\qquad
\O\left( 5 \right) = \Sp\left( 2 \right)
\end{equation}
\begin{equation}
D_3 = A_3
\qquad\qquad
\O\left( 6 \right) = \SU\left( 4 \right)
\end{equation}

Also note the following dimensions
$\dim E_n = 78,133,248$, 
$\dim F_4 = 52$,
$\dim G_2 = 14$

Recall the ADE classification 
is a special case of this. 
This is because these cases are what are called simply laced. 
This just means that all the nodes are representing roots of the same  ``length''
so there are only single lines.

\section{Generalizing scalar fields}

We understand scalar field theories relatively well. 
We also know we can promote this to a complex scalar field, which
can of course carry a charge. 
So now take $\phi$ to be a complex field.
Now in some sense we promote the 
\begin{equation}
\p_\mu \phi \fromto 
D_\mu\left( A \right)\phi
\left( \p_\mu - i eA_\mu \right)\phi
\end{equation}
When we replace with this covariant derivative, 
we not only make the theory invariant, but we also
make it an interacting theory. 

Now we can easily replace $\phi$ with an $n$ dimensional scalar field 
$\left\{ \phi^i\left(  x\right) \right\}_{i=1}^N$
which has a global $\U\left( N \right)$ symmetry.
We could replace this with any lie group $G$.
Any compact lie group can be used for Yang-Mills quantization.
This is because we need a positive definite metric. 
This metric must be invariant under the group action,
and it's a fact compact lie groups do this.
We write this group action as:
\begin{equation}
U:\phi\left( x \right) \fromto U\phi\left( x \right)
\end{equation}
now we postulate that
\begin{equation}
S = 
\int \p \bar \phi \p \phi \pm
m^2\left( \bar \phi \phi \right) - 
\frac{\lam}{4!} \left( \bar \phi \phi \right)^2
\end{equation}
Now this is clearly a nice scalar field theory, which
is invariant under a global symmetry. 
The notion of a global symmetry here, means a symmetry which acts
regardless of the space-time location. 
These global symmetries can be broken, which means the vacuum state in the field theory
is not preserved by the charges which generate the field theory:
$Q\ket{0} \neq 0$.
If it is not broken, then this leads to an actual physical symmetry
acting on states. 

Now we can ask if we can gauge this global $\U\left( N \right)$ symmetry.
In other words, can we replace this with a symmetry
which is allowed to depend on the space-time location.
So this lie group now becomes a space-time dependent 
function which takes values in $\U\left( n \right)$.\footnote{
This means it becomes a principal section of a bundle.}
This is what is referred to as a local symmetry, or gauge symmetry. 
Recall in the case of the scalar field,
we had to introduce the gauge field at this step.
We now attempt to do the same:
\begin{equation}
\p_\mu \phi  \fromto
\p_\mu\left( U\phi \right) = 
U\p_\mu \phi + \left( \p_\mu U \right) \phi - 
U\left[ \p_\mu \phi + \left( U^\dag \p_\mu U\right) \right]\phi
\end{equation}
So now by brute force we have shown
\begin{equation}
D_\mu \phi^i = 
\p_\mu \phi^i - i A^{i}_{\mu j}\left( x \right)
\phi^i\left( x \right)
\end{equation}
where $A_\mu$ is valued in the adjoint representation
of the group $G$. 

Now we have a Lie group of symmetries, where $U\in G = \U\left( N \right)$. 
So in some small neighborhood of $e$, we can write
\begin{equation}
U\left( x \right) = e^{i\theta^a\left( x \right) T_a}
\end{equation}
where we have used $a$ as the index of the adjoint rep
of $G$, and $\theta$ is a local coordinate. So for $\dim G = n$, 
there will be a basis in the $n$ dimensional 
Lie algebra space, and $a \in \left\{ 1 , \cdots , n \right\}$. 
Often times people just write both $a$ indices as covariant indices, since
when $G$ is a compact Lie group\footnote{
This will always be the case for us.}
there is a unique invariant killing metric,
$g_{ab}$.
There is a theorem which says this is not only positive definite, 
but also diagonalizable, so $g_{ab} = \pm \dd_{ab}$.
So $\theta$ is a set of $n$ coordinate on some patch in the neighborhood of $e$.

So what is the required
transformation rule for $A_\mu$?
We can guess
\begin{equation}
A_\mu \fromto 
A_\mu + i\theta^a \left[ T^a , A_\mu \right] + 
\p_\mu \theta^a\left( x \right) T^a
\end{equation}
which is a generalization of the abelian case for QED.

\begin{wrn}
Note we have hidden the charge $e$.
So where is the coupling constant here?
We have implicitly 
rescaled $A_\mu$ to $A_\mu'$ to absorb whatever gauge coupling shows up.
This will show up in the kinetic term in the Gauge fields instead.
We choose between it showing up here, and showing up in the covariant derivative.
\end{wrn}

We now remind ourselves that $\left[ A_\mu \right] = \left[ \p_\mu \right] = 1$
since the terms in our definition must have the same dimension.
When we get to Yang mills, the only dimensionless coupling is in $4$ dimensions, 
so this is important.

Another important convention distinction is that
in physics:
$A_\mu^P = A_\mu$.
However mathematicians instead define:
$A_\mu^M = -i A_\mu^P$. 
Therefore a mathematician's covariant derivative is:
\begin{equation}
D_\mu = \left( \p_\mu + A_\mu^M \right)
\end{equation}
whereas a physicist has a $-i$ in there. 
This is a reflection of the general trend
that physicists like unitary matrices to be similar to $e^i r$ where $r$ is real.
Then $T^a$ just has to be hermitian.
In the mathematician convention, this just makes $T^a$ anti-hermitian. 
Then the commutation relations become
\begin{equation}
\left[ T^a , T^b \right] = 
i\tensor{f}{^{ab}_c} T^c
\end{equation}
where the $f$ are the structure coefficients which are real. 
So the idea is, to eliminate this $i$, don't introduce it. 
In the physics convention, the ugly $i$ appears in the commutation relation, but not in
\begin{equation}
A_\mu^a \fromto
A_\mu^a + \p_\mu \theta^a - f^{abc}\theta^b A_\mu^c
\end{equation}

\begin{rmk}
Note that exchanging $a,b$ in the $f$s is antisymmetric, and a priori $c$ is unrelated. 
But when we get the metric structure, we get this antisymmetric behavior in all
three indices.\footnote{
See the appendix in \cite{weinberg_2} for details.}
In particular, compact groups automatically have a postive definite metric on their lie algebra, 
and when we raise and lower indices
on compact groups, then $f_{abc}$ are completely antisymmetric, for any simple compact lie group.
\end{rmk}

Note that physicists call $A_\mu^a$ a ``gauge field''
and mathematicians call it a ``component representation of a
connection.''
The idea here, is that what the physicist interprets as a funny one form, 
comes from a very natural mathematical construction:
Consider some manifold $M$.
For us, this will just be $\RR^{3 , 1}$.
Then for any compact $G$, 
define the principal bundle $P$ to be a manifold somehow ``bigger'' than $M$, 
equipped with a projection where the inverse under this projection 
is called a fiber of this principal bundle. 
So far this is any fibration, but a principal bundle is when
each fiber $F_x \simeqq G$. 
So we essentially have a copy of $G$ above every point.
So for every $x$ there is some neighborhood of $x$, $\U_x$, 
such that when we look at the stack of things projecting into this neighborhood,
this is of the form $\U_x\times G$.
So $G$ acts on this big manifold $P$ by right multiplication, which moves us around the 
give fiber. 
Then a space-time dependent gauge transformation would be a section $\Gamma$ of $P$. 
So a connection tells us how to parallel transport. 
Now if the values are in a lie group, then we want to transfer a lie algebra from one location
to another. 

Note that Gauge symmetries are effectively just redundancies in our description. 
We are not saying that $A_\mu \fromto \tilde A_\mu$ are two states, 
we are saying they are two different mathematical descriptions
of the same state. 
So when interested in correlation functions which are invariant under gauge transformations,
it is important to maintain this viewpoint.

Recall in QED, $F_{\mu\nu}$ was gauge invariant. 
Showing the analogous result here is a bit harder.
First of all we will have to define $F_{\mu\nu}$, 
but this will only be gauge covariant, not gauge invariant. 
This just means
\begin{equation}
\sum_a F_{\mu\nu}^a F_{\sigma \rho}^a
\end{equation}
might be invariant.
Until otherwise stated, we take the convention
$A_\mu \equiv A_\mu^M = -i A_\mu^P$. 
It is useful to think of
$A = A_\mu \d{x^\mu}$ as some sort of locally matrix valued one form.
Indeed for each choice of connection, there is a one form called $\om$
whose coordinate transformation on space-time boils down to those of $A_\mu$.
So this is really just a slight abuse of notation.
Now finally we get the transformation law:
\begin{equation}
A \fromto UAU^\dag + U\d{U^\dag}
\end{equation}
\begin{equation}
A_\mu^a \fromto
A_\mu^a 
+ \p_\mu \theta\left( x \right)^a 
- f^{abc} \theta\left( x \right)^b A_\mu^c
\end{equation}
where
$d: \Lam^p M \fromto \Lam^{p+1}M$
and $d^2 = 0$.

Now with this property ,we can check:
\begin{equation}
\d{A} \fromto
U \d{A}U^\dag + \d{U}\wedge A U^\dag - 
U A \wedge \d{U^\dag} + 
\d{U} \wedge \d{U^\dag}
\end{equation}
which is complicated.
So we should really define
$A^a \wedge A^b f^{abc}$:
$\left[ A\wedge A\right]$
to find that
\begin{equation}
F = \d{A} + A\wedge A
\end{equation}
which transforms as
covariantly as possible:
\begin{equation}
F\fromto U F U^\dag
\end{equation}

Another interesting fact, is if we take
$D = d + A$, then
we can show, that $D^2 = F$.
So in the two opposing notational systems we get:
\begin{equation}
F^{\text{math}}_{\mu\nu} = \p_\mu A_\nu - \p_\nu A_\mu + 
\left[ A_\mu , A_\nu \right]
\qquad
F_{\mu\nu}^{\text{physics}} = 
\p_\mu A_\nu - \p_\nu A_\mu - i\left[ A_\mu , A_\nu\right]
\end{equation}
In any case, we can now write down the YM action:
\begin{equation}
S_{\ym} =- \frac{1}{2g_\ym^2}
\int \d{^4x } \tr\left( F_{\mu\nu}F^{\mu\nu} \right)
\end{equation}
where $\tr F^2 \sim g_{ab}F_{\mu\nu}^a F_{\mu\nu}^b$ 
is unique up to a normalization.
This will of course affect what the YM coupling constant will be.
In particular, we have
\begin{equation}
\tr\left( T^a T^b \right) = \# \dd^{ab}
\end{equation}
where $\#$ is determined by about a million different potential conventions.

\section{Quantization}

\subsection{Initial attempts}

In this section, following Faddeev-Popov in \cite{faddeev_popov_ym},
we settle on the convention:
\begin{equation}
\tr\left( T^a T^b \right) = 
2 \dd^{ab}
\end{equation}
which will result in a factor of $1/8$ in the action.\footnote{
One has to simply get used to all of the different conventions here,
and just focus on consistency.}

We start with the same route as we did in QED.
Recall the trick was finding by inspection, that 
\begin{equation}
\lr{A_\mu A_\nu} \sim
\frac{\eta_{\mu\nu}}{p^2 + i\e}
\end{equation}
then this worked if the matter described by this field 
had a conserved current.

More formally, we want to integrate
\begin{equation}
\int \cD A_\mu \exp\left( 
\frac{i}{8g^2} \int \tr\left( F_{\mu\nu}F^{\mu\nu} \right)
\right)
\end{equation}
Now this of course diverges badly. 
So we need to figure out how to quantize systems of this sort. 
Recall that a lesson we learned earlier, is that path integrals
are a good shorthand for quantizing a system,
but only if we already know how to quantize in the Hamiltonian formalism.

Recall that we worked to get the following (naive) compact expression:
\begin{equation}
\cZ = \int \cD q\left( t \right)
\exp\left( i\int \d{t} \cL \left( q , q' , t \right) \right)
\end{equation}
but this sometimes required corrections. 
But the Hamiltonian formalism gave us the (more) correct expression:
\begin{equation}
\cZ = 
\int \cD p\left( t \right) \cD q\left( t \right)
\exp\left( i\int \d{t} \left( p q' - H \right) \right)
\end{equation}
which sometimes reduces to the same in the coordinate language, but not always.

\begin{rmk}
It is a beautiful and fundamental statement in general,
that there is a canonical measure in phase space.
In particular, we can take the exterior derivative:
\begin{equation}
d\left( p_i \d{q^i} \right) = \d{p_i} \ext \d{q^i} = \om
\end{equation}
where $\om$ is a symplectic form, and
$\om^n$ is a volume form.
\end{rmk}

So now we fix the following:
\begin{equation}
\cA_\mu \ceqq A_\mu^a T^a
A_\mu \fromto \cA_\mu^\om \ceqq
\omega\left( x \right) \cA_\mu \omega^{-1}\left( x \right) + 
\p_\mu \om\left( x \right)\om^{-1}\left( x \right)
\end{equation}
\begin{equation}
S = \frac{1}{8g^2}\int \tr\left( F_{\mu\nu}F_{\mu\nu} \right)
\qquad\qquad
F_{\mu\nu}= 
\p_\nu A_\mu - \p_\mu A_\nu + 
\left[ A_\mu , A_\nu\right]
\end{equation}

Then the equations of motion are:
\begin{equation}
D^\mu F_{\mu\nu} = 0
\end{equation}
Note that this is where we notice there is a non-invertibility of the 
Gaussian term because we see we don't have linearly independent equations of motion.
Formally, this is because $D^\mu D^\nu F_{\mu\nu}\equiv 0$.

People like Feynman\footnote{In the early 60s, Feynman was attempting to treat Yang-Mills
quantization as he had treated QED, and was shocked by the non-unitarity of the $S$-matrix
under the tricks which worked for QED.}
wanted to postulate that the propagator should still just be
\begin{equation}
D_{\mu\nu} = \frac{\eta_{\mu\nu}}{p^2 + i\e}
\end{equation}
After all this worked fine in QED, because we could just ignore the non-invertible portion
of the propagator.

We now make our way towards the Hamiltonian formalism.
We postulate that the Lagrangian takes the form:
\begin{equation}
\cL = E_k^a \p_0 A_k^a - h\left( E_k , A_k \right) + A_0^a C^a
\end{equation}
Inspired by the analogy $E_k^a, A_k^b \sim p_i q^i$, and
\begin{equation}
h = \frac{1}{2} \left\{ 
\left( E \right)^2 + 
\left( G \right)^2
\right\}
\end{equation}
Specifically, we rewrite the trace:
\begin{equation}
\tr\left\{ \left( \p_\nu \cA_\mu - \p_\mu \cA_\nu 
+ g\left[ A_\mu , A_\nu \right] - \frac{1}{2} \cF_{\mu\nu}\right)
\right\}
\end{equation}
Up to a total derivative, this will become a Hamiltonian form:
\begin{equation}
-\frac{1}{2}\tr\left\{ 
E_k \p_0 \cA_k - \frac{1}{2}\left( 
E_j^2 + G_k^2
\right) + \cA_0 \cC
\right\}
\end{equation}
where
\begin{equation}
E_k = \cF_{k0}
G_k = \frac{1}{2} \e^{ijk}\cF_{ji}
h = \frac{1}{2} \left\{ \left( E_k^a \right)^2 + 
\left( G_k^a \right)^2\right\}
\end{equation}
Note that $\cC$ is a Lie algebra valued object such that
\begin{equation}
\cC = \p_k E_k - g\left[ \cA_k , E_k \right]
\qquad\qquad
\cC = C^a T^a
\end{equation}
this $h$ will be called a Hamiltonian
even though is not a proper Hamiltonian as we are used to.
Specifically, this is a degenerate Lagrangian, 
and therefore doesn't lead to an actual Hamiltonian.
Therefore we need to apply Dirac's 
more sophisticated Hamiltonian treatment. 

We have the canonical pair $E$,$A$ so we get Poisson brackets:
\begin{eqnarray}
\left\{ E^a_k \left( x^i \right) , E_l^b\left( x^i \right) \right\}_{\PB}
&=&  \dd_{kl} \dd^{ab} \dd\left( x^i - y^i \right)
\\
\left\{ C^a\left( x^i \right) , C^b\left( y^i \right) \right\}_{\PB} &=&
g f^{abc} C^c\left( x \right) \dd\left( x^i - y^i \right)
\end{eqnarray}

Our system is a generalized Hamiltonian system in the dense of Dirac.
We now try to figure out how to treat them.

\subsection{Dirac's formalism}

Consider a general system with a finite number of degrees of freedom.
Then there is a classical system $\Gamma$ with $n$ degrees of freedom,
and phase space $\Gamma^{2n}$ of dimension $2n$ with coordinates $p_i , q_i$
for $i\in \left\{ 1 , \cdots , n \right\}$. 
The equation of motion is $\phi^\al\left( p , q \right) = 0$,
and we postulate that the action of this system is
\begin{equation}
S = \int \left\{ \sum_{i = 1}^n
p_i q_i' - h\left( p , q \right)
- \sum \lam^\al \phi^\al \left( p , q \right) 
\right\}\d{t}
\end{equation}
where $\al\in \left\{ 1 , \cdots , m \right\}$ for $m < n$.
We now assume two things. 
We already know $p,q$ satisfy canonical Poisson brackets. 
We also have
\begin{eqnarray}
\left\{ h , \phi^\al \right\}_{\PB} &=&  
C^{\al \b} \left(  p ,q \right) \phi^\b
\\
\left\{ \phi^\al , \phi^\b \right\}_{\PB} &=& 
C^{\al \b \gamma} \left(  p , q \right) \phi^\gamma
\end{eqnarray}
We will such a system, a system with first class constraints. 
It is somewhat easy to see that Yang Mills is such a theory.

First class constraints are:
\begin{equation}
\left\{ h , \phi^\al \right\} = 
C^{\al\b}_{\left( p,q \right)}\phi^\b
\qquad
\left\{ \phi^\al , \phi^\b \right\} = 
\sum C^{\al \b \gamma}_{\left( p,q \right)} \phi^\gamma
\end{equation}
Second class constraints are
\begin{equation}
\left\{ \phi^\al , \phi^\b \right\} \sim \e^{\al \b}
\end{equation}
We will be focusing on first class constraints. 
For more information on the general theory, see \cite{teitelboim_quantization}.

\begin{clm}
The system $\Gamma$, is equivalent to a simpler looking system $\Gamma^*$
with $n - m$ canonical pairs (degrees of freedom).
\end{clm}

The phase space  $\Gamma^{*2\left( n-m \right)}$ can be constructed as follows. 
\begin{enumerate}
\item In addition to the data we already have,
pick $m$ additional conditions $\chi^a\left( p , q \right) = 0$.
These can be anything as long as:
\begin{equation}
\det{\left\{ \phi^\al , \phi^\b \right\}} \neq 0
\qquad\qquad
\left\{ \chi^\al , \chi^\b \right\} = 0
\end{equation}
We will see why these conditions are useful as we develop the reduced phase space.
\item Identify $2m$ conditions, which is that
we force
\begin{equation}
\chi^\al = \phi^\al = 0
\label{eqn:step_two}
\end{equation}
This gives $\Gamma^{*2\left( n-m \right)}$.
\item Find canonical coordinates $p^*$, $q^*$.
We will call this an ``adapted coordinate system.''
We will have
\begin{equation}
q = \left( \chi^\al , q^* \right)
\qquad\qquad
p = \left( p^\al , p^* \right)
\end{equation}
Now in these variables, if we look at the first condition in the last step, 
we see that as a consequence of this, we have
\begin{equation}
\det{\abs{\frac{\p \phi^*}{\p p^\b}}}\neq 0
\end{equation}
\end{enumerate}

In principal, we can now invert $\phi^\al\left( p , q \right) = 0$ 
to solve for $p^\al = p^\al\left( p^* , q^* , \chi^\al \right)$. 
So this is not an independent momentum anymore.
We now do this manually.
So we have defined $\Gamma^{*2\left( n-m \right)}$ such that
$\chi^\al = \phi^\al = 0$. 
So this means $p^* , q^*$ are a good set of canonical coordinates, since
$p^\al = p^\al\left( p^* , q^* \right)$. 
So the hamiltonian:
\begin{equation}
h^*\left( p^* , q^* \right) = \restr{h\left( p , q \right)}
{\phi = \chi = 0}
\end{equation}
So now we just need to calculate $\cZ$ with the measure replaced with our new canonical coordinates.
We now prove that the original $\Gamma$ with constraints is the same system as $\Gamma$
without:
\begin{proof}
We know the equation of motion for $\Gamma$ is:
\begin{equation}
p_i' + \frac{\p h}{\p q_i} + 
\lam^\al \frac{\p\phi^\al}{\p q_i} = 0
\end{equation}
\begin{equation}
q_i' + \frac{\p h}{\p p_i} - 
\lam^\al \frac{\p \phi^\al}{\p p_i} = 0
\end{equation}
and $\phi^\al = 0$. 
So now solving for $\lam^\al\left( t \right)$, 
we note that $\lam^\al$ are arbitrary, but imposing $\chi^\al = 0$ fixes
this. Of course in our usual language, $\chi^\al$ is a gauge fixing condition.
Now the standard hamiltonian of $\Gamma^*$ is:
\begin{equation}
q'^* = 
\frac{\p h^*}{\p p^*}
\qquad\qquad
p'^* = 
\frac{\p h^*}{\p q^*}
\end{equation}
now go to the adapted system with $\Gamma$. 
Specifically focus on the $q^\al$ equation of motion to get
\begin{equation}
\frac{\p h}{\p p_\al} + 
\lam^\al \frac{\p \phi^\b}{\p p^\al} = 0
\end{equation}
Now it is important to notice that \eqref{eqn:step_two}
allows us to express $\lam^\al$ uniquely. 
Now moving onto the equations of motion for the remaining $q^*$,
we get that in $\Gamma$:
\begin{equation}
q'^* = 
\frac{\p h}{\p p^*} + 
\lam^a \frac{\p \phi^\al}{\p p^*} = 0
\end{equation}
And in $\Gamma^*$:
\begin{equation}
q'^* = \frac{\p h^*}{\p p^*} = 
\frac{\p h}{\p p^*} + 
\frac{\p h}{\p p^\al} 
\frac{\p h}{\p p^*}
\end{equation}
now when we set these equal, we get some condition where
we have dropped relative signs, to get:
\begin{equation}
\lam^\al\left( \frac{\p \phi_\al \phi}{\p p^*} + 
\frac{\p \phi_\al}{\p \phi_\b}
\frac{\p \phi_\b}{\p \phi^*}\right)
\end{equation}
so we are done for $q$, and we leave the $p$ case as an exercise.
\end{proof}

We could apply this directly to YM, but let's continue in this more familiar language of QM. 
For $\Gamma$, we define the path integral:
\begin{equation}
\cZ = \int
\cD p^*\left( t \right) \cD q^*\left( t \right)
\exp\left( 
i\int p^* q'^* - h^*\left( p^* , q^* \right)
\right)
\end{equation}
now we know this is perfectly accurate, by the preceding discussion, so the statement is:

\begin{prop}
\begin{eqnarray}
\cZ &=&  \int \cD p\left( t \right) \cD q\left( t \right)
\cD \lam\left( t \right)
\prod_{t , \al} \dd\left( \chi^\al \right)
\prod_{t} \det \abs{ \left\{ \phi_\al , \chi_\b \right\}}
\\
&& \exp\left( i\int p q' - h\left( p , q \right) - \lam^\al \phi^\al\left( p , q \right) \right)
\end{eqnarray}
The determinant is called the Faddeev-Popov determinant.
\end{prop}

\begin{proof}
First we integrate over $\lam$ which imposes the constraint conditions. 
So now we are on the constraint surface $\phi = 0$.
So this gives us something looking somewhat like
\begin{equation}
\int \cD p \cD q \exp\left( i\int pq' - h \right)
\prod \dd\left( \phi^\al \right) \dd\left( \chi^\al \right)
\prod \det\abs{ \left\{ \phi^\al , \chi_\b \right\}}
\end{equation}
now go to the coordinates
$\left( \chi^\al , q^* \right),
\left( p\left( p^* , q^* \right), p^* \right)$. 
Recall that on this constraint surface, 
\begin{eqnarray}
\prod \dd\left( \phi^\al \right) \dd\left( \chi^\al \right)
\prod \det\abs{ \left\{ \phi^\al , \chi_\b \right\}}
&=& \prod \dd\left( \phi_\al \right)\dd\left( q_\al \right)
\det \abs{\frac{\p\phi_\al}{\p p_\b}}
\\
&=& \prod \dd\left( q_\al \right) \dd\left[ 
p_\al - p_\al\left( p^* , q^* \right)\right]
\end{eqnarray}
Now by inspection, we can integrate out $p_\al$ and $q_\al$, to get the 
expected expression.
\end{proof}

\subsection{YM as a theory with second class constraints}

We can now apply this picture to the actual YM theory. Recall that
\begin{eqnarray}
\left\{ E_k^a\left( x^i \right) , A_l^b\left( y^i \right) \right\} &=& 
\dd_{kl} \dd^{ab} \left( x^i - y^i \right)
\\
\left\{ C^a\left( x^i \right) , X^b\left( y^i \right) \right\} &=& 
g f^{abc} C^c\left( x^i \right) \dd\left( x^i - y^i \right)
\end{eqnarray}
This means we can calculate:
\begin{equation}
\left\{ \int \d{^3 x}\left[ 
\left( E_k^a \right)^2 + 
\left( G_k^a \right)^2
\right] , C^b\left( y^i \right) \right\} = 0
\end{equation}
so this is indeed a first class set of constraints. 

So now we do step one, and fix the Coulomb gauge $\p_k A_k = 0$.
\begin{wrn}
It is tempting to set the Gauge fixing condition to be the Lorentz gauge
$\p^\mu A_\mu = 0$ but this  does not only depend on $p,q$, it also depends on the Lagrange
multiplier. So this is out for now.
\end{wrn}
This also satisfies the conditions of the second step, because clearly
\begin{equation}
\left\{ \p_k \cA_k , \p_i \cA_i \right\} = 0
\end{equation}
Now we have to calculate:
\begin{equation}
\left\{ C^a\left( x^i \right) , \p_k A_k^{\left( b \right)}\left( y \right) \right\} = 
-\p_k \left[ 
\p_k \dd^{ab} - g f^{abc} A_k^c\left( x \right)
\right]\dd\left( x - y \right)
= M_c\left[ \cA \right]
\end{equation}
So we can develop a perturbative expansion in $g$. 
The leading term here will be $\pm\p_k \p_k \dd^{ab}$, so
this is an invertible matrix.
Note that we don't have Lorentz invariance at this point, but we still have the full
expression:
\begin{multline}
\cZ = \int
\cD E_i^a\left( x \right) \cD A_\mu^a\left( x \right)
\exp\left( 
i \frac{1}{8 g_{\ym}^2}\int \d{^4 x} \tr\left( F_{\mu\nu} F^{\mu\nu} \right)
\right)
\\
\prod \dd\left( \p_k \cA_k \right)\prod\det M_c\left[ \cA \right]
\ .
\end{multline}

\begin{wrn}
Still treat $E_i$ and $A_\mu$ as independent variables.
Recall, that luckily, the $E_i$ factors only appear 
in (up to) quadratic terms.
\end{wrn}

\begin{exr}
Take the full Hamiltonian formula, and see that plugging in 
Yang-Mills gives us this expression.
\end{exr}

This $M_c$ matrix is an operator (since we're talking about field theory after all).
In particular, 
\begin{equation}
M_c = \Delta \dd^{ab} - g f^{abc} A^c_k \p_k
\end{equation}
which is the analogue of what would just be $\p_\mu\p^\mu$ is a scalar $\phi^4$ theory.
Importantly, this expression depends on $A$. 
We have two conclusions from this:
\begin{enumerate}
\item QED works without fully understanding the nature of this determinant. This is 
just a lucky coincidence. As we will see, to treat the FP determinant in full generality
requires a fair bit more work. 
\item If we want to avoid viewing this pesky determinant, 
there is a clear path to avoiding it completely. 
In particular, we can come up with some gauge fixing, 
such that the matrix is independent of the Gauge field $A$.
\end{enumerate}

\section{Gauge fixing in YM}

The general point of this section, is to find
how we can move from the Coulomb gauge fixing $\p_i \cA_i = 0$
to the Lorentz gauge fixing $\p_\mu \cA^\mu = 0$, since we obviously
want our theory to be Lorentz invariant.

To understand the relationship between the theory and the Gauge fixing
we choose, it is convenient to introduce an object $\Delta_L$
called the Faddeev-Popov (FP) determinant. This is some object associated to 
a given Gauge fixing condition $L$, such that:
\begin{equation}
\Delta_L\left( \cA \right) 
\int \prod_x \dd\left( \p_\mu \cA_\mu^\om \right)\d{\om} = 1
\end{equation}
Note that we write $\p_\mu \cA_\mu \equiv \p_\mu \cA^\mu$ in order to reserve the
upper index location to denote more features. In this case we use it to denote the 
the group action. 
So here, $\cA_\mu^\om$ is the gauge transformation of $\cA_\mu$
under a local group element $\om$:
\begin{equation}
\cA_\mu^\om = \om^{-1} \cA_\mu \om + \om^{-1} \p_\mu \om
\end{equation}
In the above definition of $\Delta_L$, we take $\d{\om}$ to be any
$G$ invariant measure on $G$. This is known as the Haar measure. 
In particular, we just define it to be whatever measure satisfies:
\begin{equation}
\d{\om \om_0} = \d{\om} = \d{\om_0 \om}
\end{equation}
So this resolution of the identity is given by an integral over the group $G$, 
and spacetime.

Now we have the following statement about $\Delta_L$:

\begin{prop}
$\Delta_L \left( \cA \right)$ is gauge invariant.
\end{prop}

\begin{proof}
We can write:
\begin{eqnarray}
\Delta_L\left( \cA^{\om'} \right)^{-1} &=&  
\int\prod \dd\left( \p_\mu \cA_\mu^{\om' \om} \right) \d{\om}\\
&=& \prod \dd\left( \p_\mu \cA^{\tilde \om}\right)\d{\left( \om'^{-1} \tilde \om \right)}\\
&=& \prod \dd\left( \p_\mu \cA^{\tilde \om}\right)\d{\left( \tilde \om \right)}\\
&=& \Delta_L\left( \cA \right)^{-1}
\end{eqnarray}
where we have taken $\tilde \om \ceqq \om' \om$.
\end{proof}

We also have the following:

\begin{prop}
On $a$ a gfc surface, 
\begin{equation}
\prod \det M_c\left[ \cA \right] = \Delta_c\left( \cA \right)
\end{equation}
\end{prop}
\begin{proof}
When $\p_k \cA_k = 0$, we have $\om = e$.
So in order to perform this integral, we only need to
concentrate on the vicinity of $e$. 
In other words, we can perform this only over elements of the Lie algebra. 
So write $\om \approx e + u\left( x \right)$ for $u$ some small deviation.
So we need to evaluate $\p_k \cA_k^\om$ for such an $\om$. 
But this is just
\begin{equation}
\p_k \cA_k^\om = 
\Delta u - g\left[ \cA_k , \p_k u \right] = 
M_c\left[ \cA \right]
\end{equation}
Now evaluating this integral gives us
$\Delta_X\left( \cA \right) = \det \left( M_c\left( \cA \right) \right)$
so we are done.
\end{proof}

Now taking the resolution of the identity from the definition of $\Delta_L$, 
we can rewrite the path integral as:

\begin{eqnarray}
\cZ &=&  \int \cD \cA_\mu\left( z \right) \Delta_L\left( \cA \right) \int \prod \dd\left( \p_\mu \cA^\om_\mu
\right)\d{\om}\\ 
&& \exp\left( 
\frac{i}{8g_{\ym}^2} \int \d{^4x} \tr\left( F_{\mu\nu}F^{\mu\nu} \right)\right)
\prod \dd\left( \p_k \cA_k \right) \Delta_c\left( \cA \right)\\
&\fromto &  \int \cD \cA_\mu\left( z \right) \Delta_L\left( \cA \right) \int \prod \dd\left( \p_\mu \cA_\mu
\right)\d{\om}\\
&&\exp\left( 
i\frac{1}{8g_{\ym}^2} \int \d{^4x} \tr\left( F_{\mu\nu}F^{\mu\nu} \right)\right)
\prod \dd\left( \p_k \cA_k^{\om^{-1}} \right) \Delta_c\left( \cA \right)\\
&=& \int \cD_\mu\left( x \right) \Delta_L
\prod \dd\left( \p_\mu \cA^\mu \right)
\exp\left( \frac{i}{8 g_\ym^2}\int \d{^4x} \tr\left( F_{\mu\nu}F^{\mu\nu} \right) \right)
\end{eqnarray}
where we brought $\cA_\mu \fromto \cA_\mu^{\om^{-1}}$
in the second line.

We now have two more tricks:
\begin{enumerate}
\item The first trick will bring us one step closer to Feynman rules. 
We notice that we are falling short in two major ways.
First of all, we can't really work with the FP determinant perturbatively yet.
We also don't really want to work directly with the product of delta distributions 
in our path integral expression. In other words, 
we want to write this $\cZ$ as an integral of the form:
\begin{equation}
\cZ = \int \cD \cA_\mu\left( x \right) \cdots
\exp\left( i S_{\eff} \left( \cA , \cdots \right) \right)
\end{equation}
where we might need to introduce additional integration variables.
So instead of $\p^\mu \cA_\mu^a = 0$, we take
$\p^\mu \cA_\mu^a = \cB^a \left( x \right)$
where $\cB$ is called an auxilary field.
This lets us rewrite:
\begin{eqnarray}
\cZ &=&  \int \cD_\mu\left( x \right) \cD B^a\left( x \right)
\Delta_L\left( \cA \right)
\prod \dd\left( \p_\mu \cA^\mu  - \cB \right)\\ 
&& \exp\left( \frac{i}{8 g_\ym^2}\int \d{^4x} \tr\left( F^2 \right) \right)
\exp\left( i\chi \int \tr\left( \cB \right) \d{^4 x} \right)
\end{eqnarray}
Note that if we integrate over $\cB$ we get back to the delta function form. 

\item The second trick is to notice that the determinant of any bosonic matrix like this, 
can be written like:
\begin{equation}
\det M_x\left( \cA \right) = \int \cD c\left( x \right) \cD b\left( x \right)
\exp\left( i \int \d{^4 x} \left( b\left( x \right) M_c\left[ \cA \right] 
c\left( x \right) \right) \right)
\end{equation}
where $b,c$ are Grassmannian variables with the same field structure as the gauge fixing condition.
The field $c$ is called a \emph{ghost} and the field $b$ is an \emph{anti-ghosts}.
The YM notation often takes $c$ as the ghost
and $\bar c$ as the anti-ghost.
We sometimes write it this way.
\end{enumerate}

\subsection{Alternative gauge choices}

The Gauge fixing we assumed above a priori lets us think
the ghost anti-ghost don't decouple from
the gauge field. 
Can we find gauges where we decouple $b,c$?

For $\U\left( 1 \right)$ gauge theories, they were already decoupled.
For non-abelian theories, we will get the wrong result from leaving off
the ghost terms.
For non-abelian theories there is a ebatufiul class of ``axial'' gauges
In general, they rely on a choice of a constant vector $v^\mu$.
This breaks Lorentz since we are choosing a direction in sopacetime. 
Then the gauge condition is just
$v^\mu A_\mu^a = 0$. 

\begin{exm}
If we take $v^\mu = \left( 1, 0,0,0 \right)$, then 
$A_0^a = 0$ which is often called a ``temporal'' gauge.

We can also take $v^\mu$ such that $v^\mu v_\mu = 0$. 
This condition will be called ``lightcone'' gauge.\footnote{
This should really be called infinite momentum gauge.}

Generically, if $v$ is spacelike, then this is referred to as ``axial proper.''
People often pick the third direction, so they set $A_3^a = 0$ which simplifies 
Feynman rules. 
\end{exm}

Let's now derive the propagator in an axial gauge:
\begin{equation}
\Pi^{ab\mu\nu}_{\text{axial}} = 
\frac{1}{p^2 + i\e}
\left[ -\eta^{\mu\nu}+ 
\frac{v^\mu p^\nu + v^\nu p^\mu}{vp} - 
\frac{\left( v^2 + \xi p^2 \right)p^\mu p^\nu}{\left( vp \right)^2}\right]\dd^{ab}
\end{equation}
Now observe that this satisfies
$p_\mu \Pi^{\mu\nu} - 9$ for $p^2 = 0$. 
In addition, if we take $\xi = 0$, then $v_\mu \Pi^{\mu\nu} = 0$. 
So
\begin{equation}
\begin{feyn}
b && \\
& \cdot\arrow[ghost]{ul}&
\mu;c\arrow[gluon]{l} \\
a\arrow[ghost]{ur} &&
\end{feyn}\sim
v^\mu f^{abc}
\end{equation}
so ghosts decouple, but in a more subtle way than in QED.

Alternative gauge choices:
The Gauge fixing we assumed above a priori lets us think
the ghost anti-ghost don't decouple from
the gauge field. 
Can we find gauges where we decouple $b,c$?

For $\U\left( 1 \right)$ gauge theories, they were already decoupled.
For non-abelian theories, we will get the wrong result from leaving off
the ghost terms.
For non-abelian theories there is a ebatufiul class of ``axial'' gauges
In general, they rely on a choice of a constant vector $v^\mu$.
This breaks Lorentz since we are choosing a direction in sopacetime. 
Then the gauge condition is just
$v^\mu A_\mu^a = 0$. 

\begin{exm}
If we take $v^\mu = \left( 1, 0,0,0 \right)$, then 
$A_0^a = 0$ which is often called a ``temporal'' gauge.

We can also take $v^\mu$ such that $v^\mu v_\mu = 0$. 
This condition will be called ``lightcone'' gauge.\footnote{
This should really be called infinite momentum gauge.}

Generically, if $v$ is spacelike, then this is referred to as ``axial proper.''
People often pick the third direction, so they set $A_3^a = 0$ which simplifies 
Feynman rules. 
\end{exm}

Let's now derive the propagator in an axial gauge:
\begin{equation}
\Pi^{ab\mu\nu}_{\text{axial}} = 
\frac{1}{p^2 + i\e}
\left[ -\eta^{\mu\nu}+ 
\frac{v^\mu p^\nu + v^\nu p^\mu}{vp} - 
\frac{\left( v^2 + \xi p^2 \right)p^\mu p^\nu}{\left( vp \right)^2}\right]\dd^{ab}
\end{equation}
Now observe that this satisfies
$p_\mu \Pi^{\mu\nu} - 9$ for $p^2 = 0$. 
In addition, if we take $\xi = 0$, then $v_\mu \Pi^{\mu\nu} = 0$. 
So
\begin{equation}
\begin{feyn}
b && \\
& \cdot\arrow[ghost]{ul}&
\mu;c\arrow[gluon]{l} \\
a\arrow[ghost]{ur} &&
\end{feyn}\sim
v^\mu f^{abc}
\end{equation}
so ghosts decouple, but in a more subtle way than in QED.

\section{Feynman rules}

We now use the preceding developments to attempt to formulate Feynman rules
for our YM theory.
Our most ``Feynman diagram friendly'' form of the path integral is:
\begin{eqnarray}
\cZ &=&  \int \cD \cA_\mu\left( x \right) \cD b\left( x \right) \cD c\left( x \right) \cD \cB\left( x \right)
\dd\left( \p_\mu \cA^\mu - \cB \right)
\\ && \exp\left( 
i\int \d{^4 x}\left\{ \frac{1}{2} \tr\left( F_{\mu\nu}F^{\mu\nu} \right) + 
\xi \tr\left( \cB^2 \right) + i b M_c\left( \cA \right)c\right\}\right)
\end{eqnarray}
where one often defines:
\begin{equation}
S_\ym = \tr\left( F^{\mu\nu}F_{\mu\nu} \right)
\qquad\qquad
S_{\text{gf}} = \xi \tr\left( B^2 \right)
\end{equation}
Here where $\xi$ is called a gauge fixing parameter. 
If we are ever trying to make some sort of physical prediction, it is always a good check
to assure that this result has no $\xi$ dependence, since no results should depend on any
Gauge fixing condition.

Propagator for Gauge fields $A_\mu$ (``gluons'') is given by
\begin{equation}
\begin{feyn}
\nu;b \arrow[gluon]{r}{p}\arrow[label,shift right=6pt]{r}&
\mu;a
\end{feyn}=
\frac{i}{p^2 + i\e}
\left( -\eta^{\mu\nu} + \left( 1 - \xi \right) p^\mu p^\nu / p^2\right)
\dd^{ab}
\end{equation}
Notice that the gluon propagator is the photon propagator with
this extra $\dd^{ab}$ factor. 

The scalar propagator is given by:
\begin{equation}
\begin{feyn}
j\arrow[scalar]{r}{p}&i
\end{feyn}
=\frac{i\dd^{ij}}{p^2 - m^2 + i\e}
\end{equation}
The fermionic propagator is given by:
\begin{equation}
\begin{feyn}
j\arrow[r,"p",fermion]&i
\end{feyn}
=\frac{i\dd^{ij}}{\fsl{p} - m + i\e}
\end{equation}
The propagator for the ghost fields $b$ and $c$ is given by:
\begin{equation}
\begin{feyn}
\nu ; b\arrow[ghost]{r}{p}&
\mu ; a
\end{feyn}=
\frac{i\dd^{ab}}{p^2 + i\e}
\end{equation}

For the triple gluon vertex, the derivative can act
on any of the gluons, so we have the Feynman rule:
\begin{equation}
\begin{feyn}
&&\mu ;a \arrow[gluon]{dl}{k}\arrow[label, shift right=6pt]{dl} \\
\nu ; b \arrow[gluon]{r}{p}\arrow[label,shift right=6pt]{r}& \cdot & \\
&& \rho; c \arrow[gluon]{ul}{q}\arrow[label, shift right=6pt]{lu}
\end{feyn}
= gf^{abc}\left[ 
g^{\mu\nu} \left( k - p \right)^\rho + 
g^{\nu \rho}\left( p - q \right)^\mu + 
g^{\rho \mu}\left( q - k \right)^\nu\right]
\end{equation}
Note that since we take all the momentum incoming, $p + k + q = 0$. 

The four-gluon vertex gives us:
\begin{eqnarray}
\begin{feyn}
\mu;a\arrow[gluon]{rd}&
&
\nu;b\arrow[gluon]{ld} \\
&\cdot & \\
\rho ; c \arrow[gluon]{ru}
& & \sigma; d\arrow[gluon]{ul}
\end{feyn}
\begin{aligned}
\\=  -i g^2 \times 
\bigg[ & f^{abe}f^{cde}\left( g^{\mu\rho}g^{\nu\sigma} - g^{\mu\sigma} g^{\nu\rho} \right) \\
+& f^{ace} f^{bde}\left( g^{\mu\nu} g^{\rho\sigma} - g^{\mu\sigma} g^{\nu\rho} \right) \\
+& f^{ade}f^{bce}\left( g^{\mu\nu} g^{\rho\sigma} - g^{\mu\rho} g^{\nu\sigma} \right)\bigg]
\end{aligned}
\end{eqnarray}
The following ghost vertex is given by the rule:
\begin{equation}
\begin{feyn}
& \mu ; b \arrow[gluon]{d} & \\
& \cdot\arrow[ghost]{dr} & \\
c^c \arrow[ghost]{ur}&
&
b^a
\end{feyn}
= -g f^{abc} p^\mu
\end{equation}
The vertex for the Fermion interaction is given by:
\begin{equation}
\begin{feyn}
& \mu ; a \arrow[gluon]{d} & \\
& \cdot\arrow[fermion]{dr} & \\
j \arrow[fermion]{ur}&
& i
\end{feyn}=
i g\gamma^\mu T_{ij}^a
\end{equation}
Notice that this is the same as in the case of QED where the orientation
of the vertex in the actual diagram does not matter. 

Finally we have the scalar vertices:
\begin{eqnarray}
\begin{feyn}
& \mu ; a \arrow[gluon]{d}{p}\arrow[label, shift right = 5 pt]{d}&\\
& \cdot \arrow[scalar]{dr}{q} & \\
j\arrow[scalar]{ur}{k}&
& i
\end{feyn}
&=& ig\left( k^\mu + q^\mu \right) T_{ij}^a\\
\begin{feyn}
\mu ; b \arrow[gluon]{rd} & & \nu ; a \arrow[gluon]{ld} \\
& \cdot \arrow[scalar]{rd} & \\
j\arrow[scalar]{ru} & & i
\end{feyn} &=& 
ig^2 T_{ik}^a T_{kj}^b g^{\mu\nu}
\end{eqnarray}

\subsection{Representation theory}

If we recall the loop corrections calculated
in QED and QCD, it turns out that the more general
YM corrections are the same up to some factors. We now lay some brief
group theoretic formalism pertaining to these factors.

Define some object $T$, for a given representation, such that
\begin{equation}
\tr\left( T^a_{R}  T^b_{R}\right) = T\left( R \right) \dd^{ab}
\end{equation}
Then this implies that
\begin{equation}
T\left( \text{fund} \right) = T_F = \frac{1}{2}
\end{equation}
that is, $T^a_{ji} T^b_{ij} = 1/2 \dd^{ab}$. 
For the adjoint representation we have
\begin{equation}
T\left( \text{adj} \right) = T_A = N
\end{equation}
that is, $f^{acd} f^{bcd} = N \dd^{ab}$. 

The quadratic Casimir:
Instead of taking the $\tr$ over the d dimensional space of the representation, we can take
the product in some representation:
\begin{equation}
T_{R}^a T_{R}^a = 
C_2\left( R \right)
\one_{R}
\end{equation}
Then it is easy to check that
the following is true:
\begin{equation}
\dim\left( R \right)C_2\left( R \right) = T\left( R \right)
\dim\left( G \right)
\end{equation}
We will call $C_2\left( F \right) = C_F = N^2 - 1\left( 2N \right)$
and $C_2\left( A \right) = C_A = N$.
These relations are used in effectively every 
QCD calculation.

\subsection{One-loop corrections}

We now want to calculate the $\b$ function to show QCD with $\SU\left( N \right)$
is asymptotically free. 
So we want to calculate all of the one loop corrections we can think of. 
First the coulomb propagator, but also for
\begin{equation}
\begin{feyn}
&& {} \\
{} \arrow[ghost]{r} & \cdot \arrow[fermion]{ur} & \\
&& {} \arrow[fermion]{ul}
\end{feyn}
\end{equation}
Note we will be using the renormalized perturbation theory to set this calculation up. 
We have seen a good amount of this. We start  by writing the bare, field, and write
the Lagrangian using this bare field, bare mass, and the coupling as bare couplings. 
Then we introduce a renormalized field, a renormalized mass, and renormalized couplings. 
So we get a bunch of counterterms. 
So here we also do this. We denote these counterterms by
\begin{equation}
\begin{feyn}
\ \arrow[gluon]{r}&
\tp \arrow[gluon,r]& \
\end{feyn}
\end{equation}
We now apply the methods of dimensional regularisation.
Commit to spacetime dimension $d = 4 - \e$,
and a renormalization group scale such that $\left[ \mu \right] = 1$.
It will turn out to be useful to also use $\tilde\mu \ceqq 4\pi e^{-\gamma_e}\mu$
where $\gamma_e$ is the Euler constant.
We now have:
\begin{equation}
S\sim - \frac{1}{2g_\ym^2}\int \tr \left( F_{\mu\nu}F^{\mu\nu} \right)\d{^d x}
\end{equation}
so we peel off the dimensional dependence, 
by taking the dimensionful $\tilde g_ym = \mu^{\left( 4 - d \right)/2} g_\ym$
since $g_\ym$ is dimensionless.
We know $\xi$ should drop out in the end, but while making calculations,
will write $\xi = 1$ and restore this in the end.
The propagator is extremely simple:
\begin{equation}
\Pi^{ab \mu\nu} = \dd^{ab}
\frac{-i \eta^{\mu\nu}} {p^2 + i\e}
\end{equation}
After these preliminaries, we can now start calculating diagrams.
We calculate the effects of the fermionic loop first. 
As mentioned earlier, we are doing something very similar to the QED case:
\begin{eqnarray}
i\cM_F^{ab\mu\nu} &=& \begin{feyn}
\ \arrow[gluon]{r}{p}&
\cdot \arrow[fermion,bend right=50pt]{r}{p}&
\cdot \arrow[fermion,bend right=50pt]{l}{k-p}
\arrow[gluon]{r}{p}& \
\end{feyn}
\\
&=&  -\tr\left[ 
T^a T^b\right]\left( ig \right)^2
\int \frac{\d{^4 k}}{\left( 2\pi \right)^4}
\frac{i}{\left( p-k \right)^2 - m^2}
\frac{i}{k^2 - m^2}
\\ && \Tr\left[ \gamma^\mu \left( \fsl{k} - \fsl{p} + m \right) \gamma^\nu 
\left( \fsl{k} + m \right) \right]
\end{eqnarray}
In particular, we notice that this is exactly the QED case, only now we have the factor 
$\tr\left[ T^a T^b \right] = T_F \dd^{ab}$ to account for the colors.
We will see
\begin{equation}
\cM = -g^2 \dd^{ab} \left( \eta^{\mu\nu} p^2 - p^\mu p^\nu \right)\Pi_2\left( p^2 \right)
\end{equation}
which is required by gauge symmetry. So this is required by the fermion loop, and for the sum of the 
four loop gluon, the gluon bubble, and the ghost bubble, but not for these three
diagrams on their own.
We have the following expression:
\begin{equation}
M_F = \dd_{ab}T_F \left( 
\frac{g^2}{16\pi^2}\right)\left( \eta^{\mu\nu} p^2 - p^\mu p^\nu \right)
\left[ 
-\frac{8}{3}\frac{1}{\e} - \frac{20}{9}
-\frac{4}{3} 
\log\left(-\frac{\tilde \mu^2}{p^2}\right)
\right]
\end{equation}
so we have isolated our first one loop divergence.
Now the sum for all four diagrams is given by:
\begin{equation}
\cM^{ab\mu\nu} = 
\dd^{ab} \frac{g^2}{16\pi^2} \left( \ldots \right)^\mu \times
\left[ 
C_A\left( 
\frac{10}{3\e} + \frac{5}{3} \log \frac{\tilde \mu ^2}{-p^2} 
\right)
-n_f T_F\left( 
\frac{8}{3} \frac{1}{\e} + \frac{4}{3} 
\log \left( 
\frac{\tilde \mu^2}{-\p^2} 
\right)
\right)\right]
\end{equation}
The next step is to write the renormalized perturbation theory. 
\begin{eqnarray}
\cL &=&  -\frac{1}{4} \left( \p_\mu A_\nu^B - \p_\nu A_\mu^B \right)^2 + \cdots \\
&=&  -\frac{1}{4} Z_3\left( \p_\mu A^a_\nu - \p_\nu A_\mu^a \right)
+ Z_2 \bar \psi_i \left( i\fsl{\p} - Z_m m \right)\psi_i \\
&&\qquad - Z_{3c} \bar c^a \dalem c^a - g_R Z_{A^3} f^{abc} 
\left( \p_\mu A_\nu^a \right)A_\mu^b A_\nu^c \\
&&\qquad - \frac{1}{4} g_R^2 Z_{A4}\left( f^{eab} A_\mu^a A_\nu^b \right) \left( f^{ecd}A^c_\mu A_\nu^d \right)\\
&&\qquad  + g_R Z_{1} A_\mu^a \bar \psi_i \gamma^\mu T_{ij}^a \psi_j
+ g_R Z_{1c} f^{abc} \left( \p_\mu \bar c^a \right) A_\mu^b c^c
\end{eqnarray}
where we have written
$\phi_0 \equiv \phi_B$ for bare quantities, and $\phi_R$ for renormalized.
Whenever a quantity has no subscript, it is assumed to be renormalized.
Note that
\begin{equation}
Z_X = 1 + \dd_x
\end{equation}
Now we have to calculate these $Z_X$ in the perturbation theory.
As we saw in the one loop calculation for the diagram:
\begin{equation}
\begin{feyn}
{}\arrow[r,gluon]&
\tp \arrow[r,gluon]&
{}
\end{feyn}
\end{equation}
we get the expression:
\begin{equation}
\dd_3 = \frac{1}{\e} \frac{g^2}{16 \pi^2}
\left[ \frac{10}{3} C_A - \frac{8}{3} n_f T_F + \left( 1 - \xi \right)C_A \right]
\end{equation}
Now again from the previous calculations, we can calculate the counter-terms:
\begin{equation}
\dd_2 = \frac{1}{\e} \frac{g^2}{16 \pi^2} \left[ -2 C_F + 2\left( 1 - \xi \right)C_F\right]
\qquad\qquad
\dd_m = \frac{1}{\e} \frac{g^2}{16\pi^2} \left( -6C_f \right)
\end{equation}
Recall we have the following results:
\begin{eqnarray}
\begin{feyn}
&& {}\arrow[dl,fermion] \\
{}\arrow[r,gluon]&
\cdot  \arrow[dr,fermion]&
\\ && {}
\end{feyn} &=&  ig \Gamma_{ij}^{a\mu}
\\
\begin{feyn}
&&& {} \arrow[ddll,fermion]\\
&& {} \arrow[bend left,dd,gluon] & \\
{}\arrow[r,gluon]&
\cdot \arrow[ddrr,fermion]
&& \\
&& {} & \\
&&& {}
\end{feyn}
&=&  ig\left( T^c T^a T^b \right)_{ij} \dd^{bc} \Gamma_{\left( 2A \right)}^\mu
\end{eqnarray}
Now since we have the following expressions:
\begin{eqnarray}
\left( T^b T^a T^b \right)_{ij} &=&  \left( C_F - C_A/2 \right)\left( T^a \right)_{ij}
\\
\Gamma^\mu_{\left( 2A \right)} &=& 
F_1^{\left( 2A \right)} \left( \frac{P^2}{m^2} \right)\gamma^\mu + 
\frac{i\sigma^{\mu\nu}}{2m} p_\nu F_2^{\left( 2A \right)} \left( \frac{p^2}{m^2} \right)
\end{eqnarray}
so we have the final expressions for the counter terms:
\begin{eqnarray}
\dd_1     &=&  \frac{1}{\e} \left( \frac{g^2}{16\pi^2} \right)
\left[ -2 C_F - 2C_a + 2\left( 1 - \xi \right)C_F + \frac{1}{2} \left( 1 - \xi C_A \right) \right]
\\
\dd_2     &=&  \frac{1}{\e} \left( \frac{g^2}{16\pi^2} \right) 
\left[ -2C_F + 2\left( 1 - \xi \right)C_F \right]
\\
\dd_m     &=&  \frac{1}{\e} \left( \frac{g^2}{16\pi^2} \right) 
\left[ -6C_F \right]
\\
\dd_{3}   &=&  \frac{1}{\e} \left( \frac{g^2}{16\pi^2} \right) 
\left[ \frac{10}{3} C_A - \frac{8}{3} n_f T_F + \left( 1 - \xi \right)C_A \right]
\\
\dd_{3c}  &=&  \frac{1}{\e} \left( \frac{g^2}{16\pi^2} \right) 
\left[ C_A + \frac{1}{2}\left( 1 - \xi \right)C_A \right]
\\
\dd_{A^3} &=&  \frac{1}{\e} \left( \frac{g^2}{16\pi^2} \right) 
\left[ \frac{4}{3} C_A - \frac{8}{3} n_f T_F + \frac{3}{2} \left( 1 - \xi \right)C_A \right]
\\
\dd_{A^4} &=&  \frac{1}{\e} \left( \frac{g^2}{16\pi^2} \right) 
\left[ -\frac{2}{3} C_A - \frac{8}{3}n_f T_F + 2\left( 1 - \xi \right)C_A \right]
\\
\dd_{1c}  &=&  \frac{1}{\e} \left( \frac{g^2}{16\pi^2} \right) 
\left[ -C_A + \left( 1 - \xi \right)C_A \right]
\end{eqnarray}

\subsection{Physical observables}



\section{BRST symmetry}

Recall that we have the following form of the path integral:
\begin{eqnarray}
\cZ &=&  \int \cD A_\mu^a \cD \cB^a \cD
\prod \dd\left( \p^\mu \cA_\mu^a - \cB^a \right)\\
&&\exp\left( i\int \d{^4 x} \tr\left( F_{\mu\nu}F^{\mu\nu} + 
\xi \tr C^2 + ib M\left( \cA \right) c \right) \right)
\end{eqnarray}
But recall this is no longer gauge invariant. 
It is however invariant under the BRST\footnote{
This stands for Becchi, Rouet, and Stora and Tyutin.
}
local fermionic symmetry.
Note that the auxiliary field $\cB$ can be integrated
out since it is only present in a quadratic exponential, and a delta function. 
So this is often simplified to:
\begin{equation}
\cZ = \int \cD A_\mu^a
\exp\left( i\int \d{^4 x} \tr\left( F_{\mu\nu}F^{\mu\nu} \right)
+ 1 / \xi\left( \p^\mu \cA_\mu \right)^2 + i b M\left( \cA \right)c
\right)
\end{equation}

As before, this symmetry is given by some charge $Q$\footnote{
This was originally called $s$ and is sometimes called $\Delta$.}
except now $Q$ is a fermionic scalar.
In particular $Q$ satisfies the anti-commutator relation:
$\left\{ Q , Q \right\} = 0$. 
In other words, $Q_\brst$ satisfies the super charge condition $Q^2 = 0$.
So what were people doing to find this?

\begin{enumerate}
\item Consider QED. 
\begin{enumerate}
\item
First observe that if $\dalem \al = 0$, 
then there is a residual symmetry:
\begin{equation}
A_\mu \fromto A_\mu + \p_\mu \al
\end{equation}

\item Observe that the equations of motion for $c$ are
$\dalem x = 0$, and $\dalem \bar c = 0$. 

\item If I define $A_\mu \fromto A_\mu + \theta \p_\mu c$
where $\theta$ is an infinitesimal parameter of this transformation
it is also a real Grassmannian which cannot depend on spacetime. 
This means $\theta^2 = 0$.
Then we define this to be the action of $Q$:
\begin{equation}
\left( \theta Q \right) A_\mu
\end{equation}
Note now that we have
\begin{equation}
\dd\left( \p^\mu A_\mu \right)^2 = 
2\left( \p^\mu A_\mu \right)\left( \theta \dalem c \right)
\end{equation}
and now we can define the variations of the ghost/antighost to compensate for this:
\begin{equation}
\dd \bar c = - \frac{\theta}{\xi} \p^\mu A_\mu
\end{equation}
so $Q$ is indeed a symmetry. Note we didn't have to introduce any complicated variation of the ghost. 
Implicitly we set this to be $0$. 
In summary, the Abelian action of $Q$ is given by:
\begin{eqnarray}
\dd A_\mu =  \theta \p_\mu c 
\qquad\qquad
\dd\bar c = -\frac{\theta}{\xi} \p^\mu A_\mu 
\qquad\qquad
\dd c = 0
\end{eqnarray}
\end{enumerate}
\item
For non-Abelian Yang-Mills, it turns out this same formalism basically works. 
We can guess
\begin{eqnarray}
\dd A_\mu^a =  \frac{\theta}{g_\ym}
D_\mu c^a 
\qquad\qquad
\dd \bar c = -\frac{1}{g}\frac{\theta}{\xi}
\p^\mu A_\mu^a
\end{eqnarray}
but it tuns out this is not a good symmetry, because this object $M\left( \cA \right)$
is effectively $\p^\mu D_\mu$ which clearly depends on $\cA$. 
But now this symmetry will spit out some term like $\theta \bar c c c$ so we need to get rid 
of this cubic term. To fix this, postulate
\begin{equation}
\dd c^a = - \frac{1}{2} \theta f^{abc} c^b c^c
\end{equation}
Note that if we didn't integrate out the auxiliary field $\cB$, 
then $\dd \bar c^a = \theta \cB^a$ and $\dd\cB^a = 0$. 
This is the most elementary transformations where $Q^2 = 0$. 
\end{enumerate}

\chapter{Generic gauge theories}

Yang Mills is an example of a gauge theory, 
but this is really a broad family of theories.
Even gravity can be viewed as one. 
Recall we found this so called BRST symmetry almost by accident in Yang Mills.
We have this BRST charge $Q$ where $Q^2 = 0$. 

This is reminiscent of the differential geometric setup
where we have some differential which squares to $0$. 
So we have some manifold $\left( M , \om \right)$. 
Then we write the exterior derivative as $d$.
Now we can ask what the Hilbert space of physical states is?

The best definition will be the following:
\begin{enumerate}
\item Construct $\cH$ for $A_\mu$, and the ghost and anti-ghost $c , b$.
This is a huge Fock space of all the fields.
Here we even have the possibility that the ghosts can carry momentum.

\item Define an equivalence of states such that they are considered the same iff
they differ by the action of $Q$ on something:
\begin{equation}
\ket{\phi} \simeqq \ket{\phi} + Q \ket{\chi}
\end{equation}
Then we claim  that the cohomology of $Q$ is the space of physical states.
\end{enumerate}

\section{Instantons}

Much of this section is from \cite{coleman_symmetry}.
See this for a more detailed treatment. 

\begin{defn}
An instanton is a classical solution to the equations of motion (up to the total derivative
in $S$) with finite action.
\end{defn}

Basically everything in this section works with the semi-classical approximation. 
So using Hamiltonian methods in quantum mechanics, we use WKB.

Say we have some $\phi^4$ theory. 
So we have some Lagrangian:
\begin{equation}
\cL = \frac{1}{2} \p_\mu \phi \p^\mu \phi - \frac{1}{2} m^2 \phi^2 - 
\frac{g^2}{4!} \phi^4
\end{equation}
Classically, we have that $\phi' = g\phi$, 
so we have the classical Lagrangian:
\begin{equation}
\cL = \frac{1}{g^2}\left( 
\frac{1}{2} \p_\mu \phi \p^\mu \phi - \frac{1}{2} m^2 \phi^2- \frac{1}{4!} \phi^4\right)
\end{equation}

Now recall that we have another parameter to worry about. Namely, $\hbar$. 
Then there is a sense in which $\hbar$ and $g$ conspire to get
$\cL / \hbar = 1/ \left( g^2 \hbar \right)\left( \cdots \right)$, 
so when we take the $\hbar \fromto 0$ limit to get a semi-classical approximation, 
this is like taking a small $g$ limit as well. 
We really just have one parameter: $g^2 k$. 

We might wonder at this point whether we get a complete
picture of things with perturbation theory alone. 
This is in fact not the case. 
We will see this even in the basic theory of quantum mechanics
for a single particle.

\begin{exm}
We have a simple example:
\begin{equation}
L = \frac{1}{2} x'^2 - V\left( x ; g \right)
\end{equation}
where
$V\left( x ;g \right) = 1/g^2 F\left( gx \right)$. 
and $F$ is some Taylor expansion which
begins with terms of order $x^2$. 
It is a basic exercise in quantum mechanics to
calculate the transmission amplitude through a potential barrier:
\begin{equation}
\abs{T\left( E \right)} = 
\exp\left\{ -\frac{1}{\hbar} \int_{x_1}^{x_2} \d{x}
\left[ 2\left( V-E \right) \right]^{1/2}\right\} \left( 1 + \cO\left( \hbar \right) \right)
\end{equation}
We see, however, that transmission is not seen
in any order of perturbation theory, because
our expression vanishes more rapidly than any power of $\h$
and therefore more rapidly than any power of $g$.
So we need other methods to see these instanton solutions.
\end{exm}

\subsection{Quantum mechanics of the harmonic oscillator}

We will be focusing on the path integral formalism.
Recall that the path integral for euclidean time is written:
\begin{equation}
\braket{x_f}{e^{-HT /\hbar}}{x_i} = 
\cN \int \left[ \d{x} \right] e^{-S_E / \hbar}
\end{equation}
Now recall that the harmonic oscillator has
$H\ket {n} = E_n\ket{n}$ 
where
\begin{equation}
H = \frac{p^2}{2} + V\left( x \right) 
\end{equation}
and we can write:
\begin{equation}
\braket{c_f}{e^{-HT / \hbar}}{x_i} = \sum_n e^{-E_n T / \hbar} \brkt{x_f}{n} \brkt{n}{x_i}
\end{equation}
where
\begin{equation}
S_E = \int_{-T/2}^{T/2} \d{t} \left[ \frac{1}{2} \left( \frac{dx}{dt} \right)^2 + V\left( x \right) \right]
\end{equation}
Now if we take
\begin{equation}
x\left( t \right) = \bar x\left( t \right) + \sum_n x_n x_n\left( t \right)
\qquad
\left[ \d{x} \right] = \prod_n \left( 2\pi \hbar \right)^{-1/2} \d{c_n}
\end{equation}
Now the euclidean equation of motion is given by:
\begin{equation}
\frac{\dd S_E}{\dd\bar x} = 
-\frac{-\d{^2 \bar x}}{\d{t^2}} + V;\left( x \right) = 0
\end{equation}
then for $x_n$, we have
\begin{equation}
-\frac{d^2 x_n}{dt^2} + V''\left( \bar x \right) x_n = \lam_n x_n
\end{equation}
Now we can calculate
\begin{equation}
\braket{x_f}{e^{-HT / \hbar}}{x_i} = 
\cN e^{-S\left( \bar x \right)/\hbar}
\prod_n \lam_n^{-1/2}
\end{equation}
So this is exact for the actual harmonic oscillator, but if we are using a generic potential
there will be  a factor of $\left( 1 + \cO\left( \hbar \right) \right)$ tagged on.

Now the limit we are really interested in is $T\fromto \infty$/ 
Now we know $\om^2\left( \bar x \right) = V''\left( \bar x \right)$
with boundary conditions $x_f = 0$ at $T/2$ and $x_i = 0$ at $-T/2$. 
This means $\om^2 = V''\left( 0 \right)$, 
so $\cN \det \left( \cdots \right) = \left( \frac{\om}{\pi \hbar} \right)e^{-\om T/2}$
so $E = \hbar \om /2$.

\subsection{Quantum mechanics of a double well}

In regular (real) time, we have two minima.
We have two maxima in euclidean (imaginary) time.
Now we can always just use Hamiltonian methods to treat this, 
but we will use the path integral formalism, 
so that we can more easily generalize to treat gauge theories.
Assign the maxima in euclidean time to be at $\pm a$. 
Now we are interested in the two expectation values:
\begin{align}
\braket{\pm a}{e^{-HT / \hbar}}{\pm a}
&&
\braket{\pm a}{e^{-HT / \hbar}}{\mp a}
\end{align}
Then our goal is to find solutions to the classical euclidean equations of motion. 
Now looking at the equations of motion, we can see the following is true. 
For $E = 0$,
\begin{equation}
\frac{\d{x}}{\d{t}} = \left( 2V \right)^{1/2}
\end{equation}
Now we can solve this rather easily to get
$t = t_1 + \int_0^x \d{x'} \left( 2C\left( x' \right) \right)^{-1/2}$. 
So this is a pseudo particle. It seems to be a solution centered at some point in time.
This is where the name instanton came from.
We call the corresponding solution for $a\fromto -a$ an anti-instanton. 
The action here can be written:
\begin{equation}
S_E = \int \d{T} \left[ \frac{1}{2} \left( \frac{\d{x}}{\d{t}} \right)^2 + V \right]
=\int \d{T} \left( \frac{\d{x}}{\d{t}}\right)^2
= \int_{-a}^a \d{x}\left( 2V\left( x \right) \right)^{1/2}
\end{equation}
so in the large $T$ approximation, $x\fromto a$, and
$\d{x}/ \d{t} = \om\left( a- x \right)$. 

Now from the differential equation, we can solve for $a-x$. 
In particular, 
\begin{equation}
a- x \propto e^{-\om t}
\end{equation}
This is some decaying function, so $1/\om$ is the size of the instanton.

We now make a few remarks:
\begin{enumerate}
\item For $n$ widely separated instantons ($\Delta d \gg 1/\om$)
then $S = nS_E$ where $n$ is the number of instantons and anti-instantons.
\item We already know the usual form of the determinant, and here we just have a correction $K$:
\begin{equation}
\left( \frac{\om}{\pi \h} \right)^{1/2} e^{-\om T / 2} K^n
\end{equation}
\item We just integrate over all centers.
\item Because of the double well structure, every instanton has to be following by
an anti-instanton.
This  means if we are doing  calculations, we will have corrections:
\begin{equation}
\braket{-a}{e^{-HT / \hbar}}{-a} = \left( \frac{\om}{\pi \hbar} \right)^{12}
e^{-\om T / 2}
\sum_{n \text{odd}}
\frac{\left( K e^{-S_E/\hbar} T \right)^n}{n!} \left[ 1 + \cO\left( \hbar \right) \right]
\end{equation}
\end{enumerate}

We have the expression:
\begin{equation}
\braket{\pm }{e^{-HT / \hbar}}{-a} = 
\frac{1}{2}\left( \frac{\om}{\pi \hbar} \right)^{1/2}
e^{-\om T / 2}\left( 
e^{K\exp\left(-S_E / k\right)T} \mp e^{-Kexp\left( -S_E / \hbar \right)}\right)
\end{equation}
Now the important point is, that the energy itself is modified by
barrier penetration.
\begin{equation}
\abs{\brkt{+}{\pm}}^3 = \frac{1}{2}\left( \frac{\om}{\pi \hbar} \right)^{1/2}
\fromto E_{\pm} = \frac{1}{2} \hbar \om \pm \hbar Ke^{-S_E/\hbar}
\end{equation}
So the final term is non-perturbative.

\subsection{Quantum mechanics of a sinusoidal potential}

Now consider a sinusoidal potential $e^{ix_j \theta}$ where $x_j$ is the $j$th
minimum of the real time potential. 
We obtain
\begin{multline}
\braket{j_+}{e^{-HT / \hbar}}{j_-} = 
\sum_{n = 0}^\infty \sum_{\bar n = 0}^\infty
\frac{!}{n! \bar n!}
\left( K e^{-S_E / \hbar} T \right)^{n + \bar n} \dd_{n - \bar n - j_+ + j_-}
\\
\left( \frac{\om}{\pi\hbar} \right)^{-1/2} e^{-\om T/2}
\end{multline}
where $n$ is the number of instantons and $\bar n$ is the number of 
anti-instantons. If we use the identity
\begin{equation}
\dd_{ab} = \int_0^{2\pi} \d{\theta} e^{i\theta \left( a-b \right)} / 2\pi
\end{equation}
the sum becomes two independent exponential series, and we can write:
\begin{multline}
\braket{j_+}{e^{-HT / \h}}{j_-} =
\sqrt{
\frac{\om}{\pi \h}} e^{-\om T / 2}
\int_0^{2\pi} e^{i\left( j_- - j_+ \right)\theta}
\frac{\d{\theta}}{2\pi}  \\
\exp\left( 2KT \cos\theta e^{-S_0 / \h}\right)
\end{multline}
As such, the continuum of energy eigenstates is labeled by $\theta$.
The energy eigenvalues themselves are given by:
\begin{equation}
E\left( \theta \right) = \frac{\h \om}{2} + 2\h K \cos \theta e^{-S_0 / \h}
\end{equation}
and
\begin{equation}
\brkt{\theta}{j} = \sqrt{\frac{\om}{\pi \h}}
\left( 2\pi \right)^{-1/2} e^{ij\theta}
\end{equation}

\subsection{Instantons in $\SU\left( 2 \right)$ gauge theory in 4D}

Now the real object of study here is the finite contribution to the action. 
For us to have finite action, there are some constraints that show up.
We can refactor $g \equiv g_\ym$ out front:
\begin{equation}
\cL = \frac{1}{g^2} \Tr\left( F_{\mu\nu}^a F_{\mu\nu}^b \right)
\end{equation}
now $F$ needs to fall off like $F\sim \cO\left( r^{-3} \right)$
and $A$ needs to fall off like
$A\sim g\p_\mu g^{-1} + \cO\left( r^{-2} \right)$
from the derivative, and since $A$ is only defined up to a gauge transformation.
So the problem with the first term of this asymptotic expansion for $A$, 
is that we cannot gauge away globally.
This is because, if we try to perform a gauge transformation to gauge these away, 
these fall into different classes. 
But not all of these can be continuously deformed such that they all go to the identity.

In any case, we have a boundary space, which maps into the Gauge group. 
so we are looking for non-trivial continuous mappings. 

\begin{exm}
If our gauge group is $\U\left( 1 \right)$, and we are in two dimensions, 
the one point compactification of the boundary is $S^1$.
So are there any mappings where $S^1 \fromto S^1 \simeqq \U\left( 1 \right)$?
Yes, we can have $g^{\left( 0 \right)} = 1$. 
But globally we can't make this equal $1$, $g^{\left( I \right)} = e^{i\theta}$.
Similarly, 
\begin{equation}
g^{\left( n \right)} = \left( g^{\left( I \right)} \right)^n = e^{in\theta}
\end{equation}
for $n\in \ZZ$. 
\end{exm}

\begin{exm}
In four dimensions, our boundary looks like $S^3$, 
and we can map $S^3 \fromto S1$.
So we don't have topological obstructions for a $\U\left( 1 \right)$ gauge field in $4$ dimensions.
\end{exm}

So we don't have to worry for Abelian theories in $4$ dimensions. 
But we do have  to worry for non-abelian theories.

\begin{exm}
Now consider $\SU\left( 2 \right)$ in $4$ dimensions. 
We have
\begin{equation}
g = \nu \Pi + i \vec{b}\vec{\sigma}
\end{equation}
where $\sigma$ denotes the  pauli matrices, and $\p$ depends on spacetime. 
Indeed we have the same as $\U\left( 1 \right)$ in $2$ dimensions:
\begin{equation}
g^{\left( 0 \right)} = 1
\qquad
g^{\left( I \right)} \left( x \right) = 
\frac{x_0 + i \vec{x} \vec{\sigma}}{r}
\qquad
g^{\left( \nu \right)} \left( x \right) = \left( g^{\left( I \right)} \right)^{\nu}
\end{equation}
\end{exm}

For $\U\left( 1 \right)$ in two dimensions, 
\begin{equation}
\nu = \frac{1}{2\pi} \int_0^{2\pi} \d{\theta} d\frac{\d{g^{-1}}}{\d{\theta}} = 
\frac{1}{2\pi} \int \d{^2 x} \cE_{\mu\nu}F_{\mu\nu}
\end{equation}

For $\SU\left( 2 \right)$, where $\theta_i$ parameterizes $S^3$, 
\begin{align}
\nu &= -\frac{1}{24\pi^2} \int \d{\theta_1}
\d{\theta_2} \d{\theta_3} \Tr\left( \e^{ijh} d \p_i g^{-1}\times
g \p_j g^{-1} g \p_\mu g^{-1}\right) \\
&= \frac{1}{32 \pi^2} \int \p{^4 x} \tilde F F
\\
\tilde F &= \frac{1}{2} \e^{\mu\nu \rho \sigma} F_{\rho \sigma}
\end{align}
Now recall that if we are in the axial gauge, we have no ghosts. 
So in this setting, we can explore instantons without concerning ourselves over ghosts. 
Of course we can always go to the  Faddeev-Popov setting and do some calculations in a simpler fashion,
the axial Gauge will suit our purpose well.
So here $A_3 = 0$. 
Then we have
\begin{equation}
F\left( V , T , n \right) = \cN \int \left[ \d{A} \right] e^{-S} \dd_{\nu n}
\end{equation}
Now we find, when we impose different instanton  solutions:
\begin{equation}
F\left( V , T_1 + T_2 , n \right) = 
\sum_{n_1 + n_2 = n} F\left( V , T_1 , n_1 \right)\otimes F\left(V , T_2 , n_2 \right)
\end{equation}
Now we can basically perform a Fourier transform to get:
\begin{equation}
F\left( V , T , \theta \right) = \sum_n e^{in\theta} F\left( V, T , n \right) = 
\cN \int \left[ \d{A} \right] e^{-S} e^{i\nu\theta}
\end{equation}
which means we can factor:
\begin{equation}
F\left( V , T_1 + T_@ , \theta \right) = 
F\left( V , T_1 , \theta \right)F\left( V , T_2 ,\theta \right)
\end{equation}
and we know
\begin{equation}
\nu = \frac{\theta}{32 \pi^2} \int F \tilde F
\end{equation}

\section{BRST quantization of gauge theories}

A good additional reference for this section is chapter 4.3 of
\cite{polchinski_st_1}.
We will first treat a general gauge theory.
We will have fields $\phi_i$ where $i$ is a single index 
standing for $\mu$, $a$, and $x$. 
In particular, for a YM theory this will be $A_\mu^a\left( x \right)$. 
We will call the gauge transformations $\dd_\al$. 
Here $\al$ collectively refers to $a$ and $x$.
For YM, this will of course be $\dd A_\mu^a\left( x \right) = D_\mu \b^a\left( x \right)$
We will assume that the infinitesimal
gauge transformations form a closed Lie algebra.
This means that we essentially have two  assumptions:
$\left[ \dd_\al , \dd_\b \right] = \tensor{f}{^\gamma_\al_\b} \dd_\gamma$,
and that $\tensor{f}{^\gamma_\al_\b}$ are constants. 
These two assumptions can however fail. This can even happen in
gravity or super-gravity.\footnote{
We can treat this with the BRST-BV formalism.}
Now we choose a gauge fixing condition
$F^A\left( \phi_i \right) = 0$.
For example, we have seen the Lorentz gauge fixing condition to be 
$0 = \cF = \p^\mu A_\mu^a$
so this collective index $A$ stands for both $\mu$ and $a$ in this example. 

Now we have three objects, the $\phi_i$, $\dd_\al$, 
and $F^A\left( \phi_i \right)$. 
This admits a path integral representation:
\begin{equation}
\int \frac{\cD\phi_i}{\Vol\left( \cG \right)}
e^{-S_1 \left(  \phi_i \right)}
\end{equation}
which is somewhat ill defined, since we are
taking the volume of the infinite group $\cG$
to cancel the redundancies in the description.
Recall in the case of YM, $S_1 = 1/\left( 2g^2 \right) \int \tr\left( FF \right)$. 
Now we claim that we are defining, and disentangling this expression to get:
\begin{equation}
\int \cD \phi_i \cD B_A \cD b_A \cD c^\al
e^{-S_1 - S_2 - S_3 - S_4}
\end{equation}
where $c$ will be the so-called ghost, $b$ the anti-ghost
if we are assuming the system is fermionic, then these will be fermions.
$B$ is the auxiliary field boson.
So in addition to the three objects from before, 
we have now introduced $c^\al , b_A, B_A$.
Two of the additional terms in the action are:
\begin{equation}
S_2 = -i B_A F^A\left( \phi \right)
\qquad\qquad
S_3 = b_A \dd_\al F^A\left( \phi \right) c_A
\end{equation}
We will call the BRST charge $Q_{\text{BRST}}\equiv Q$.
This generated a global symmetry.
The action is given by:
\begin{align}
\dd_B \phi_i &=  -i \e c^\al \dd_\al \phi_i
&
\dd_B c^\al &=  \frac{i}{2}\e \tensor{f}{^\al_\b_\gamma} c^\b c^\gamma
\\
\dd_B B_A &= 0
&
\dd_B b_A &=  B_A
\end{align}
where $\e$ is not in any way spacetime dependent.
This is just a single Grassmannian parameter, so $\e^2 = 0$. 
The final term in the action is then given by:
\begin{equation}
\e S_4 = \dd_B \left( b_A B_B M^{AB} \right)
\end{equation}
where $M$ is an arbitrary matrix of constants independent of any fields. 
These terms seem arbitrary at this point, but after we discuss the space
which we will identify with the physical states of the system, we will understand
why these terms make sense.

In any case, we now have four fields,
which split into two subcollections $\phi_i$ with $c^\al$ 
and $b_A$ with $B_A$. 
This is because $b_A$ and $B_A$ comprise
the smallest collection of fields on which $Q^2 = 0$
without any equations of motion.
In addition to $Q$, we also have a $\U\left( 1 \right)$
symmetry, given by the ghost number. 
In particular, we denote this by $U$ and take
\begin{equation}
U\left( \phi_i \right) = 0
\qquad
U\left( c^\al \right) = 1
\qquad
U\left( b_A \right) = -1
\qquad
U\left( B \right) = 0
\end{equation}
This is a global symmetry with an anomaly.

First define
\begin{equation}
\psi \ceqq b_AQ F^A\left( \phi \right)
\end{equation}
where $\Psi$ is often called the gauge fixing fermion. 
This gives us
\begin{equation}
i\e \left( S_2 + S_3 \right)
\end{equation}
In other words, the $S_2 + S_3$ part of the action is the BRST transformation of something.
In fact, we have this for $S_4$ as well. Recall that
\begin{equation}
\e S_4 = \dd_B\left( b_A B_B M^{AB} \right)
\end{equation}
where $M$ is an arbitrary matrix of constants independent of any fields. 
Clearly the constants are BRST invariant, and $B_B$ is invariant, so
the action of $Q$ gives us a quadratic term $B_A B_B M^{AB}$.
Now notice we have added something which is $Q$ of something, so it doesn't
change the physics. 
$S_1$ wasn't like this, it wasn't $Q$ of anything, but this is okay
because it was a starting point for us. 

Why is $Q^2 = 0$ physically meaningful?
First we have the question of what the Hilbert space of
physical states is. 
We know if we just take the Fock space of $\phi_i$, this is too large.
What about the Fock space $\cH$ of $\phi_i$,$b$,$c$?
Well $\cH$ is clearly even larger, so we need to identify conditions for 
$\ket{\psi}\in \cH$ to be physical. 
In particular, we know physics doesn't change
when we change $F^A \fromto F^A + \dd_G F^A$.
This means
\begin{equation}
0 = \e \dd_G \brkt{f}{i} 
= i \braket{f}{\dd_B\left( b_A \dd F^A \right)}{i} 
= -\e \braket{F}{ \left\{ Q , b_A \dd_G F^A \right\}}{i}
\end{equation}
so $Q\ket{i}= 0$ and $Q\ket{f} = 0$.
This is clearly necessary, but this is in fact not sufficient. 
To see this, consider any state $\bra{f}$ such that $\bra{F}Q = 0$, 
and choose $\ket{i}$ such that
there is some $Q\ket{\chi}$, meaning $Q\ket{i}=0$.
Now we have:
\begin{equation}
\brkt{f}{i} = \braket{f}{Q}{\chi} = 0
\end{equation}
Therefore we need to identify 
\begin{equation}
\ket{\psi} \simeqq \ket{\psi} + Q \ket{\chi}
\end{equation}
for any $\chi\in \cH$.
More technically $\cH_{\phys}$ is the $Q$-cohomology on $\cH$.
States that are annihilated by $Q$ are called $Q$-closed,
and then $\cH_\phys$ can be said to be the $Q$-closed states modulo
$Q$-exact states.

\subsection{Cohomology}

An abstract operator $d$ which squares to $0$ gives a more general cohomology theory.
Start with some differentiable manifold $M$ of dimension $d$, with some coordinates $x^i$.
Take a special subclass of all possible tensors. 
In particular take covariant, rank-$p$, fully anti-symmetric tensors
for $p\in \left\{ 0 , \cdots , d \right\}$.
These are called closed forms.
\begin{equation}
\om^{\left( p \right)} = \om_{i_1 , \ldots , i_p}\left( x \right)
\d{x^{i_1}} \wedge \cdots \wedge \d{x^{i_p}}
\end{equation}
We also have a differential operator
$d: \om^{\left( p \right)} = \left( \d{\om} \right)^{\left( p+1 \right)}$
where $d^2 = 0$.
Now if we define
\begin{align}
V_{\text{closed}} = \left\{ v \st dv = 0 \right\}
&&
V_{\text{exact}} = \left\{ v\st \exists w \suchthat  v = vw \right\}
\end{align}
We can define the cohomology of $d$ to be
\begin{equation}
H\left( V , d \right) = H\left( V \right) = V_{\text{closed}} / V_{\text{exact}}
\end{equation}
We can then refine this to be restricted to forms of degree $p$, 
and then this would refine our definition to the so called $p$th cohomology group
$H^p\left( V \right)$. 
For our manifold $M$, this is called the de Rham cohomology $H^p\left( M \right)$. 

\begin{exm}
We can now calculate the de Rham cohomology of 
the two-torus. 
In particular, $H^0\left( T^2 \right) = \RR$
and $H^2\left( T^2 \right) = \RR$.
Any constant is a generator of $H^0$, and we can check there are no others.
We have that $d\left( \d{x} \wedge \d{y} \right) = 0$
so this is a non-trivial generator of $H^2$, and we can check there
are no others.
We also have that $H^1\left( T^2 \right) = \RR^2$
where the basis is $\d{x}$ and $\d{y}$.
These are $d$ of something locally, but not globally.
\end{exm}

\subsection{BRST cohomology}

Returning to our BRST charge $Q$, we have the natural treatment of $Q$ as the differential, 
and $\cH$ as $V$. The degree $p$ of the cohomology is analogous to the
ghost number of our theory.
This is self consistent with the fact that $d$ raises the degree of the form by $1$, 
and $Q$ raises the ghost number by $1$.
Note that as in the general case, we can label our cohomology
groups by an integer. In this case this corresponds to the ghost number.

Now that we see how important this condition is, we justify why $Q$ should square to $0$.
In particular, we see that don't have to actually assume this artificially.
We do assume that $Q$ is conserved under Hamiltonian evolution,
and between gauge fixing conditions:
\begin{equation}
0 = \left[ Q , \dd H \right] = 
\left[ Q , \left\{ Q , b_A \dd_G F^A  \right\} \right]
=\left[ Q^2 , b_A \dd F^A \right]
\end{equation}
so $Q^2$ commutes with any such object. 
This doesn't mean $Q^2= 0$, but it does mean it is a constant. 
But now we realize that there is no constant bosonic object with ghost number $2$, 
so it better be $0$.

\begin{exm}
Consider the system given by a point like particle where we have
the fields
$\phi^i \corr X^\mu\left( \tau \right)$ and
$e\left( \tau \right)$.
The metric here is given by 
\begin{equation}
g_{\al\b}\left( \sigma \right) = 
e_\al^A\left( \sigma \right) e_\b^B\left( \sigma \right)\eta_{AB}
\end{equation}
Now the most generic gauge invariant action we can cook up is:
\begin{equation}
S = \int \d{\tau} \left( 
\frac{1}{2} e
X'^\mu X'^\nu \eta_{\mu\nu}
+ \frac{1}{2} em^2
\right)
\end{equation}
note $\sqrt{\abs{g}} = e$.
Then the $\dd_\al$ are some single dimensional differentials.
In particular, 
\begin{equation}
\dd X^\mu\left( \tau \right) = i\e c\left( \tau \right) X'^\mu\left( \tau \right)
\end{equation}
following the strategy of BRST. Similarly, 
\begin{equation}
\dd e = i\e \left( c' e \right)
\end{equation}
Now to get that $Q^2 = 0$, we need to insist on the sense in which the ghost transforms:
\begin{equation}
\dd c = i \e c c'
\end{equation}
Now to write down a decent kinetic term for the ghost, we need to introduce an auxiliary field $B$
and antighost $b$. In particular these transform as:
\begin{align}
\dd B\left( \tau \right) = 0
&& \dd b\left( \tau \right) = B\left( \tau \right)
\end{align}
Now recall in the generic BRST treatment we had
$S_{\text{BRST}} = S_1 + S_2 + S_3 = S_1 + \left\{ Q , \Psi \right\}$
Now if we take the sensible gauge fixing condition $e = 1$, then
\begin{align}
S_2 = \int i B\left( e - 1 \right)
&&
S_3 = -\int e b' c
\end{align}
Now integrating out the auxiliary field, we get $e = 1$
and no $B$ in the path integral. So 
\begin{equation}
S = \int \d{\tau} \left( 
\frac{1}{2} \left( X'^\mu  \right)^2 + \frac{1}{2} m^2 - b' c\right)
\end{equation}
Not this is a good Gaussian, we can just solve it.
We consider the physical Hilbert space. 
Take $Q = cH$ where 
$H = 1/2\left( p^2 + m^2 \right)$. 
We can write:
\begin{align}
\left[ p^\mu , X^\nu \right] = -i \eta^{\mu\nu}
&&
\left\{ b ,c \right\} = 1
\end{align}
which shows us that
\begin{equation}
\cH = \cH_{p,x} \tp \cH_{b,c}
\end{equation}
where $\cH_{p,x}$ is given by the kets $\ket{k^\nu}$ such that
\begin{equation}
p^\mu \ket{k^\nu} = k^\mu \ket{k^\nu}
\end{equation}
So it consists of states labeled by a certain momentum.
The subtlety comes from the interpretation of the $\cH_{b,c}$.
In particular, this is a two dimensional Hilbert space given by 
\begin{equation}
\cH_{b,c} = \Span\left\{ 
\ket{\down} , \ket{\up}\right\}
\end{equation}
where we declare:
\begin{align}
b\ket{\down} &= 0
& 
b\ket{\up} &= \ket{\down} \\
c\ket{\down} &=\ket{\up}
& 
c\ket{\up} &= 0
\end{align}
Now we have:
\begin{align}
Q\ket{k , \down} = \frac{1}{2}\left( k^2 + m^2 \right)\ket{k , \up}
&&
Q\ket{k , \up} = 0
\end{align}
Now following the preceding treatment, we declare the $Q$-closed states
to be all $\ket{k , \down}$ with $k^2 = -m^2$ and all $\ket{k , \up}$. 
Then take none of the $\ket{k , \down}$ to be $Q$-exact, and take 
$\ket{k , \up} = Q\ket{\chi}$ iff $k^2 + m^2 \neq 0$. 
Now we can simple take
\begin{equation}
\cH_{\phys} \ceqq \cH_{\closed} / \cH_{\exact} 
\end{equation}
which consists of $\ket{k ,\down}$ such that $k^2 + m^2 = 0$
and $\ket{k , \up}$ such that $k^2 + m^2 = 0$.
Then we have ghost numbers $\cU\left( \ket{\down} \right) = -1/2$ and
$\cU\left( \ket{\up} \right) = 1/2$ so we can see that
\begin{equation}
\cH_{\phys} = \cH_{\phys}^{-1/2} \oplus \cH_{\phys}^{1/2}
\end{equation}
Now we add the condition that $b\ket{\phys} = 0$. 
So we see that the rules usually basically work, but we sometime shave to add
this additional condition to get rid of some degeneracies. 
This is referred to as equivariant cohomology.
\end{exm}

\section{Topological quantum field theories}

There is some ambiguity as to what topological means in this context.
We will mostly focus on cohomological quantum field theories. In particular, 
we will concern ourselves mostly with topological YM in four dimensions.
Other examples of cohomological quantum field theories are given by the so-called
topological $\sigma$-models in $1+1$ dimensions.
The fields and transformations for this theory are given by:
\begin{align}
\phi: X^i\left( \tau , \sigma \right) &&
\dd_\al: \dd X^i \left( \tau , \sigma \right) = \xi^i\left( \tau , \sigma \right)
\end{align}
The action is given by:
\begin{equation}
S_1 = \int \om_2 = T + \left\{ W , \Psi \right\}
\end{equation}
where $T$ is topologically invariant.

Another example of a cohomological QFT is given by topological quantum mechanics. 
This is a theory in $0+1$ dimensions.
The fields are given by
\begin{align}
X^i\left( t \right)
&& \dd: \dd X^i = \xi^i\left( t \right)
\end{align}
Then performing BRST gives us that this is automatically $N = 2$ super-symmetric quantum mechanics.
For more on this theory, see \cite{baulieu}.

There are also theories considered to be topological which are not cohomological.
In particular we will discuss the s-called Chern-Simons theories with action given by:
\begin{equation}
S = \int \tr \left( A \ext \d{A} + \frac{2}{3} 
A\ext A \ext A\right)
\end{equation}

\section{Topological Yang-Mills gauge theories}

Consider $M^4$ a $4$ dimensional Riemannian manifold 
with a metric $g_{\mu\nu}\left( x^{\mu} \right)$
and take $G = \SU\left( 2 \right)$.
Then our fields are given by
\begin{align}
A_\mu^a \left( x \right) &&
\dd A_\mu^a = f_\mu^a\left( x \right) + D_\mu f^a\left( x \right)
\end{align}
So we state that any deformation of $A$ is a gauge symmetry.
This means we have no propagating degrees of freedom, 
so no ``gluons'' in this theory. 
This is what we mean by topological here.
We might wonder what are chances are to treat this sort of theory 
perturbatively and come up with Feynman rules?
In fact, things get even worse
when we realize that the best invariant action we can have is:
\begin{equation}
S_{\tym} = \int_{M^4} \tr\left( F\ext F \right)
\end{equation}
since $\tr\left( F\ext F \right)$ is a total derivative and topological invariant.
Here, topological means that it is independent of local changes in $g_{\mu\nu}\left( x \right)$. 

\begin{wrn}
We have met two different notions of a topological theory.
First that the theory has no local degrees of freedom, and therefore no gluons. 
The second, is that the theory is independent of
$g_{\mu\nu}\left( x \right)$. 
We will also see that there is a notion that a theory is topological if it can be
written without $g_{\mu\nu}$.
\end{wrn}

\begin{exm}
Take $M = \RR$, then we seek to show this action is a topological invariant. 
We have that
\begin{equation}
S = \int_{\RR^4} \tr\left( F\ext F \right) = 
\int_{S^3} \tr \left( A\ext \d{A} + \frac{2}{3} A\ext A\ext A \right)
\end{equation}
where the three form in the second integral is called the \emph{Chern-Simons form}.
This form clearly doesn't care about the metric on the $3$-sphere.
This action give us the Chern-Simons gauge theory. This is a special case for $2 + 1$ dimensions. 
\end{exm}

Now recall that an instanton is a solution of the equation of motion of YM in a sector with
instanton number $1$. Then a multi-instanton has instanton number $n$. 
We have
\begin{equation}
\cZ= \int \frac{\cD A}{\Vol\left( \cG \right)} d^{-S\left( A \right)}
\end{equation}
Now to take a $\h \ll 1$ semi-classical approximation, we want to find the saddle
points. This yields
\begin{equation}
\cZ \approx \sum_{n} e^{-\h S\left( A_0 \right)} \left( \det + \cO\left( \h \right) \right)
\end{equation}
where $n$ runs for instanton numbers. Now 
instantons minimize the action, so we get
\begin{equation}
0\leq \int \d{^4 x} \tr\left( F_{\mu\nu} \pm \frac{1}{2} \e_{\mu\nu \sigma\rho} 
F^{\sigma\rho}\right)^2 = 
\int \d{^4 x} \tr\left( F_\mu\nu^{\left( \pm \right)} \right)^2
\end{equation}
so
\begin{equation}
S\left( A \right) \sim \int \d{^4 x} + \left( F_{\mu\nu}F^{\mu\nu} \right) \geq 
\abs{\int F\ext F } \sim \abs{n}
\end{equation}
so the minima correspond to satisfying the self-dual and anti-self-dual 
conditions $F_{\mu\nu}^{\left( \pm \right)} = 0$. We call this the instanton equation.

In any case, we set
\begin{equation}
S_1 = \int \tr\left\{ F\ext F \right\}
\end{equation}
Our plan will be to consider a trivial Gauge bundle, and set $S_{\tym} = 0$.
Now use the BRST machinery to add further terms to $S$, and see
if we can get to a point where we can perform perturbative calculations.

So we have chosen our fields and our gauge transformations,
now we need to determine our gauge fixing conditions, and then we can determine the ghosts
and auxiliary field.
So we need some insight in order to understand how we can insist on a useful gauge fixing condition. 
To find this, we can take advantage of our knowledge of instantons.
So what is $F^A$? Well if we have two choices, such that they can be interpolated between
using Feynman rules, then we know these theories are the same.
But if there is some choice where we cannot interpolate arbitrarily, then these two
``disconnected'' conditions are not physically equivalent. 
Recall that we can decompose $F_{\mu\nu} = F_{\mu\nu}^+ + 
F_{\mu\nu}^-$ into the self dual, and anti-self dual pieces.
Recall that instantons are given by either  $F^- = 0$ or $F^+ = 0$. 
This is called the instanton equation in physical YM. 
Ed Witten was smart to propose the gauge fixing condition: $F_{\mu\nu}^+=0$.

So now as we have seen, 
\begin{equation}
S_{\tym} = S_1 + \left\{ Q , \Psi \right\}
\end{equation}
But now we notice that we have to add ghosts, anti-ghosts, and their ghosts and anti-ghosts, 
and things can get quite complicated.
Embed this into a class of specific types of topological theories. 

\begin{wrn}
We have seen three notions of topological theories. 
There is another class of theories, known as cohomological topological quantum field theories. 
They always start with a topologically invariant action $S_1$, which gives no equation of motion,
and are then treated with the BRST prescription.
This is detailed in the paper \cite{witten_cqft}.
\end{wrn}

We call the primary ghost $\psi_\mu^a$:
\begin{equation}
\left[ Q , A_\mu^a \right] = \psi_\mu^a
\end{equation}
which will not be covariant under the BRST charge $Q$. 
So we have
\begin{align}
\left\{ Q , \psi_\mu^a \right\} = - D_\mu \phi^a
&&
\left[ Q , \phi^a \right] = 0
\end{align}
where $\phi^a$ is bosonic and $\cU\left( \phi^a \right) = 2$.
Note that the way we have formulated things, 
$Q^2$ squares to a YM gauge transformation, and not zero. 
But we can just introduce a ghost for this symmetry as well, and get the augmentations:
\begin{align}
\left[ Q , A_\mu^a \right]= \psi_\mu^a + D_\mu c^a
&&
\left\{ Q , \psi_\mu^a \right\} = -D_\mu \phi^a + \left[ c , \psi_\mu \right]^a
&&
\left[ Q , \phi^a \right] = 0 + \left[ x , \phi \right]^a
\end{align}
This resolution gives us ghosts for ghosts. 
So for any cohomological quantum field theory, 
we have that $\left[ Q , \text{FIELD} \right] = \text{GHOST}$
and $\left\{ Q , \text{GH} \right\} = \dd_\phi^{\text{GAUGE}}\left( \text{FIELD} \right)$
and $\left[ Q , \phi \right] = 0$ then $Q^2$ is a gauge transformation.

This gets quite complicated, 
and we haven't even treated anti-ghosts yet. 
We have two choices to treat this. If we have the full treatment, clearly
$Q^2 = 0$, but we can simply by leaving out the terms concerning $c^a$, 
then the theory still has YM symmetry implicitly, and we just keep track
of the fact that $Q^2$ is some YM gauge transformation.

Now we define the observable:
\begin{equation}
\cO_{k,0}\left( x \right) = \Tr\left( \phi^k\left( x \right) \right)
\end{equation}
In particular, we have ghost number $\cU\left( \cO_{k,0} \right) = 2k$.
The $0$ tells us this is a $0$-form.

The idea is, we can take
\begin{equation}
d \cO_{k,0} = \left\{ Q , \cO_{k ,1}\left( x \right) \right\}
\end{equation}
where 
\begin{equation}
\cO_{k,1} = \k \Tr \left( 
\phi^{k=1} \psi_\mu\right)\d{x^\mu}
\end{equation}
Now we can continue doing this until we exhaust the dimension of spacetime. 
This gives us:
\begin{equation}
\d \cO_{k,4} = 0
\end{equation}
So we have $\cO_{k,i}\left( x \right)$
for $i\in \left\{ 0,1,2,3,4 \right\}$.
Now for some cycle $C$, we can integrate 
\begin{equation}
\int_C \cO_{k,1} \left( x \right)
\end{equation}
which has ghost number $2k - 1$
and eventually get
\begin{equation}
\int_M \cO_{k,4}
\end{equation}
which is a candidate deformation of $S$.
Now we can take
\begin{equation}
S = S_1 + \left\{ Q , \Psi \right\} + \sum_k t_k \int \cO_{k,4}
\end{equation}
but to do this we need to specify auxiliary fields $B^{\left( \pm \right)}_{\mu\nu}$
$\chi_{\mu\nu}^{\left( \pm \right)}$ and antighost for ghost.
Actually we will have two sets:
$\left( \lam , \eta \right)$ where $\lam$ has the quantum numbers of $\phi$, 
and $\cU\left( \lam \right) = -2$. 
The second set will be 
$\left( \chi , H \right)$
where $\chi$ is a fermionic self-dual $2$-tensor, and $H$ is the auxiliary bosonic field.
Finally, $\cU\left( \chi \right) = -1$ and
$\cU\left( H \right) = 0$.
We now postulate that the action is:
\begin{equation}
S = \frac{1}{e^2} \int \left\{ Q , V \right\} \d{^4 x}
\end{equation}
where we have dropped the static term $S_1$ to focus
on the terms which give us our dynamics in the theory and
\begin{align}
\Psi =\int V \d{^4 x}&&
V = \sqrt{g}
\Tr\left\{ -2\chi_{\al \b} \left[ H^{\al \b} - 
\frac{1}{2}\left( F^{\al\b} - F^{\star \al \b} \right)\right] - D_\al \lam \psi^\al\right\}
\end{align}
where we recall that
\begin{align}
F^A\left( \phi \right): F_{\mu\nu}^{\pm} = F_{\mu\nu} \pm F_{\mu\nu}^\star
&&
F_{\mu\nu}^\star = \frac{1}{2} \e_{\mu\nu \sigma \rho}F^{\sigma\rho}
\end{align}
Now we can write the leading terms
\begin{equation}
S = \frac{1}{e^2} \int \sqrt{g} \Tr \left[ 
\frac{1}{8} \left( F_{\al \b} F_{\al\b}^\star \right)^2 + 
D_\al \lam D^\al \phi - D_\al \eta \psi^\al + \cdots\right]
\end{equation}
Now we can recognize the terms here in the first term with the square,
as being of the form 
$F_{\al\b}F^{\al\b}$ which looks like the YM lagrangian, and then
terms like $F_{\al\b}F^{\star \al\b} = F\ext F$ which are topological terms.
The next terms show us that fermions have their own statistics, which is as it should be,
since they are a result of this BRST treatment. 
There are of course many more terms, which are not of as much interest to us. 
Now we have the fundamental observation originally made by Witten,
that this looks very much like $N = 2$ supersymmetric (SUSY) YM. 

Now we have some properties of this action:
\begin{enumerate}
\item $\ZZ$ and correlation functions for the observables 
are independent of $g_{\mu\nu}\left( x \right)$.
\begin{proof}
If we change the metric $g_{\mu\nu}\fromto g_{\mu\nu} + \dd g_{\mu\nu}$
then we get $\Psi \fromto \Psi + \dd \Psi$
so $\dd S = \left\{  Q , \dd \Psi \right\}$
but since this is BRST exact, this doesn't change the observables.
\end{proof}
\item This is one loop exact. In other words, we can take $e\fromto d + \dd e$
so we can take the semi-classical limit and get the correct answer for the entire
correlation function.
This is closely related to the notion of localization in topological QFT.
To compute this semi-classical limit we need the saddle points. 
These are in fact given by the instanton equations $F_{\mu\nu}^{\pm}=0$.
Now we can take
\begin{equation}
n = \frac{1}{64 \pi^2} \int F\ext F
\end{equation}
Now from the Riemann-Roch theorem we get that the dimension of the moduli space
of instantons $\dim \cM_n$ is given by the number of fermion zero modes of $\cU = 1$
minus the number of zero modes of $\cU = -1$.
Now on the space associated with $F_{\mu\nu}^+$ we get:
\begin{equation}
D_\al \dd A_\b - D_\b \dd A_\al - \e_{\al\b \gamma\dd} D^\gamma \dd A^\dd =0 
\end{equation}
In addition, if we replace $\dd A_\al$ with $\psi_\al$ then we get the equation of motion
from the action.
So the number of independent solutions of $\psi_\al$ is a good starting prediction for the dimensionality
of this space.
\end{enumerate}


\section{Super-symmetry}

\subsection{Features of $N = 1$ Yang-Mills}

A lot of this section will be only considering superspace. 
The standard reference for this subject is 
\cite{wess_bagger}.
Note that we will be using the notation
\begin{equation}
\sigma^m_{\al , \dot{\al}} \gamma_m = \gamma_{\al , \dot{\al}}
\end{equation}

The starting point of this theory is the super-Poincar\'e algebra.
Just like we would start our consideration of field theory with the Lorentz algebra,
we start this consideration of super-symmetry considering super-Poincar\'e algebra.
This is just the normal rotations boosts and translations
with the added supersymmetry(SUSY) generators $Q_\al$, $\barr{Q}_{\dot{\al}}$.
These are spinorial fermionic objects. We insist that:
\begin{equation}
\left\{ Q_\al  , \barr{Q}_{\dot{\al}} \right\}
2_i\sigma_{\al , \dot{\al}}^m \p_m
\end{equation}
The actual realization of these is given by:
\begin{equation}
Q_\al = \frac{\p}{\p \theta^\al} - i \sigma^m_{\al , \dot{\al}} \barr{\theta}^{\dot{\al}} \p_m
\end{equation}
The representations of this algebra are superfields classified by superspin.
Superfields are functions on a supermanifold $\RR^{4, n}$ with coordinates:
\begin{align}
z^A = \left( x^m , \theta^\al , \barr{\theta}^{\dot{\al}} \right)
\end{align}
then superfields will effectively be functions on these coordinates:
$\Phi = \Phi\left( x, \theta, \barr{\theta} \right)$.
Typically we then take a ``Taylor expansion'' in the Fermionic Grassmann coordinates:
\begin{equation}
\Phi\left( x , \theta , \barr{\theta} \right) = 
\phi\left( x \right) + \theta^\al \psi_\al\left( x \right)+ 
\barr{\theta}_{\dot{\al}} \barr{\psi}^{\dot{\al}}\left( x \right) + 
\theta^2 F\left( x \right) + \cdots + \theta^2 \barr{\theta}^2 D\left( x \right)
\end{equation}
the coefficient functions are called the component fields.
Then we define the action as follows:
\begin{align}
\dd_\e \Phi = \left( \e Q + \barr{\e}\barr{Q} \right) \Phi
\end{align}
We have the component projection:
\begin{gather}
\phi\left( x \right) = \Phi\left( x , 0 , 0 \right)
= \restr{\Phi}{\theta=0}
\qquad\qquad
\restr{Q_\al \Phi}{} = 
\frac{\p}{\p\theta^\al} \restr{\Phi}{} = \psi_\al\left( x \right) \\
\restr{\dd_\e \Phi}{} = \dd_\e \phi = 
\left(\e^\al Q_\al + \barr{\e}_{\dot{\al}} \barr{Q}^{\dot{\al}}\right) \restr{\Phi}{} = 
\e^\al \psi_\al + \barr{\e}_{\dot{\al}} \barr{\psi}^{\dot{\al}}
\end{gather}
Note however that this gives us:
\begin{equation}
\dd_\e \left( Q_\al \Phi \right) \neq Q_{\al} \dd_\e \Phi
\end{equation}
To resolve this, we introduce the super covariant derivative:
\begin{gather}
\left\{ D_\al , Q_\b \right\} = \left\{ D_\al, \barr{Q}_\al \right\} = \cdots = 0 
\qquad\qquad
\left\{ D_\al , \barr{D}_{\dot{\al}} \right\} = -2i \p_{\al , \dot{\al}}
\\
D_\al = \frac{\p}{\p \theta^\al} + i \barr{\theta}^{\dot{\al}}\p_{\al \dot{\al}}
\end{gather}
Now we use this to consistently define components of a superfield to be:
\begin{align}
\phi \equiv \restr{\Phi}{} &&
\restr{\psi_\al}{} \equiv D_\al \restr{\Phi}{} &&
F \equiv D^2 \restr{\Phi}{} &&
\cdots
\end{align}
We will also use this to covariantly constrain superfields. 
In particular, we can consider a real superfield $V = \barr{V}$:
\begin{multline}
V\left( x, \theta , \barr{\theta} \right) = c\left( x \right)
+ i\theta \chi\left( x \right) - i \barr{\theta} \barr{\chi}\left( x \right)
+ \frac{1}{2} \theta^2\left[ M\left( x \right) + i N\left( x \right) \right]
\\- \frac{1}{2} \barr{\theta}^2\left[ M\left( x \right) - iN\left( x \right) \right]
- \theta\sigma^m \barr{\theta}A_m\left( x \right) 
\\+ i \theta^2 \barr{\theta}\left[ \barr{\lam} \left( x \right) 
+ \frac{1}{2} \barr{\sigma}^m \p_m \chi\left( x \right) \right]
\\- i\barr{\theta}^2 \theta \left[ \lam\left( x \right) 
+ \frac{i}{2}\sigma^m \p_m \barr{\chi}\left( x \right) \right]
\\+\frac{1}{2} \theta^2 \barr{\theta}^2 \left[ D\left( x \right) 
+ \frac{1}{2} \dalem c\left( x \right) \right]
\end{multline}
now note if we take a chiral superfield and add it to an anti-chiral superfield,
we get:
\begin{multline}
\Phi + \barr{\Phi} = \left( \phi + \barr{\phi} \right) + 
\sqrt{2}\left( \theta \psi + \barr{\theta}\barr{\psi} + \theta^2 F + \barr{\theta}^2 \barr{F}\right)
+ i\theta \sigma^m \barr{\theta} \p_m \left( \phi - \barr{\phi} \right) 
\\ + \frac{i}{\sqrt{2}} \theta^2 \barr{\theta}\barr{\sigma}^m \p_m \psi + 
\frac{i}{2} \barr{\theta}^2 \theta \sigma^m \p_m \barr{\psi} + 
\frac{1}{4} \theta^2 \barr{\theta}^2 \dalem\left( \phi + \barr{\phi} \right)
\end{multline}

\begin{prop}
\begin{equation}
\dd V = \Phi + \barr{\Phi}
\end{equation}
as the superfield gauge transformation for $V$.
\end{prop}

Note that this means the component transformations are algebraic for $C$,
$\chi$, $M$, $N$. 
The gauge where these all vanish is called the Wess-Zumino (WZ) gauge.
This gauge actually breaks supersymmetry.
The $N = 1$ vector multiplet is given by
$\left( A_m , \lam , D \right)$.
Note that in WZ gauge, $V^3 = 0$.
Now what we would like, is to find a superfield invariant under this gauge transformation. 
To do this, we make use of a chiral ``projector.''
We can define the field strength superfield:
\begin{equation}
W_\al = -\frac{1}{4} \barr{D}^2 D_\al V
\end{equation}
this kills both chiral and anti-chiral fields, and is itself chiral.
This invariance is given by:
\begin{align}
\dd W_\al &= -\frac{1}{4} \barr{D}^2 D_\al
\left( \Phi  +\barr{\Phi} \right) = 
-\frac{1}{4} \barr{D}^2 D_\al \Phi  \\
&= \frac{1}{4} \barr{D}^{\dot{\al}} \left\{ \barr{D}_{\dot{\al}} , D_\al \right\}\Phi = 
-\frac{i}{2} \barr{D}^{\dot{\al}} \p_{\al , \dot{\al}} \Phi = 0
\end{align}
This satisfies what are sometimes called ``Bianchi identities'' given by:
\begin{align}
\barr D_{\dot{\al}}W_\al = 0
&&
D^\al W_\al = \barr{D}_{\dot{\al}} \barr{W}^{\dot{\al}}
\end{align}
the first one just says chirality, the second says $D$ is real.
Now looking at the components, we do in fact find the field strength for $A$:
\begin{equation}
W_{\al} = -i\lam_{\al}
+ \left[ \dd^\b_\al D - \frac{i}{2} \left( \sigma^m \barr{\sigma}^n \right)_\al^\b
\left( \p_mA_n - \p_n A_m \right)\right]\theta_\b + 
\theta^2 \p_{\al \dot{\al}} \barr{\lam}^{\dot{\al}}
\end{equation}

Now we want to find an action.
The idea here is that every top component $D^2 \barr{D}^2 \Psi$
or $D^2 W_\al$ of a superfield is a total derivative.
The notation we will use will be:
\begin{gather}
\int \d{^2 \theta} S\left( x , \theta , \barr{\theta} \right) \equiv 
\frac{1}{16} D^2 \barr{D}^2 \restr{S}{}
\\
\int \d{^2 \theta} S'\left( x , \theta , \barr{\theta} \right)
= -\frac{1}{4} D^2 \restr{S'}{}
\end{gather}
which are supersymmetry invariant. 
For us, our action is given by:
\begin{align}
\cL& = \int \d{^2 \theta}W^\al W_\al + 
\int \d{^2\barr{\theta}} \barr{W}_{\dot{\al}}\barr{W}^{\dot{\al}}\\
&\sim W^{\al} D^2 \restr{W_\al}{} + \cdots
\sim \lam \p \barr{\lam} + \cdots + F^{mn}F_{mn} + i F\ext F
\end{align}
To get the equations of motion, vary $V$ to get $D^\al W_\al = 0$.
This leads to 
\begin{equation}
D^2 W_\al\lvert = \left( \p \barr{\lam} \right)_\al = 0
\end{equation}
So we really don't have to do anything with components if we don't want to.

\subsubsection{Matter}

We start with just a $\U\left( 1 \right)$ gauge theory. 
We attempt to introduce $l$ species of matter given by $\Phi_l$.
For this, these chiral superfields will have the transformations:
\begin{align}
\Phi'_l = e^{-i t_l \lam} \Phi_l
\end{align}
We have kinetic terms:
\begin{equation}
K\left[ \Phi , \barr{\Phi} \right] = \int \d{^4 \theta} \barr{\Phi}_l \Phi_l
\end{equation}
This is sometimes called the K\"ahler potential.
We also have the potential terms:
\begin{equation}
W\left[ \Phi \right] =
\int \d{^2 \theta} \left[ m_{ij} \Phi_i \Phi_j + g_{ijk} \Phi_i \Phi_j \Phi_k \right]
\end{equation}
this is the most general renormalizable $\U\left( 1 \right)$ invariant coupling setup.

In the local setting, we take $\Lam$ to be spacetime dependent, 
and we take the following transformations:
\begin{align}
\Phi_l' = e^{-it_l \Lam} \Phi_l
\end{align}
This does not affect the potential term, but the kinetic term is changed. 
We now have:
\begin{equation}
\barr{\Phi}'_l \Phi'_l = 
\barr{\Phi}_l\Phi_l e^{it_l \left( \barr{\Lam} - \Lam \right)}
\end{equation}
but we can simple compensate this with $\dd V = i\left( \Lam - \barr{\Lam} \right)$.
So now the K\"ahler potential is modified to:
\begin{equation}
K\left[ \Phi , \barr{\Phi} \right] = \int \d{^4\theta} \barr{\Phi}_l
e^{t_l V}\Phi_l
\end{equation}

In the nonabelian setting, for $\Lam_{ij} = T_{ij}^a \Lam_a$ we get
\begin{gather}
e^{V'} = e^{-i \barr{\Lam}} e^V e^{i\Lam} \\
W_\al = -\frac{1}{4} \barr{D}^2 e^{-V} D_\al e^V 
\qquad\qquad
\dd W_\al = e^{-i\Lam} W_\al e^{i\Lam}
\end{gather}
Now one more invariant is called the Fayet-Iliopoulos term:
\begin{equation}
\xi\int \d{^4 \theta} V
\end{equation}
which is relevant for supersymmetry breaking.


\subsection{Yang-Mills with $N = 2$ SUSY}

In this section we consider $N = 2$ SUSY
working with $N = 1$ superfields as sort of components.
Recall the general $N = 1$ action was given by:
\begin{equation}
\cL = \int\d{^2 \theta} W^\al W_\al + \int \d{^4 \theta} \barr{\Phi} e^V \Phi
+ \int \d{^2 \theta} W\left[ \Phi \right] + \xi \int \d{^4 \theta} V
\end{equation}
In order to enhance this to the $N= 2$ case we take
\begin{equation}
\cL = \frac{1}{16\pi} \Ima\left[ 
\tau \int \d{^2 \theta}W^\al W_\al + \tau \int \d{^4 \theta}\barr{\Phi} e^{-2gV} \Phi\right]
\end{equation}
where 
\begin{equation}
\tau = \theta / 2\pi +  4\pi i / g^2
\end{equation}
In components this is written:
\begin{multline}
\cL = - \frac{1}{4} F_{mn}F^{mn} - i \lam \sigma^m \nab_m \barr{\lam} + 
\frac{\theta}{32\pi^2} F_{mn}\tilde F^{mn}
+ \abs{\nab_m \phi}^2 \\
- i \barr{\Psi}\barr{\sigma}^m \nab_m \psi + \barr{F} F 
- g\barr{\phi} \left[ D , \phi \right]
-\sqrt{2} i g \barr{\phi} \left\{ \lam , \psi \right\} + 
\sqrt{2} i g \barr{\phi} \left[ \barr{\lam} , \phi \right]
\end{multline}
Connection to top YM:
For $N = 2$ we have
\begin{equation}
\SU\left( 2 \right)_L \times \SU\left( 2 \right)_R \times
\ubr{\SU\left( 2 \right)_A \times \U\left( 1 \right)_R}{R\text{-symmetry}}
\end{equation}
for the twisted theory, take $\SU\left( 1 \right)_R'$ as the diagonal subgroup of 
$\SU\left( 2 \right)_R\times \SU\left( 2 \right)_A$.

In general, for ``manifest'' $N = 1$ SUSY, 
we can introduce a second set of SUSY generators and corresponding $\tilde\theta_\al$
and $\barr{\Tilde{\theta}}_{\dot{\al}}$. 
Now we want the $N  = 2$ chiral superfield. 
Here this means something in a coordinate basis independent of $\barr{\theta}$ and 
$\barr{\tilde \theta}$.
\begin{equation}
\Psi = \Phi + \tilde \theta^\al W_\al + \tilde \theta^2 G\left( \Phi , W \right)
\end{equation}
where $G$ is just some unimportant function of $\phi$ and $W$.
The simplest action here for $\Phi$ is given by:
\begin{equation}
\Ima \int \d{^2 \theta}\d{^2 \tilde \theta} \Phi^2
\end{equation}
then the most general one is given by
\begin{equation}
\Ima \int \d{^2 \theta} \d{^2 \tilde \theta} \cF\left( \Psi \right)
\end{equation}
$\cF$ is called the $N = 2$ pre-potential\footnote{
There is a formal notion of pre-potential, and this is in fact not a pre-potential in this sense.}
The importance here is that our simple kinetic form could potentially hold this 
general form after renormalization.
Classically,
\begin{equation}
\cF_{\cl}\left( \Psi \right) = \frac{1}{2} \tr\left[ \tau \Psi \right]
\end{equation}
In $N = 1$ language,
\begin{equation}
\cL = \frac{1}{16\pi}
\Ima\left[ \d{^2 \theta}
\cF''\left( \Phi \right) W^\al W_\al + 
\int \d{^4 \theta} \barr{\Phi} e^{-2g V} \cF'\left( \Phi \right)
\right] 
\end{equation}

\subsection{Seiberg-Witten theory}

This is looking at the low energy effective action of the $N = 2$ theory
as detailed in the original paper by Seiberg and Witten \cite{seiberg_witten}.
As it turns out, this gives a lot of exact results.
Why is this of interest to us?
As it turns out, $N = 2$ YM with SUSY is a ``sweet-spot'' between
$N = 1$ which is not exactly solvable and
$N = 4$ which is exactly solvable classically, that is it is a finite theory.
This is also a very nice realization of the so called $S$-duality from string theory. 
This is also called the electric-magnetic duality, and the Mentonen-Olive duality.
This is important because it relates strongly coupled theories to weakly coupled theories.
For example, in QCD we have a large coupling constant, so perturbative calculations
don't hold much water.
This stands in comparison with QED where we have a small coupling.
There is also an example of confinement in this theory. 
A lot of things in this low energy regime
have their origins in string theory from brane constructions. 
As with most papers by Witten, there are many interesting mathematical results, 
in this case there is attention given to the study of elliptic curves.

Consider a gauge group given by $\SU\left( 2 \right)$.
We do the natural thing, and first consider the classical potential. 
The auxiliary part of this action is given by:
\begin{equation}
\cL_{\aux} = \frac{1}{g^2}\left[ 
\frac{1}{2} D^2 + \barr{F}F - g\phi\left[ D , \phi \right]
\right]
\end{equation}
now integrating out $F$ and $D$ gives:
\begin{equation}
\cL_{\aux} = -\frac{1}{2} \left( \left[ \barr{\phi} , \phi \right]^2 \right)
\leadsto V\left( \phi \right) = \frac{1}{2}\tr\left( \left[ \barr{\phi} , \phi \right] \right)^2
\end{equation}
For a vacuum $\phi_0$, $N = 2$ supersymmetry is unbroken when 
$V\left( \phi_0 \right) = 0$.

Note this is for a pure YM theory. 
Even though it looks like we have coupled additional matter, this is really just
pure $N =2$ YM. Of course if we add mass here, we will get additional contributions to the 
potential.
That is we don't consider hypermultiplets at this point.
So we are working on the Coulomb branch of the moduli space.
The moduli space is just the space of gauge inequivalent vacua.

We can decompose
\begin{equation}
\phi\left( x \right) = 
\sum_{i = 1}^3 \left( a_i\left( x \right) + ib_i\left( x \right) \right) \sigma_i
\end{equation}
now $a_1 = a_2 = 0$ by gauge transformations, and $b_1 = b_2 = 0$
by the condition $\left[ \barr{\phi} , \phi \right] = 0$.
Then we can take $a \equiv a_3 + i b_3$ and $\phi = \frac{1}{2} a\sigma_3$
which leads to an invariant $\tr \phi^2$.
Then take the coordinate on the moduli space $\mu$ to be 
$u = \lr{\tr \phi^2}$.
$\lr{\phi} = \frac{1}{2} a \sigma_3$.

\begin{rmk}
Note that since $\lr{\phi}\neq 0$
the $\SU\left( 2 \right)$ is broken to a $\cU\left( 1 \right)$ 
by a Higgs mechanism which gives mass to some of our fields.
\end{rmk}

In $N = 1$ language we have the component expression:
\begin{equation}
\cL = \frac{1}{4\pi} \Ima\left[ \cF''\left( \phi \right)\left( \p_\mu \phi \right)^2 + \cdots \right]
\end{equation}
this suggests that $\cF''$ plays the role of a $\sigma$-model
metric on our moduli space.
More precisely, this plays the role of a K\"ahler metric on a K\"ahler manifold. 
On such a manifold we have a real K\"ahler potential defined by
\begin{equation}
g_{i\bar j} = \frac{\p^2 K\left( z, \barr{z} \right)}{\p z^i \p\barr{z}^{\bar j}}
\leadsto
\d{s^2} = g_{i\bar j} \d{z^i} \d{\bar z}^{\bar j}
\end{equation}
for us we will simple have
\begin{equation}
k\left( \Phi , \barr{\Phi} \right) = \Ima\left( \barr{\Phi} \cF'\left( \Phi \right) \right)
\end{equation}
which gives us the line element:
\begin{equation}
\d{s^2} = \Ima \cF''\left( a \right) \d{a}\d{\bar a}
= \Ima \tau\left( a \right) \d{a} \d{\bar a}
\end{equation}
But now we have a problem. 
It's nice that this $\cF''$ is holomorphic, but it is actually not
positive definite. 
The solution is that the moduli space actually has singularities, so
we need to find coordinate patches to construct 
a metric which is positive definite in this geometrically local sense.
Note that in \cite{seiberg_witten}
there is an exact non-perturbative result obtained 
for $\cF\left( \Psi \right)$ given by:
\begin{align}
\cF_{1\text{-loop}}\left( \Psi \right) = \frac{1}{2\pi} \Psi^2 \log\frac{\Psi^2}{\Lam^2}
&&
\Delta \cF_{\text{non-pert}} = 
\sum_{k = 1}^\infty \cF_k\left( \frac{\Lam}{\Psi} \right)^{4k} \Psi^2
\end{align}

We now perform the Legendre transformation:
\begin{equation}
\cF_D\left( \Phi_D \right) = \cF\left( \Phi \right) - \Phi \Phi_D
\end{equation}
so we take $\Phi_D$ to be $\cF'$,
then this means $\cF'_D\left( \Phi \right) = -\Phi$.
As it turns out, this doesn't affect the K\"ahler potential at all,
but it has a very important effect on the $W^2$ part.
\begin{multline}
\int \cD V
\exp\left[ 
\frac{1}{16\pi}\int \d{^4 x}\d{^2 \theta} \Ima\left[ \cF''\left( \Phi \right)W^\al W_\al \right]\right]
\\ \simeqq
\int \cD W \cD V_D \exp\left[ \frac{i}{16\pi} \Ima\int
\d{^4 x}\left[ \d{^2 \theta} \cF''\left( \Phi \right)W^2 + 2 \d{^4 \theta}V_D D^\al W_\al \right]\right]
\end{multline}
where we have introduced a suggestively named Lagrange multiplier $V_D$.
Now we can integrate out $W$ in favor of its Lagrange multiplier.
A simple calculation shows:
\begin{align}
\int \d{^4 \theta} V_D D^\al W_\al = - \int \d{^2 \theta} W_D^\al W_\al
&&
W_{\al D} = -\frac{1}{4} \barr{D}^2 D_\al V_D
\end{align}
so integrating out $W_\al = W_D/ \cF\left( \Phi \right)$, 
\begin{equation}
\int \cD V_D \exp\left[ \frac{i}{16\pi} \Ima\int \d{^4 x}\d{^2 \theta}
\left( -\frac{1}{\cF ''} W_D^\al W_{D\al} \right)\right]
\end{equation}
so this is the $S$-dual formulation of the theory.
In terms of the coordinates $a$,$\bar a$ on the moduli space
we have that $a_D = \cF'\left( a \right)$
gives access to the remainder of the moduli space.

\section{Chern-Simons theory}

This is a $3$-dimensional
gauge theory given by a $1$-form gauge field $A$.
In components this is written $A_\mu \d{x^\mu}$.
Now for a gauge group $G$, this is given by:
\begin{equation}
A = A_\mu^a T^a \d{x^\mu}
\end{equation}
which is Lie algebra valued.
$T^a$ are the generators and satisfy
\begin{equation}
\left[ T^a , T^b \right] = i \tensor{f}{^{ab}_c} T^c
\end{equation}
Then we can raise and lower indices using:
\begin{equation}
h^{ab} = \Tr\left( T^a T^b \right)
\end{equation}
The Chern-Simons action we will consider is given by:
\begin{align}
S_{\text{cs}} &= \frac{k}{2\pi} \int \tr \left( A\ext \d{A} + \frac{2}{3} A\ext A \ext A \right)\\
&= \frac{k}{4\pi}\int \e^{\mu\nu \sigma} \left( A_{\mu a} \p_\nu A_\sigma^a + 
\frac{2}{3} f_{abc} A_\mu^a A_\nu^a A_\sigma^a \right)\d{^3 x}
\end{align}

\subsection{Quantum hall effect}

In $3$-dimensions, there is something special about YM theory. 
Recall the YM action is given by:
\begin{equation}
S_\ym = \frac{1}{4g^2} \int \Tr\left( F\ext \hodge{F} \right)
\end{equation}
\begin{rmk}
If we have
\begin{equation}
F = \frac{1}{2} F_{\mu\nu}\d{x^\mu} \exp \d{x^\nu}
\end{equation}
then
\begin{equation}
\hodge{F} = \frac{1}{4} \tensor{\e}{_\mu_\nu^\sigma^\tau}
F_{\sigma\tau} \d{x^\mu} \ext \d{x^\nu}
\end{equation}
and we have
\begin{equation}
F\ext \hodge{F} = \tr\left( F_{\mu\nu}F^{\mu\nu} \right)\frac{1}{4!} \e_{\mu\nu\sigma\tau} 
\d{x^\mu} \ext \d{x^\nu} \ext \d{x^\sigma} \ext \d{x^\tau}
\end{equation}
\end{rmk}
So we take the total action to be:
\begin{equation}
S_\text{Tot} = S_\ym + S_{\text{cs}}
\end{equation}
What does this theory look like at low energy?
From ``dimensional analysis'' we get that
\begin{equation}
\left[ g^2 \right] = 
\begin{cases}
0 & 4D \\
\left[ M \right] & 3D
\end{cases}
\end{equation}
the first case is referred to as the running of the coupling constant.
The second case follows from the fact that $A_\mu$
also has the dimension of mass.
In any case, this means the coupling $g$ becomes large in this low energy limit, 
so we can disregard the YM action,
and concern ourselves with the Chern-Simons action.

We now consider the quantum hall effect (QHE).
There are two versions of this. 
The first is the integer case, which is completely understood
and the second is the fractional case which is largely an open problem.
Let's first consider the integer case.
Consider a thin film, modelled as a $2$-dimensional strip of width $L$
under the influence of a very strong magnetic field of strength $B$.
Then apply a voltage $V_y$ over the width of the strip,
and measure the current $I_x$ along the strip. 
Recall that the resistance is given by $\rho_H = V_y / I_x$.
Now if the charge carriers have charge $Q$ and density $n$, we have
\begin{align}
\rho_H = \frac{V_y}{I_x}
= \frac{E_y L}{J_x L}
= \frac{E_y}{J_x} &&
J_x = nq v_x
\end{align}
then we have $E_y = v_x B$
and can write
\begin{equation}
\rho_H = \frac{E_y}{nq v_x} = \frac{B}{nq}
\end{equation}
Note that more generally, 
\begin{equation}
\sum_{j\in \left\{ x,y \right\}}\rho_{ij} J_j = 
E_i
\end{equation}
and if we measure $\rho_{xy}$ in terms of $B$, 
we get a linear increase which plateaus at values 
$\rho_{xy} = h/ \left( \nu e^2 \right)$ for $\nu\in \ZZ$. 
If we measure $\rho_{xx}$, we get bumps in the regions where
$\rho_{xy}$ is linear, but it drops to $0$
at the plateaus.

Why do we see this behaviour?
Let's consider a single particle with charge $q$ in $\vec{B} = B \hat z$. 
Then we have the lowest two Landau levels
with an energy gap given by $\h \om_B = qB / m$.
The degeneracy is 
\begin{equation}
\phi_0 = \frac{2\pi \h}{\abs{e}}
\end{equation}
so the number of states is given by $N= BA / \phi_0$. 
So each state has surrounded itself with this magnetic flux, 
and the question is how many we have. 
Now these values $1/\nu$ occur exactly when an integer number of Landau levels are filled.
So $\nu$ gives the number of $LL$ filled.
But this would lead us to think the plateau would only occur at a single point. 
As it turns out, the impurities in the material cause these larger plateaus to form.
Now if $\nu$ states are filled, then 
\begin{equation}
n = \frac{\nu N }{A} = 
\frac{\nu B}{\phi_0}
\end{equation}
so for $q = -c$, 
\begin{equation}
\rho_H = \frac{1}{\nu} \left( \frac{2\pi \h}{c^2} \right)
\end{equation}
which explains the $\rho_{xy}$ behavior. 
The $\rho_{xx}$ behaviour is determined by the question of determining $J_x$ solely from turning on
$E_{x} =\rho_{xx}J_x + \rho_{xy} J_y$.
Since $\rho = \sigma^{-1}$, if we can show that
$\sigma_{xx} = 0$, then $\sigma$ would have zeros on the diagonal which would mean
$\rho_{xx} = 0$.
Now we can make a Galilean boost which brings $E$ to zero.
In particular, with this large $B$ we can find such a velocity
where there will not be any current. 
But when we boost back to our reference frame, we can 
only get a $y$ current, not an $x$-current, since we boosted in the $y$ direction.
But this only applies to pure materials, since the impurities boost as well, so we do
indeed get this $\rho_{xx}$ behaviour.

The fractional quantum hall effect is similar to the integer case, only now we
get plateaus at fractional values of $\nu$.
For example, we might have a plateau
at $\nu = 2/3$ and $\nu = 3/5$.
As it turns out, the numbers that appear here are called
filling fractions. Somore additional examples are given by:
\begin{align}
\frac{1}{3} &&
\frac{2}{5} &&
\frac{3}{7} &&
\frac{2}{3} &&
\frac{3}{5} &&
\frac{4}{7} &&
\cdots
\end{align}
or more generally:
\begin{equation}
\nu = \frac{p}{q} = \frac{n}{2mn \pm 1}
\end{equation}
This problem is more difficult because we can't think about
this problem as free electrons in a field. 
Here this interpretation means Landau levels are partially filled. 
So if $N$ is the number of states in $LL$,
and $\nu = 1/3$, then there are 
\begin{equation}
\binom{N}{N/3}
\end{equation}
ways to fill the levels
which is very large. 

Now what does field theory have to say about this?
The field theory approach takes $A_{ext}$ to be the EM potential
which creates the $B-$field $\vec{B} = B\hat z$
as well as $\vec{E}$ which leads to $V_y$. 
Then our action will be a function of $A_{ext}$ and the electrons in the sample
which is given by $\hat \phi\left( \vec{r} , t \right)$. 
Now we attempt the path integral
\begin{equation}
\exp\left( -\frac{i}{\h}S_\eff \left[ A_{ext} \right] \right)=
\int \exp\left( -\frac{i}{\h} S\left[ A_{ext} , \Psi \right]\left[ \cD \Psi \right]\right)
\end{equation}
The claim is that we have:
\begin{equation}
S_\eff = \frac{k}{4\pi} \int A \ext \d{A}
\end{equation}
which is Lorentz invariant and topological since it is
independent of the metric, and has no propagating degrees of freedom.
This is called the abelian Chern-Simons action.
Now for $J_i = \sigma_{ij} E_j$, we seek:
\begin{align}
\lr{J_i\left( x \right)}_{A_{ext}} &= 
\lr{\frac{\dd S}{\dd A_{ext}}}
= \int \left( \frac{\dd S}{\dd A} \right)e^{iS /\h} \cD \Psi
\\
&= \frac{\h}{i} \frac{\dd}{\dd A}
\int e^{\frac{i S}{\hbar}} \cD \Psi 
= \frac{\dd}{\dd A}\left( e^{iS_\eff / \h }\right)
\\
&= \frac{\dd S_\eff}{\dd A} = 
\frac{k}{2\pi} \e_{ijl} F_{jl}
\end{align}
so macroscopically, we get the relation:
\begin{equation}
J_i = \frac{k}{2\pi} \e_{ij} E_j
\end{equation}
That is, for $\sigma_{xy} = k / 2\pi$ we get $\rho_{xy} = 2\pi / k$. 

We can also obtain the density of states in a similar manner:
\begin{align}
\lr{J^0} = \frac{k}{2\pi} B
\end{align}
which should be compared with
\begin{equation}
n= \frac{\nu e B}{2\pi k} = \frac{\nu B}{\phi_0}
\end{equation}
So how do we know $k$ is an integer?
Well if we are just on $\RR^2$ there are no restrictions on $k$. 
But if we put the theory on some compact space with some genus, such as a sphere, 
torus, or general Riemann surface we get some restrictions.
We can also put the theory on some surface with boundary to get some restrictions on our $k$ value.

First we put the theory on the torus $T^2$, which we think
of as being a periodic space with coordinates $x_1$ and $x_2$.
The action is of the form
\begin{equation}
S = \frac{k}{4\pi} \int A\ext \d{A}
\end{equation}
We now quantize this theory. 
First take the $A_0 = 0$ gauge fixing condition.
But recall we still have the condition:
\begin{equation}
0 = \frac{\dd A}{ \dd A_0} = F_{12}
\end{equation}
so there is in fact no field strength. 
Now this condition along with gauge invariance allows us to set the gauge field to be of the form:
\begin{equation}
A = a_1 \d{x_1} + a_2 \d{x_2}
\end{equation}
where $a_1$ and $a_2$ are just functions of time.
This means we have
\begin{align}
S &= \frac{k}{4\pi} \int A\ext \d{A}
= \frac{k}{4\pi} \int  
\left( 
a_2 \dot a_1
- a_1 \dot a_2 
\right)\d{t}
\\ &= \frac{k}{2\pi} \int a_2 \dot a_1 \d{t} \d{x}\d{y}
+ \text{boundary terms}
\end{align}
Now we have the canonical relation:
\begin{equation}
\left[ a_2 , a_1 \right] = -\frac{2\pi i }{k}
\end{equation}
But what is the Hilbert space?
Recall the residual gauge transformation
\begin{equation}
\Lam = 2\pi N \d{x_1}
\end{equation}
where $N\in \ZZ$.
This doesn't look like a gauge transformation though, since this is not single
valued.
But this is ok, because only $e^{i\Lam}\in \U\left( 1 \right)$ actually
needs to be single valued.
So let's see how this acts on $A$:
\begin{equation}
\begin{tikzcd}
A\arrow{r}{\Lam} &
a_1 \d{x_1} + a_2 \d{x_2} + 2\pi N\d{x_1}\\
a_1 \arrow{r}{2\pi N x_2}
&
a_1 + 2\pi N \\
a_2 \arrow{r}{2\pi M x_1}
&
a_2 + 2\pi M
\end{tikzcd}
\end{equation}
This means
\begin{align}
a_1 \sim a_1 + 2\pi
&&
a_2 \sim a_2 + 2\pi
\end{align}
$a_1$ and $a_2$ are not really what we want. 
So instead, we define:
\begin{align}
W_1 = e^{ia_1}
&&
W_2 = e^{ia_2}
\end{align}
Now take $C_1$ and $C_2$ to be loops along the two copies of $S^1$ making up
$T^2$. Then we have that
\begin{equation}
E_i = \exp\left( i\oint_{C_i} A \right)
\end{equation}
If we were to just integrate $A$ around these loops we would end up with
an extra shift, whereas if we integrate the exponentials these constants are in the exponential.
Now we can write:
\begin{equation}
W_1 W_2 W_1^{-1} W_2^{-1} = e^{-\left[ a_1 , a_2 \right]} = e^{-2\pi i /k} = -\om
\end{equation}
where $\om$ denotes a $k$th root of unity.
In particular, 
\begin{equation}
W_2^{-1} W_1 W_2= \om^{-1}W_1
\end{equation}
So we can write:
\begin{align}
W_1 &= e^{i\phi}
\begin{pmatrix}
1& \cdots & \cdots & \cdots \\
\cdots & \om & \cdots & \cdots \\
\cdots & \cdots & \cdots & \cdots \\
\cdots & \cdots & \cdots & \om^{k-1}
\end{pmatrix}
\\
W_2 &= e^{i\phi'}
\begin{pmatrix}
0 & 1 & 0 & \cdots & \cdots & \cdots \\
0 & 0 & 1 & 0      & \cdots & \cdots \\
\cdots & \cdots & \cdots & \cdots & \cdots & \cdots \\
1 & 0 & 0 & \cdots & \cdots & 0
\end{pmatrix}
\end{align}
where $\phi$ , $\phi'$ are effectively arbitrary. 
Now if we want a finite dimensional Hilbert space, then $k$ must be an integer. 

We can also geometrically quantize our phase space $T^2$.
Take
\begin{align}
u = a_1 + i a_2
&&
\bar u = a_1 - i a_2
&&
\left[ u , \bar u \right] = \frac{2\pi}{k}
\end{align}
so if we have $u^\dag = \bar u$ this leads to $4\pi \p_u / k$.
But what are the wavefunctions?
Well we can't take
\begin{equation}
\psi\left( a_1 , a_2 \right)\fromto
\psi\left( u , \bar u \right)
\end{equation}
so we have to pick one. 
This means 
\begin{equation}
\frac{\p}{\p \bar u} \psi = 0
\end{equation}
which means these are holomorphic. Which means 
\begin{equation}
u^\dag = \frac{4\pi}{k} \frac{d}{du}
\end{equation}
But how could this be? Let's see what this means from a different perspective. 
Take the inner product:
\begin{equation}
\brkt{\chi}{\psi} \ceqq \int \chi^* \psi \d{^2 u}
\end{equation}
which means:
\begin{align}
\brkt{\chi}{\frac{d}{\d{u}} \psi} &=
\int \chi^* \frac{d}{du} \psi \d{^2 u} = 
-\int \frac{d \chi^*}{\d{u}} \psi \d{^2 u} + 
\frac{k}{4\pi} \int \bar u \chi^* \psi \d{^2 u}
\\ &= 
\frac{k}{4\pi} \int \bar u \chi^* \psi \d{^2 u}
\end{align}
so the fix here is to instead define the inner product to be:
\begin{equation}
\brkt{\chi}{\psi} \ceqq \int_{T^2} \chi\left( u \right)^* \psi\left( u \right) 
e^{-k\abs{u}^2 / \left( 4\pi \right)} \d{^2 u}
\end{equation}
which means 
\begin{equation}
\brkt{\chi}{\frac{d}{\d{u}} \psi} = 
\frac{k}{4\pi} \int \bar u \chi^* \psi e^{-k\abs{u}^2 / \left( 4\pi \right)} \d{^2 u}
\end{equation}

So the problem is that
\begin{align}
\chi\left( u \right)^* \psi\left( u \right)
e^{-k\abs{u}^2 / \left( 4\pi \right)} &=
\chi\left( u + 2\pi \right)^* \psi\left( u + 2\pi \right)
e^{-k\abs{u + 2\pi}^2 / 4\pi} \\
&= \chi\left( u + 2\pi i \right)^* \psi\left( u + 2\pi i \right)
e^{-k\abs{u + 2\pi i}^2 / 4\pi} \\
\end{align}
So the solution is to demand:
\begin{align}
\psi\left( u + 2\pi \right) &=
\psi\left( u \right) \exp\left( k\left( u + \pi \right) / \left( 2\pi \right) \right) \\
\psi\left( u + 2\pi i \right) &=
\psi\left( u \right) \exp\left( k\left( -i u + \pi \right) / \left( 2\pi \right) \right)
\label{eqn:demand}
\end{align}
So what is the Hilbert space of holomorphic function which satisfy
\cref{eqn:demand}?
Well if $k\not\in \ZZ$, there are no solutions. 
But if $k\in \ZZ^+$, then this just means
\begin{equation}
\psi = \sum_{j = 0}^{k-1} c_j \psi_j
\end{equation}
and the explicit solutions are given by:
\begin{align}
\psi_j\left( u \right) = 
e^{k u^2/ \left( 8\pi \right)}
\sum_{n = -\infty}^\infty
e^{\left( kn + j \right)^2 / \left( 2k \right)}
\exp\left( i\left( kn + j \right)u \right)
\end{align}
for $j\in \left\{ 1 , \cdots , k-1 \right\}$
and $q = e^{-2\pi}$. 

We now consider a generalization of the previous considerations.
Consider a general compact phase space, and now quantize the theory. 
Equip this space with a symplectic $2$-form $\om$. This means
\begin{align}
\om - \sum_{i = 1}^m \d{x_i} \ext \d{p_i}
&& \d{\om} = 0
\end{align}
So $\om^{\mu\nu}$ is $2m$ dimensional.
Since this is Symplectic we have:
\begin{align}
\om_{\mu\nu} \fromto \left( \om^{-1} \right)^{\mu\nu}
&&
\left( \om^{-1} \right)^{\mu\nu} \om_{\nu\sigma} = \tensor{\dd}{^\mu_\sigma}
\end{align}

\begin{exm}
In the case of a sphere, 
\begin{equation}
\om = \sin\theta \d{\theta} \exp \d{\rho}
\end{equation}
\end{exm}

Taking local coordinates $x^i$ we have
\begin{equation}
\left[ x^l , x^j \right] = -i \om^l j
\end{equation}
We now equip this with a complex structure. Pair our local cooridnates 
into $z^\al$ for $\al\in\left\{ 1 , \cdots , m \right\}$. 
Then take
\begin{align}
\om_{\al \bar \b} = - \om_{\bar \b \al}
&&
\om_{\al \b} = \om_{\bar \al \bar \b} = 0
\end{align}

Now locally we have
\begin{equation}
\om_{\al \bar \b} = i \p_\al \p_{\bar \b} K
\end{equation}
where $K$ is the K\"ahler function which means
\begin{equation}
\left[ \p_{\bar \al} K , z^{\bar l} \right] = \dd_{\bar \al}^{\bar \l}
\end{equation}

\begin{exm}
On $T^2$, $Z^1 = u$
and $Z^{\bar 1} = \bar u$. 
Then $K = \frac{k}{4\pi} \bar{u}^2$.
\end{exm}

So now define the inner product to be:
\begin{equation}
\brkt{\chi}{\psi} = \int \chi^* \psi e^{-K} \d{^{2M}z}
\end{equation}
so now our wave functions are
\begin{equation}
\psi\left( z^1 , \cdots , z^M \right)
\end{equation}
But these K\"ahler functions are only defined locally, so for two patches
we might have
two different functions $K_I$ and $K_{II}$.
Then we have
\begin{equation}
K_{II} = K_I + f + \bar f
\end{equation}
where $f$ is holomorphic. 
In particular, 
\begin{equation}
f = \frac{k}{4\pi} u + \frac{k}{8\pi}
\end{equation}
and 
\begin{equation}
\psi_I = e^f \psi_{II}
\end{equation}
So the problem is to find holomorphic sections $\psi_I , \Psi_II , \cdots$
such that
\begin{equation}
\psi_I = e^{f_{I , II}} \psi_{II}
\end{equation}
so we are looking for a holomorphic line bundle over out phase space. 
This is just a space of sections.


FQHE:
Take $\nu\in \ZZ$. Then sending $A\fromto A + \d{\Lam}$
we get
\begin{equation}
\frac{k}{4\pi} \int_M A\ext \d{A}
\fromto 
\frac{k}{4\pi} \int_M A\ext \d{A} + 
\frac{k}{4\pi} \int_{\p M} \Lam \d{A}
\end{equation}
which is not gauge invariant.
Now consider this on a cylinder with $t$ along the middle axis, and $\theta$
the angular coordinate.
Then $\d{A} = F$
and
\begin{equation}
\int F_{t \theta} = \d{t} \d{\theta} = 2\pi m
\end{equation}
\begin{equation}
\oint A_{\theta} \d{\theta}\vert_{t_i}^{t_f} = 2\pi m
\end{equation}
so according to this formalism, there is no such FQHE. 

But in fact, there can be additional topological degrees of freedom, which means we can
have fractionally charged particles on the boundary which will give us the FQHE. 
We know we have the coupling $J\ext A$
where $d\hodge{A} = 0$ and $\hodge{J} = \d{a} / 2\pi $.
So the full action is
\begin{equation}
\int \hodge{J} \ext A + 
\frac{\tilde k}{4\pi} \int d\ext \d{a} = 
\frac{1}{2\pi} \int \d{a} \ext A + \frac{k}{4\pi} \int a\ext \d{a}
\end{equation}
where we have the equation of motion:
\begin{equation}
\left[ a \right] = 0 = \tilde k \d{a} + \d{a}
\end{equation}
which leads to:
\begin{equation}
J = -\frac{1}{2\pi\tilde k} \hodge{\d {A}}
\end{equation}
which of course gives us our fractional case. 

\subsection{Non-abelian Chern-Simons}

Consider the action
\begin{equation}
S = \frac{k}{4\pi} \int \tr\left( A\ext \d{A} + \frac{2}{3}
A\ext A\ext A\right)
\end{equation}
now one way to quantize is to take
\begin{equation}
A_0 = 0 \implies F = 0
\end{equation}
now what about some compact Riemann surface?
When we consider gauge fields on this surface, 
there is no fiedl strength, which seems to mean there is nothing going on. 
But we can still have a nontrivial gauge field which satsfies
\begin{equation}
0 = \d{A} + A\ext A
\end{equation}
and if we look at these fields modulo
\begin{equation}
A \sim \Om^{-1} A \Om + \Om^{-1} \d{\Om}
\end{equation}
this gives a finite moduli space of what are called flat connections. 
Since this is finite, we can define
\begin{equation}
\int \dd A_z \ext \dd A_{\bar z}
= \om_{\al\bar \b} \dd z^\al \al z^{\bar \b}
\end{equation}
then we can do geometric quantization, and get a finite dimensional Hilbert space. 
What is this good for?
Jones polynomials:

If we have
\begin{equation}
\O\left( x \right) = \tr F_{\mu\nu} F^{\mu\nu}
\end{equation}
then
\begin{equation}
\lr{O\left( x_1 \right)O\left( x_2 \right) \cdots} = 0
\end{equation}
But now if we take some knot $C$ we can calculate the more general 
\begin{equation}
\lr{W\left( C\right)}
\end{equation}
which gives $f\left( K \right)$ which leads to the knot invariant
\begin{equation}
q = \exp\left( \frac{2\pi i}{k + 2} \right)
\end{equation}

See \cite{witten_jones} for a more in depth approach on this subject.

\section{BRST-BV Formalism}

\subsection{Motivation}

This is a mix of the tools from BRST formalism, such as the cohomological 
charge, $Q^2 = 0$, along with new tools.
In addition to the fields (ghosts anti-ghosts, ghosts for ghosts)
we will also introduce antifields.
This is why this formalism is also called anti-field BRST.
We will justify the addition of these anti-fields by defining
what it means for a symmetry algebra to be open.
We will also see an anti-bracket, $\left( \cdot , \cdot \right)$
which is in some sense a generalization of a Poisson bracket.
At the quantum level, we will also introduce an operator $\Delta$ which
is a second order operator.

The added BV stands for Batalin-Vilovinsky.
They did the majority of their work with this formalism
between 1981-1985. This formalism will be important for the following applications:
\begin{enumerate}
\item Some theories cannot be quantized in a straight-forward way using the FP procedure. 
In particular this is the case when the symmetry object is not closed.
This just means that instead of
$\left[ \dd_\al , \dd_\b \right] = \tensor{f}{_{\al\b}^\gamma} \dd_\gamma$
we have
\begin{equation}
\left[ \dd_\al , \dd_\b \right] = \tensor{f}{_{\al\b}^\gamma}\left( \phi\right) \dd_\gamma 
+ \text{EoM}\left( \phi \right)
\end{equation}

\item Even for YM in, say, $3 + 1$ dimensions, 
when we try to show renormalizability to all loop orders, 
BRST is not enough. This was shown using the BV formalism by Zinn-Justin in \cite{ym_renorm}.

\item This is important for understanding the structure of both global and gauge anomalies. 
The important question here is determining when 
a symmetry is necessarily anomalous in quantum mechanics.
This is cohomological in nature, and the BRST formalism is sufficient for this. 
But in order to show:
\begin{prop}
The chiral anomaly (in the standard model) is one-loop exact.
\end{prop}
we need the full BRST-BV extension of the $Q$-cohomology.
\end{enumerate}

\subsection{Open symmetry algebras}

Start with the action:
\begin{align}
S_0\left[ \phi \right] = \int \d{^D x} \cL_0\left( \phi^i , \p_\mu\phi^i ,\cdots , 
\p_{\mu_1 \ldots \mu_k} \phi^i\right)
\end{align}
The equation of motion is given by:
\begin{equation}
\frac{\dd S_0}{\dd \phi^i\left( x \right)} \equiv
\frac{\dd \cL_0}{\dd \phi^i}\left( x \right) = 0
\end{equation}
where we recall
\begin{equation}
\frac{\dd \cL_0}{\dd \phi^i} \left( x \right) = 
\frac{\p \cL_0}{\p \phi^i} - \p_\mu 
\frac{\p \cL_0}{\p\left( \p_\mu \phi^i \right)} + 
\p_\mu \p_\sigma \frac{\p \cL_0}{\p\left( \p_\mu \p_\sigma \phi \right)} + \cdots
\end{equation}

Then the gauge transformations are given by:
\begin{equation}
\dd_{\e} \phi^i\left( x \right) = 
\barr{R}_\al^i \e^\al\left( x \right) + \barr{R}_\al^{i\mu} \p_\mu \e^\al + \cdots
+ E_\al^{i \mu_1 \ldots \mu_s} \p_{\mu_1} \cdots \p_{\mu_s} \e^\al
\end{equation}
and the rule
\begin{align}
\dd_\e \cL_0 = \p_\mu K_\e^\mu
&&
K^\mu_\e\left( \phi , \p \phi , \cdots , \e , \cdots ,  \p \e \right)
\end{align}

It will be convenient to use some of the compact DeWitt notation here.
Using this we can simplify:
\begin{align}
\p_\e \phi^i= R_\al^i \e^\al
&&
\dd_\e \phi^i\left( x \right) = \int \d{^D y} R^i_\al\left( x, y \right) \e^\al \left( y \right)
\end{align}
where
\begin{equation}
R^i_\al \left( x  ,y \right)  = 
\barr{R}_\al^i \dd\left( x - y \right) - \barr{R}_\al^{i\mu}\p_\mu \dd\left( x - y \right)
\end{equation}

These symmetries give us the so-called local Noether
identities.
We know we have:
\begin{equation}
\dd S_0 = 0 = \frac{\dd S_0}{\dd\phi^i}
\dd_\e p^i = 
\frac{\dd S_0}{\dd \phi^i} R^i_\al \e^\al
\end{equation}
and since this holds for any $\e$, this implies:
\begin{equation}
\frac{\dd S_0}{\dd\phi^i} R^i_\al = 0
\label{eqn:noether_id}
\end{equation}

Note that if we don't want to use the compact notation this is written:
\begin{equation}
\frac{\p \cL_0}{\p \phi^i} \barr{R}_\al^i - 
\p_\mu\left( \frac{\dd \cL_0}{\dd \phi^i} \barr{R}^{i\mu}{\al} \right) + \cdots = 0
\end{equation}

\begin{thm}
All such infinitesimal gauge transformations always form a Lie algebra.
\end{thm}

\begin{proof}
This is the claim that these commutators:
\begin{equation}
\left[ \dd_\eta , \dd_\e \right] = \tensor{f}{_{\eta \e}^\chi} \dd_\chi
\end{equation}
satisfy the Jacobi identity. 
This is trivial from \eqref{eqn:noether_id}.
\end{proof}

Note that we take $\barr{\cG}$ to be the object generated by all such infinitesimal
gauge transformations.

So what does this mean about open algebras?
Well this seems to imply all such algebras are open. 
So what about theories with algebras which are not open?
To understand this, we introduce the notion of a ``trivial'' gauge transformation.

Take a matrix of gauge transformations
$\mu^{ij} =-\mu^{ji}$
where
\begin{equation}
\dd_\mu \phi^i = \mu^{ij} \frac{\dd S_0}{\dd \phi^i}
\end{equation}
clearly these are indeed gauge transformations since
\begin{equation}
\dd_\mu S = 
\mu^{ij} \frac{\dd S_0}{  \dd\phi^i  }
\frac{\dd S_0}{   \dd \phi^j  }
\equiv 0 
\end{equation}
except these also don't imply any Ward identities.
Their action on the fields is just proportional to the equations of motion.
This means they don't constrain the action or imply anything relevant about the dynamics.
Then we have the following theorem:

\begin{thm}
The group $\cN$ generated by trivial gauge transformations, 
is a normal subgroup in $\barr{\cG}$.
\end{thm}

\begin{proof}
This can be shown explicitly for some transformation $\tau$:
\begin{align}
\dd_\tau \phi^i = \tau^i
&&
\left[ \dd_\mu , \dd_\tau \right]\phi^i = 
\left( \frac{\dd \tau^i}{\dd \phi^k} \mu^{kj} - 
\frac{\dd \tau^j}{\dd \phi^k} - t^k \frac{\dd \mu^{ij}}{\dd \phi^k}\right)
\frac{\dd S^0}{\dd \phi^i}
\end{align}
\end{proof}

So now we have two possibilities:
\begin{enumerate}
\item $\barr{\cG} = \cN \rtimes \cG$
which means we can pick a particular embedding of $\cG\embed \barr{\cG}$.
Then $\cG$ is a Lie algebra since the commutators descend from $\cG$.

\item Alternatively $\barr{\cG}$ is not a semidirect product, so
we are stuck with $\barr{\cG}$. 
This will be the case which leads to the so-called open symmetry algebra.
\end{enumerate}

\begin{prop}
Not all Noether identities are independent. 
\end{prop}

\begin{rmk}
What separates the physical gauge transformations
from the trivial transformations are whether or not they give us
independent Noether identities.
This is because these constrain the structures of the correlation functions.
\end{rmk}

\begin{exm}
If we take $\dd_\e \phi^i = R^i_\al \e^\al$
and define a different transformation:
\begin{equation}
\dd_\eta \phi^i = R^i_\al M^B_A\left( \phi \right)\eta^A
\end{equation}
for an arbitrary matrix $M$, 
then trivially:
\begin{equation}
\frac{\dd S_0}{\dd \phi^i} R^i_\al M^\al_A = 0
\end{equation}
\end{exm}

Define the so-called \emph{generating set} $G$ in $\barr{\cG}$\footnote{
If we were able to reduce to $\cG$ using the semidirect product,
then we would define $G$ within $\cG$.}
to be any $G$ which generates all of the Noether identities. 

\begin{wrn}
Such a $G$ is not necessarily even a subalgebra. 
\end{wrn}

\begin{prop}
$G$ is a generating set if any gauge transformation (that is any $\dd \phi^i$
such that 
\begin{equation}
\dd \phi^i \frac{\dd S_0}{\dd \phi^i} = 0
\end{equation}
is satisfied) 
can be written 
\begin{equation}
\dd \phi^i = \lam^\al\left( \phi \right)R^i_\al
+ M^{ij} \frac{\dd S_0}{\dd \phi^j}
\end{equation}
for $R^i_\al \in G$ and such that $M^{ij} = -M^{ji}$.
\end{prop}

We will take this characterisation as an effective definition.

\begin{prop}
For any two elements of $G$, 
\begin{equation}
\left[ R_\al^i , R_\b^j \right]
\end{equation}
is a gauge transformation as well, so indeed this must be expressible in the form:
\begin{equation}
\dd \phi^i = \lam^\al\left( \phi \right)R^i_\al
+ M^{ij} \frac{\dd S_0}{\dd \phi^j}
\end{equation}
\end{prop}

This means we can write:
\begin{equation}
R^j_\al \frac{\dd R^i_\b}{\dd \phi^j} - 
R^j_\b \frac{\dd R^i_\al}{\dd \phi^j} = 
C^\gamma_{\al\b}\left( \phi \right)R^i_\gamma + M^{ij}_{\al\b}\left( \phi \right)
\frac{\dd S_0}{\dd \phi^j}
\end{equation}
So this is a general expression, but we can now consider some special cases.

\begin{enumerate}
\item Open vs. closed: If $M^{ij}_{\al\b} \neq 0$, then $G$ must be an open symmetric algebra.

\item Reducible vs. irreducible: If there exists some $\lam^\al$ such that
\begin{equation}
\lam^\al R^i_\al = N^{ij} \frac{\dd S_0}{\dd \phi^j}
\end{equation}
for $N^{ij} = -N^{ji}$ then $G$ is called reducible.
We can manually remove such cases to get a new $G$ such that no such $\lam$ exist so $G$ would
be called irreducible. 
\end{enumerate}

\begin{exm}
Consider topological YM. Start with a gauge field $A_\mu^a$. 
Then we have
\begin{equation}
\dd A_\mu^a = \dd_\mu^a\left( x \right) + 
D_\mu f^a\left( x \right)
\end{equation}
which led us to introduce the ghost $\psi_\mu^a\left( x \right)$
and the chose $c^a\left( x \right)$. But we can just gauge this away by setting
$\e_\mu = -D_\mu f$. This is an example of a reducible $G$.
To make this theory irreducible, we would have to introduce these ghost-for-ghosts 
$\phi^a\left( x \right)$.
\end{exm}

So now we can determine the smallest possible generating set $G$,
and try to calculate commutation relations, to see what this means for the structure of $G$.
In particular, we are interested to see if $G$ is a proper Lie algebra.
First we consider some examples:

\begin{exm}
Consider the abelian Chern-Simons gauge theory in $2+1$ dimensions. 
We can write the action:
\begin{equation}
S = \int A\ext F = 
\int \d{^3 x} \e^{\mu \nu \sigma}F_{\mu\nu} A_\sigma
\end{equation}
this has some gauge symmetries, and we are interested in which 
such symmetries give independent Noether identities.

We claim this theory is invariant under two symmetries. 
It is certainly invariant under $\U\left( 1 \right)$:
\begin{equation}
\dd A_\mu = \p_\mu A\left( x \right)
\end{equation}
and diffeomorphisms of our spacetime $M$:
\begin{equation}
\dd A_\mu = \xi \p_\rho A_\mu + \p_\mu \xi^\rho A_\rho
\end{equation}
The difference between these two is a trivial gauge transformation. In particular we can show:
\begin{exr}
Show that these two transformations only differ by a term which is proportional to
$F^{\mu\nu}$.
\end{exr}
This means they differ by something which gives us a trivial Ward identity. 
\end{exm}

\begin{exm}
Consider point-like particle quantum mechanics:
\begin{equation}
S_{QM} = \int \d{t} \left( p \dot q = H\left( p,q \right) \right)
\end{equation}
which has a large amount of gauge symmetry for any Hamiltonian $H$.
In particular, this is invariant under:
\begin{align}
\dd q = \e\left( t \right) \left( \dot q - \frac{\p H}{\p p} \right)
&&
\dd p= \e\left( t \right) \left( \dot p + \frac{\p H}{\p q} \right)
\end{align}
which are in $\barr{\cG}$.
In fact we can show that the algebra of the 
$\e\left( t \right)$ is isomorphic to $\Diff\left( \RR \right)$.
These are in fact also in our normal subgroup $\cN$, 
so the statement here should really be that there is no gauge symmetry, 
since these have no implication for the dynamics of the theory. 
That is, $\cG$ is trivial.
\end{exm}

\subsection{Cohomological approach}

So we have seen that we should not be thinking of $\barr{\cG}$ as the gauge group.
We should first try for $\cG$, and if not we should reduce to this minimal $G$. 
The cost we might pay would be that
$\left[ g , g' \right]$ doesn't close to a Lie algebra.
We consider only the irreducible case.
Then we can define the \emph{stationary surface} to be the surface $\Sigma$
where the equations of motion:
\begin{equation}
\frac{\dd S_0}{\dd \phi^i} = 0
\end{equation}
are satisfied. 
We have the following theorem:

\begin{thm}
Generators $X_i$ of $G$
\begin{equation}
\left[ X_i , X_j \right] \propto X_k
\end{equation}
form an integrable distribution on this stationary surface $\Sigma$
in the sense that they generate gauge orbits.
\end{thm}

We want to determine what it means for something to be an observable in this context.
As we have often done, we now briefly return to the Hamiltonian formalism to make sense of things here. 
We know that in Hamiltonia QM, the observables are $C^\infty$ functions on phase space.
What about for some complicated quantum field theory?
In particular, for some theory with gauge invariance, what is the appropriate phase space?
Well in some sense we want to integrate over all possible histories, so we want 
to consider initial conditions for the equations of motion, which are equivalent to solutions
of the equations of motion. 
We take the covariant phase space to be the space $\Sigma$ of classical solutions. 
From the previous theorem, we should be identifying observables with functions on 
the space of orbits in $\Sigma$ which is clearly $\Sigma / G$.
So we have the space of all fields $\Phi$ on $\Sigma$, then the orbits of $G$ on $\Sigma$,
which gives a projection to $\Sigma / G$.

But how do we identify the collection of smooth functions $C^\infty\left( \Sigma / G \right)$?
To do this we define a charge $s$ which plays the role of the BRST charge $Q$. 
That is, we want $s^2 = 0$ to use the tools of cohomology. 
We introduce a grading by the ghost number. 
In particular, we take the ghost number of $s$ to be $1$, 
and we take
\begin{equation}
H^0\left( s \right) = C^\infty\left( \Sigma / G \right)
\end{equation}
We will also find the novelty that
\begin{equation}
sA = \left( A,S \right)
\end{equation}
for any observable (or more generally any function) $A$
where $S$ starts as the classical action of the theory and gets quantum corrections.

So now we desire to see this from first principles.
First we localize to $C^\infty\left( \Sigma \right)$, 
and second we incorporate $G$.
The first step is resolved using the Koszul-Tate resolution.
For any operator $\dd$, we have
\begin{equation}
H_0\left( \dd \right) = \frac{\ker \dd}{\im \dd} = C^\infty\left( \Sigma \right)
\end{equation}
Functions from this space which vanish on the surface $\Sigma$
form an ideal which we call $\cI$. 
Then we have
\begin{equation}
C^\infty\left( \Sigma \right) = C^\infty\left( \Phi \right) / \cI
\end{equation}
So if we want to identify a $\dd$ which gives us this space, 
we need to construct $\dd$ such that:
\begin{align}
\ker \dd = C^\infty\left( \Phi \right)
&&
\im \dd = \cI
\end{align}
so define $\dd \phi^i = 0$ for the first requirement, 
and for the second, for each $\phi^i$ we introduce
an anti-field $\anti{\phi_i}$, where we take:
\begin{equation}
\dd\anti{\phi_i} = - \frac{\dd S_0}{\dd \phi^i}
\label{eqn:brst_bv_cob}
\end{equation}
which gives us such a $\dd$.
Clearly since $\dd$ has ghost number $1$, $\gnum{\anti{\phi_i}} = -1$. 
We also define an antighost number which assigns $\agnum{\anti{\phi_i}} = 1$.
Now in the absence of gauge symmetry, $H_p\left( \dd \right) = 0$.
However if we have any nontrivial gauge symmetry, and therefore a Noether identity, 
we have a nontrivial equation:
\begin{equation}
\frac{\dd S_0}{\dd\phi^i} R^i_\al = 0
\end{equation}
which will ruin this identically zero homology. The RHS in \eqref{eqn:brst_bv_cob} 
is degenerate in this case. Now by this identity, we can in fact write:
\begin{equation}
\dd\left( R^i_\al \anti{\phi_i} \right) = 0
\end{equation}
so this is a nontrivial cohomology element.
So now introduce $\anti{\phi_\al}$ such that
\begin{equation}
\dd \dual{\phi_\al} = R^i_\al \dual{\phi_i}
\end{equation}
We introduce one of these for every element of $G$. 
Since $\gnum{\dd} = 1$, $\agnum{\dual{\phi_\al}} = 2$.
So if we are in the irreducible case, then there will be a potential obstacle at
antighost number $2$. 
So we have to introduce some objects with antighost number $3$ to kill this nontrivial cohomology.
We do not pursue this further.

Consider only the space $\Sigma$. Now we treat this as having a fibration as given by the orbits
of $G$. So we project each orbit to this space $\Sigma / G$.
So the punchline here is that this reduction $C^\infty\left( \Sigma / G \right)$ is just
the old BRST procedure.
To see this we define vertical vectors to be the 
vectors on $\Sigma$ tangent to the orbits $\cO$.
We also define vertical $p$-forms to be $p$-forms along the orbits. 
Then we want to define an exterior derivative $d$ on these forms such that it 
takes vertical $p$-forms to vertical $p+1$-forms. 
For $p = 0$ define
\begin{equation}
\d{F}\left( X \right) = \frac{\p}{\p X} F
\end{equation}
If we have $\al$ some vertical $1$-form we define:
\begin{equation}
\d{\al}\left( X , Y \right) =
-\cL\al\left( x \right) - \cL_x \al\left( y \right)
+ \al\left( \left[ X , Y \right] \right)
\end{equation}
where $\cL$ is the Lie derivative. 
Then we claim $d^2 = 0$, and that $d$ plays the role of the BRST charge.
Now pick a basis of vertical vectors:
\begin{equation}
X_\al F = \frac{\dd F}{\dd \phi^i} R_\al^i
\end{equation}
which satisfy commutation relations:
\begin{equation}
\left[ X_\al , X_\b \right] = C_{\al \b}^\gamma\left( \phi \right) X_\gamma
\end{equation}
on $\Sigma$. So now introduce fields $\left\{ c^\al \right\}$ called ghosts
to be dual to the $\left\{ X_\al \right\}$. 
These are indeed the same ghosts form the typical BRST story. 
Now define
\begin{align}
\d{F} = \left( X_\al F \right) c^\al
&&
\d{c^\al} = \frac{1}{2} \tensor{c}{_{\al \b}^\gamma} c^\b c^\gamma
\end{align}

So we want the BRST charge to satisfy both of these.
That is, we want some differential which satisfied the notable qualities of both $\dd$ and $d$. 
We might guess it is $\dd + d$, but $\dd d + d \dd \neq 0$, 
so we need to add more terms. This process is called Homological perturbation theory.
But there is actually a shortcut to showing this order by order. 
It turns out the action of $s$ is just given as the antibracket which is given by some function $A$
and acts on $S$:
\begin{equation}
sA = \left( A , S \right)
\end{equation}

So we have fields 
$\phi^i$ and $\anti{\phi_j}$
and we define 
\begin{align}
\left( \phi^ , \dual{\phi_j} \right) = \dd_j^i
&&
\left( c^\b , \dual{\phi_\al} \right) = \dd^\b_\al
&&
\cdots
\end{align}
which would continue if we had ghosts for ghosts etc.
Now we insist on the following three properties:
\begin{enumerate}
\item $\gnum{\left( A , B \right)} = \gnum{A} + \gnum{B} + 1$
\item $\e\left( \left( A , B \right) \right) = \e\left( A \right) + \e\left( B \right) + 1$
\item $\left( A , B \right) = -\left( -1 \right)^{\left( \e_A + 1 \right)\left( \e_B + 1 \right)}
\left( B , A \right)$
\item Jacobi:
$\left( -1 \right)^{\left( \e_A + 1 \right)\left( \e_B + 1 \right)}
\left( A , \left( B , C \right) \right) + \text{ cyclic permutations } = 0$
\end{enumerate}

In addition we have  $\gnum{S} = 0$ and $\e\left( S \right) = 0$ along with
what is called the master equation:
\begin{equation}
\left( S , S \right) = 0
\end{equation}
So finding a solution for the master equation amounts to finding such an operator $s$. 
We can also write $S$ as a sum of various objects with various ghost numbers:
\begin{equation}
S = \sum_{m\geq 0} S^{\left( m \right)}
\end{equation}
Then $S$ has the following properties:
\begin{enumerate}
\item $S^{\left( 0 \right)} = S$
\item $\left( \anti{\psi_i} , S\left( \phi_i \right) \right) = \dd \anti{\phi_i}$
\item $S^{\left( 1 \right)} = \anti{\phi_i}R^i_\al c^\al$
\end{enumerate}

\subsection{Quantization}

Now we can finally perform quantization of this theory.
The first step is to introduce the antighosts $b_A, B_A$. 
But now we have to introduce antifields $\anti{b_A}$ and $\anti{B_A}$
in anticipation of the gauge fixing terms. 
Then we simplify notation by introducing a collective name
\begin{align}
\phi^{\tilde{A}} = \left( \phi^i , c^\al , b_A , B_A, \cdots \right)
&&
\anti{\phi_{\tilde{A}}} = \left( \anti{\phi^i} , \anti{c^\al} , \anti{b_A} , \anti{B_A}, \cdots \right)
\end{align}
Then define
\begin{equation}
\cZ^{\tilde{a}} = \left( \phi^{\tilde{A}} , \anti{\phi_{\tilde{B}}} \right)
\end{equation}
and the antibracket on these to be:
\begin{equation}
\left( A,B \right) = 
\frac{\dd^r A}{\dd \cZ^{\tilde{a}}}
\z^{\tilde{a}\tilde{b}}\frac{\dd^l B}{\dd \cZ^{\tilde{b}}}
\end{equation}
where
\begin{equation}
\cZ^{\tilde{a}\tilde{b}} =
\begin{pmatrix}
0 & \dd^{\tilde{A}}_{\tilde{B}} \\
-\dd_{\tilde{B}}^{\tilde{A}} & 0
\end{pmatrix}
\end{equation}

Now we have $N$ gauge symmetries which still exist, so we set
\begin{equation}
\Om^{\tilde{A}}\left( \cZ^a \right) = 0
\end{equation}
while maintaining $s^2 = 0$. 
Then it turns out that it is equivalent to satisfy the involution:
\begin{equation}
\left( \Om^{\tilde{A}} , \Om^{\tilde{B}} \right) = 0
\end{equation}
So if these satisfy this, then they all follow from one gauge fixing fermion
$\Phi\left( \phi^A \right)$ and we have expressions of the form:
\begin{equation}
\anti{\phi_A} = 
\frac{\dd \Psi}{\dd \phi^A}
\end{equation}
So the procedure to get back to our familiar BRST procedure is to write down the action for the 
theory, and substitute this gauge fixing condition.

We will be returning to this formalism to discuss its relationship with anomalies, but
first we offer an overview of large $N$ theories
and the AdS / CFT correspondence. 

\chapter{Large \texorpdfstring{$N$}{N} (AdS / CFT correspondence)}

\section{Historical context}

We now consider a non-perturbative technique known as taking the
``large $N$ limit.'' We aim to give some motivation and overview as to why
this is called this and provide some historical context.
The parallel we should really have in mind here is bosonization.
Recall that in our treatment of this we first considered the Thirring model
and we discovered a duality between this and the sine-Gordon theory.

The first real application of this formalism came in H.E. Stanley's publication
\cite{stanley_68} in 1968.
In particular, he proposed to start with a spin system Hamiltonian:
\begin{equation}
H =  -J \sum_{\lr{ij}} \vec{s}_i \vec{s}_j
\end{equation}
where $\vec{s}_i$ is an $\O\left( n \right)$ vector
and we are summing over nearest neighbors on some lattice.
These are surprisingly rich statistical theories.
For $n = 0$ we have self avoiding walks on the lattice, 
$n = 1$ is the so-called Ising model, $n = 2$ 
gives the $XY$ model, and $n = 3$ is the Heisenberg model.
Now Stanley suggests that we take a $1/n$ expansion. 
In order for this to be a legitimate expansion, we of course have to take $n\fromto \infty$.
As a part of this process, we have to decide what we keep fixed as we take this limit.
For example we might allow $J$ to scale. 

The large $N$ approach was brought to high energy theory by G. 't Hooft in 1974
\cite{hooft_large_n}.
Recall the action from QCD:
\begin{equation}
S_{\text{QCD}} = \frac{1}{2 g_\ym^1} \int \tr\left( F_{\mu\nu} F^{\mu\nu} \right) + 
\int \barr{\Psi}\fsl{D} \Psi + m \barr{\Psi} \Psi
\end{equation}
but we have seen that this theory exhibits dimensional transmutation.
That is $g_\ym$ turns itself into a function of the scale, which means
all we really have is a characteristic energy scale of the theory.
So we don't have a free renormalized coupling, but rather
we end up with $\Lam_{\text{QCD}}$ which is 
the scale where $g_\ym\left( \Lam \right)\sim 1$. 
But we can't expand in terms of $\Lam$, so we cannot use perturbative techniques in this way. 

But 't Hooft suggested we can consider a different parameter. 
For this $F_{\mu\nu}$ we have the gauge group $G = \SU\left( N \right)$
Now we can use $1/ N$ as an expansion parameter as Stanley did for statistical mechanics.
Again we have to take a large $N$ limit. 
It turns out this is even more practical than bosonization because it
is easier to generalize to arbitrary quantum field theories.

\section{\texorpdfstring{$1+1$}{1+1} dimensions}

Recall the Thirring model which we covered in our treatment of bosonization. 
On some flat spacetime, we have some self interacting fermions:
\begin{equation}
S_T = \int_{\RR^2} \d{^2 x} \left\{ 
\barr{\psi} i \fsl{\p} \psi - g_T\left( \barr{\psi} \gamma_\mu \psi \right)^2\right\}
\end{equation}
We consider a similar model called the Gross-Neveu model:
\begin{equation}
S_{GN} = \int_{\RR^2} \d{^2 x} \left\{ 
\barr{\psi} i \fsl{\p} \psi - \frac{g}{2} \left( \barr{\psi}\psi \right)^2\right\}
\end{equation}
where $\psi^a$ running from $a\in \left\{ 1, \cdots , N \right\}$ are Dirac spinors. 
This has a global $\SU\left( n \right)$ flavor symmetry, which extends to $\SO\left( 2n \right)$,
but we are only concerned with keeping track of the $\SU\left( n \right)$ symmetry. 
Note that in this case we of course have $\left[ g \right] = \left[ g_T \right] = 0$.
Classically this theory has a discrete $\ZZ / 2\ZZ$ chiral symmetry. 
This action is given by the chirality matrix $\gamma_3$ which acts as:
\begin{align}
\psi \fromto \gamma_3 \psi
&&
\barr{\psi} \fromto -\barr{\psi}\gamma_3
\end{align}
This theory also has no classical mass. 
But just as we saw before, we might meet some non-perturbative techniques to get that
$g$ is asymptotically free.
Then this may lead to the so-called ``running'' of the coupling constan which give us
a non-perturbative mass scale.

For $N = \infty$ and no $1/N$ corrections we have the following Feynman rules:
\begin{equation}
\begin{tikzcd}
a\arrow[fermion,dash]{r}&
b
\sim \dd_{ab}
\end{tikzcd}
\end{equation}
\begin{equation}
g\sim
\begin{tikzcd}
a\arrow[fermion,dash]{dr}&&
a\\
& \arrow[fermion,dash]{ur} \arrow[fermion,dash]{dr}&
\\
b\arrow[fermion,dash]{ur}&&b
\end{tikzcd}
\equiv
\begin{tikzcd}
a\arrow[fermion,dash]{dr}&&
a\\
& \vdots\arrow[fermion,dash]{ur} \arrow[fermion,dash]{dr}&
\\
b\arrow[fermion,dash]{ur}&&b
\end{tikzcd}
\end{equation}
Now we consider the scattering of $a,\barr{a} \fromto b , \barr{b}$. 
\begin{equation}
\begin{tikzcd}
\, &
b\arrow[fermion,dash]{dr}&&
b\\
&& \, \arrow[fermion,dash]{ur} \arrow[fermion,dash]{dr}&
\\
\, \arrow{uu}{t} & 
a\arrow[fermion,dash]{ur}&
&
a
\end{tikzcd}
\end{equation}
Now there is no naive way to take $N\fromto \infty$. We have to rescale the parameters in 
the theory as we take this limit. Typically there is a unique way to do this such that
the theory is nontrivial and finite. We fix the so-called 't Hooft coupling $\lam^2 = gN$.

Now we have
\begin{equation}
S = \int \d{^2 x} \left\{ \barr{\psi}^a i \fsl{\p} \psi^a - 
\frac{\lam^2}{2N}\left( \barr{\psi} \psi \right)^2\right\}
\end{equation}
It might be interesting to integrate in some auxiliary field $\sigma\left( x \right)$.
We can then rewrite this action as:
\begin{equation}
S = \int \d{^2 x} \left\{ 
\barr{\psi} i \fsl{\p}\psi - \sigma \barr{\psi} \psi + \frac{N}{2\lam^2} \sigma^2\right\}
\end{equation}
since we can integrate out $\sigma$ to get the original action. 
At the classical level, the EOM for $\sigma$ is just:
\begin{equation}
\sigma= \frac{\lam}{N} \barr{\psi} \psi
\end{equation}
We get an honest propagator for this field:
\begin{equation}
\sigma\sim
\begin{tikzcd}
\cdot \arrow[ghost,dash]{r}& \cdot
\end{tikzcd}
\end{equation}
but now we can integrate $\psi , \barr{\psi}$ out with a Grassmanian Gaussian.
This is perturbatively equivalent to summing an infinite sequence of diagrams
to obtain an effective potential for $\psi$. This can be seen in 
\cref{fig:large_n_loops}.
Notice that the orders of the terms in \cref{fig:large_n_loops} are given by:
\begin{align}
\cO\left( g \right) = \cO\left( \lam^2 / N \right)
&&
\cO\left( g^2 \right) = \cO\left( \lam^4 / N^2 \right)
\\
\cO\left( g^2 N \right) = \cO\left( \lam^4 / N \right)
&&
\cO\left( g^3 N^2 \right) = \cO\left( \lam^6 / N \right)
\\
\cO\left( g^3 N^2 \right) = \cO\left( \lam^6 / N \right)
\end{align}
\begin{figure}
\centering
\includegraphics[width=\textwidth]{loops.pdf}
\caption{Loop diagrams for the $a, \barr{a} \fromto b , \barr{b}$ scattering:}
\label{fig:large_n_loops}
\end{figure}
Explicitly we calculate:
\begin{align}
-iV\left( \sigma_0 \right) =& -i \frac{N}{2\lam^2} \sigma_0^2 - 
N\sum_{r = 1}^\infty \frac{1}{2r}
\Tr \int \frac{\d{^2 p}}{\left( 2\pi \right)^2}\left( -\frac{\fsl{p}\sigma_0}{p^2} \right)^{2r}
\\
V =& \frac{N}{2\lam^2} \sigma^2 - N\int \frac{\d{^2 p}}{\left( 2\pi \right)}
\log\left( 1 + \frac{\sigma_0}{p^2} \right)
\end{align}
Now doing dimensional regularization, $\barr{MS}$ scheme, we get
\begin{equation}
V = N\left\{ 
\frac{\sigma^2}{2\lam^2} + \frac{\sigma_0^2}{4\pi}
\left( \log \frac{\sigma_0^2}{\mu^2} - 1 \right)
\right\}
\end{equation}
This also satisfies a renormalization group equation:
\begin{equation}
\left\{ 
\mu \frac{\p}{\p\mu} + \b\left( \lam \right) \frac{\p}{\p \lam} - 
\gamma_\sigma\left( \lam \right)\sigma \frac{\p}{\p\sigma}
\right\}V\left( \sigma \right) = 0
\end{equation}
where $\gamma_\sigma$ is the anomalous dimension of $\sigma$.
Notice this is satisfied for $\gamma_\sigma\left( \lam \right) =0$
and $\b\left( \lam \right) = -\lam^3 / \left( 2\pi \right)$.
Now the novelty is that this result is exact to all $\lam$ orders as $N\fromto \infty$
so we have found the exact $\b$ function. 
We can also notice from the sign of $\b$ that the theory is asymptotically free.
Finally we get that there is indeed dimensional transmutation present in this theory.
So even though the theory originally
had no scale, through the RG process, we develop a dependence on an effective scale. 
One way of showing this is to just fish for the minima of the effective potential. 
There is a naive one at $\sigma = 0$, and then at $\lr{\sigma} = \pm \mu e^{-\pi / \lam^2}$.
And since this is nonzero, this is the analogue of $\Lam_{\text{QCD}}$ from before. 
\begin{rmk}
Any time we have $e^{-\# / G^2}$ for some coupling $G$, this expression 
around $G = 0$ has the trivial expansion $0$.
This is a hallmark of a nonperturbative effect, since it cannot be seen to any order in perturbation
theory.
\end{rmk}
Now returning to our effective potential, we can substitute $\lr{\sigma}$ for $\sigma$
and we get chiral symmetry breaking.

Recall that we just made a guess for this form of $\sigma$.
This was really only easy to guess here. 
Treating something more complicated such as QCD makes this much more difficult.

Also notice, that if we write the theory in the language of $\sigma$, 
then we can treat $\h\sim 1/ N$. 
This is justified by the original path integral when we have integrating out $\psi$:
\begin{equation}
\int \cD \barr{\psi}\cD\psi \cD \sigma
e^{-S\left( \psi , \barr{\psi} , \sigma \right)}
\fromto 
\int \cD \sigma e^{-S_{\eff}\left( \sigma \right)}
\end{equation}
where
\begin{equation}
S_\eff\left( \sigma \right) = N\tilde S_\eff\left( \sigma , \lam \right)
\end{equation}
If we didn't rewrite the theory in the language of $\sigma$ so we just keep
the original $\psi$ variables, we might wonder if we would have been able to show the same thing?
Well we can take $\Psi = \psi / \sqrt{N}$ and get
\begin{equation}
S_{\text{GN}} = N\int \Psi^a \fsl{\p} \Psi - \lam^2\left( \barr{\Psi}^a \Psi^a \right)^2
\end{equation}
but there is this implicit dependence on $N$ within the integral given by the index $a$.
Since $\sigma$ was an $\SU\left( n \right)$ singlet, 
we didn't have the dependence on $N$ within the integral before.


\section{Riemannian surfaces}

Before we apply this methodology to a general gauge theory, we develop some formalism. 
Consider all two-dimensional orientable surfaces. 
We desire to classify such surfaces. 
As it turns out, we can classify these surfaces by their genus. This is effectively just
the number of handles. We can see this in \cref{fig:genus}.
\begin{figure}
\includegraphics[width=\textwidth]{genus.pdf}
\caption{Illustrations of the genus of a Riemann surface.}
\label{fig:genus}
\end{figure}

We can also classify compact orientable surfaces with the Euler number.
For a surface $\Sigma$ of genus $g$, this is given by 
$\chi\left( \Sigma \right) = 2 - 2g$. Without appealing to the genus, we can write this as:
\begin{equation}
\chi\left( \Sigma \right) = \sum \left( -1 \right)^k \dim
H^k_{\text{de Rham} }\left( \Sigma \right)
\end{equation}
Recall the following De Rham cohomologies:
\begin{align}
H^0 = \RR
&&
H^1 = \RR^2 
&&
H^2 = \RR
\end{align}
As an example we can see that $\chi\left( T^2 \right) = 0$.
We can also calculate this by integrating the curvature:
\begin{equation}
\chi\left( \Sigma \right) = \frac{1}{4\pi} \int \d{^2 \sigma} \sqrt{g} R
\end{equation}

As a final method for calculating the Euler characteristic, we can 
consider the simplicial decomposition of a surface. 
This just means we compose the surface into its irreducible elements.
So we preserve the space topologically, but 
decompose it into vertices, edges, and plaquettes. 
For example, the sphere can be topologically deformed to a tetrahedron, 
which provides the explicit simplicial decomposition
where we have $4$ vertices, $6$ edges, and $4$ surfaces. Then the Euler characteristic is given by:
\begin{equation}
\chi\left( \Sigma \right) = \# V - \# E + \# L
\end{equation}
where $L$ gives the number of closed loops, or plaquettes. 
We can check that for the sphere this agrees with the value of $\chi\left( S^2 \right) = 2$
that we have seen via other methods.

\begin{defn}
A planar graph is 
a graph which can be embedded into a surface of genus $0$ without crossing edges. 
\end{defn}

In other words, a planar graph is just a graph that can be drawn on a sphere or plane.
As motivated by the definition of a planar graph and the simplicial decomposition, 
if a finite, connected, planar graph is drawn in the plane without any edge intersections,
then the Euler characteristic of a planar graph (which is just $v - e + f$)
must be $2$.

\section{Generic treatment}

We now treat a generic QFT with some fields $M$, which are
matrices of some Lie group such as $\U\left( N \right)$, 
$\SU\left( N \right)$, $\SO\left( N \right)$, $\Sp\left( n \right)$ etcetera. 
We take $\U\left( N\right)$ for simplicity. 
We label this matrix as $\tensor{M}{^i_j}$, and assume it is Hermitian.
We have the $\U\left( 1 \right)$ action (gauge or global, this argument doesn't care)
given by:
\begin{equation}
\U\left( 1 \right): M\fromto U M U^\dag
\end{equation}
We want to treat any theory with a Lagrangian which we can write 
in terms of $M$ as:
\begin{equation}
S = \frac{1}{g_c^2}\int \Tr\left\{ \left( \p M \right)^2
+ \ubr{M^2 + M^3 + \cdots}{V\left( M \right)} \right\}
\end{equation}
We write $g_c$ for the coupling constant for reasons which will become obvious below.
Now we make these large $N$-limit considerations. 
We will see that every such field theory under this large-$N$
expansion is a string theory.
In particular, we eventually define $1/N$ as the string coupling constant.

Now we write Feynman diagrams, and determine which loops can be kept or discarded in this limit.
Since $M$ now has two indices, each propagator looks like a ribbon here,
and the Feynman diagrams become ribbon diagrams.
\begin{equation}
\begin{cd}
i&&
l\arrow[fermion,dash]{ll}\\
j\arrow[fermion,dash]{rr}&&
k
\end{cd}\sim
\lr{
\tensor{M}{^i_j} \tens{M}{^k_l}
}
\end{equation}
\begin{rmk}
If we were using $\SO\left( n \right)$ rather than $\SU\left( n \right)$ we would have
that these are unoriented, so these would just be double lines. 
\end{rmk}
We can think of these as thickened lines.
Now what happens as $N\fromto \infty$?
Take $P$ to be the number of propagators. This goes as $g_c^2$.
Each vertex will go as $1 / g_c^2$. 
Each loop $L$ goes as $N$. See \cref{fig:ribbon_loop} for an example of such a loop.
\begin{figure}
\includegraphics[width=\textwidth]{ribbon_loop.pdf}
\caption{Example of a one-loop ribbon diagram.}
\label{fig:ribbon_loop}
\end{figure}
Now for each diagram, just from the combinatorics we get:
\begin{equation}
\left( g_c^2 \right)^{\# P - \# V}N^{\# L} = \left( g_c^2 N \right)^{\# P - \# V}
N^{\# V - \# P + \# L}
\end{equation}
This leads us to anticipate a t' Hooft coupling
$\lam = g_c^2 N$. We now claim that:
\begin{equation}
N^{\# V - \# P + \# L}
= \left( \frac{1}{N} \right)^{2g - 2}
\end{equation}
for some $g\in \NN$ which will eventually be the genus of some surface.

For a general ribbon diagram, we can consider it as a graph. As such, we can
consider whether or not it is planar. 
To see this, we fill in all of the ``closed'' loops with a disk, and see if we cover 
the full plane. 
Alternatively we can check if any edges cross. 
So in \cref{fig:ribbon_loop} we fill in the inner circle
and the portions above and below, so this is clearly a planar graph.
In \cref{fig:ribbon_2} on the left, we see another example of a planar graph.
Also notice that as mentioned in the previous section, this graph does
indeed have Euler characteristic $2 = 2 - 3 + 3$.
However in \cref{fig:ribbon_2} on the right, this is not a planar graph. Note that
it has Euler characteristic $4 - 6 + 2 = 0\neq 2$. 
It can however be drawn on the torus $T^2$.
\begin{figure}
\centering
\includegraphics[width=0.4\textwidth]{ribbon_2.pdf}
\includegraphics[width=0.4\textwidth]{ribbon_3.pdf}
\caption{More examples of ribbon diagrams.}
\label{fig:ribbon_2}
\end{figure}
Then we have the following statement:

\begin{thm}
Each such ribbon diagram is associated uniquely with a Riemann surface
of some genus $g$.
\end{thm}

Now take
\begin{align}
\cZ =& \sum_{\text{diag.}} \left( \frac{1}{N} \right)^{2g - 2}
= \sum_{g = 0}^\infty \left( \frac{1}{N} \right)^{2g - 2} 
\sum_{\text{diag. of genus }g} f\left( \lam \right)
\\=& \sum_{g = 0}^\infty g_{\text{string}}^{2g - 2} F_g\left( \lam \right)
\end{align}
where as we mentioned in the beginning, we have taken $1/N$ as our string coupling
$g_{\text{string}}$.
A string theory can be pragmatically defined to be a theory which rather than 
an expansion into graphs with $1$-dimensional propagator lines, 
has an expansion of Feynman graphs with the geometry of $2$-dimensional surfaces.

\section{AdS space}

So we have this correspondence between large $N$ theories of matrices and string theories. 
But we don't have a good prescription of calculating these $F_g$ coefficients. 
Indeed, very few ``string theories'' are known.
So which large $N$ theories of $\tens{M}{^i_j}$ can be described by known string theories?
This is essentially what led Maldacena to the AdS/CFT correspondence in 1997
\cite{ads_cft}.

In particular we seek quantum field theories with RG flow, and Poincar\'e symmetry. 
We follow our procedure from the past, 
and restrict our attention to the fixed points of the RG. 
It is easier to first classify these fixed points and then classify the flow in the RG. 
These fixed points will have an extension of the Poincar\'e symmetry to conformal symmetry. 
For example, $\SO\left( 3,1 \right)\rtimes \RR^4$ extends to the conformal symmetry
$\SO\left( 4,2 \right)$.
Then perhaps, we can use this symmetry to more uniquely identify a 
candidate for the dual string theory. 

It is beautiful to have this idea, but how many such points do we know besides free field theories?
One possible trick to get further, is to add SUSY. 
In particular, $N=4$ SUSY in $3+1$ dimensions was shown to be an exact CFT by Stanley Mandelstam.
So we have the fixed point of RG given by $c^\mu$ for $\mu = 0 , \cdots , 3$,
and we have this enhanced global $\SO\left( 4 , 2 \right)$. 
So we have the equation
\begin{equation}
\cZ\left( \lam , N \right) = \sum_{g = 0}^\infty 
g_{\text{string}}^{2g - 2}
F_g\left( \lam \right)
\end{equation}
If we can define these independently of one another, then they share the same Hilbert space.
We know $\SO\left( 4,2 \right)$ is a conformal symmetry of $\RR^{3,1}$,
but how do we realize this $\SO\left( 4,2 \right)$ on the string theory side?
We call the Riemann surface the string worldsheet, and postulate:
\begin{equation}
F_g\left( \lam \right) = \int \cD \left( \text{fields} \right) 
e^{iS_{WS}}
\end{equation}
where we have
\begin{equation}
S_{WS} = \int \d{^2 \sigma} \cL\left( f , \lam , \cdots \right)
\end{equation}
So say we have the worldsheet surface $\Sigma_g$ with coordinates $\sigma$. 
The simplest way to associate some fields with this string theory, is
to consider this spacetime $\cM$ with coordinates $x^\mu$, 
and consider maps from this $\Sigma_g$ to $\cM$. Then these are our candidate fields. 

The simplest worldsheet Lagrangian\footnote{
This is a cartoon version of what is called a ``Polyakov action.''}
is:
\begin{equation}
S_{WS} \sim \int \d{^2 \sigma} \p_\al x^\mu \p_\al x^\nu \eta_{\mu\nu}
\end{equation}
If we stare at this, we can see that it is indeed $\SO\left( 3,1 \right)$ invariant,
but is not $\SO\left( 4,2 \right)$ invariant. This seems to be very bad. 
As it turns out, the way to refine this Lagrangian, is to realize that the reason for this
is that the target space $\cM$ only has Lorentz invariance. So if we want the $\SO\left( 4,2 \right)$,
let's search for a target space $\hat \cM$ which has $\SO\left( 4,2 \right)$ as an isometry group.
This is easy enough to accomplish, since $\cM$ with coordinates $c^\mu$ 
already realizes our $\SO\left( 3,1 \right)$.
If we want to extend this Poincar\'e symmetry into something bigger, we have to add dimensions. 

The simplest way to do this is by brute force. Just introduce a new dimension $u$. 
There will be a metric on this bigger spacetime given by:
\begin{equation}
\d{s^2} = 
w^2\left( u \right)
\d{x^\mu}\d{x^\nu} \eta_{\mu\nu}
+
w\left( u \right)^2 \d{u^2}
\d{u^2}
\end{equation}
for some $w$. The first term term has this Minkowski term up to a multiple, 
so it still gives us the $\SO\left( 3,1 \right)$. 
Now we insist that this metric has $\SO\left( 4,2 \right)$ isometries. 
So add conformal rescalings of $\cM$: $x^\mu \fromto \lam x^\mu$. 
This forces us to also rescale $u\fromto \lam u$, 
which restricts our $w$. Namely, $w^2 = 1/ u^2$
will produce the desired invariant metric. 
So our prediction is that the string theory which is dual to this CFT should live
on a spacetime in $5$-dimensions, with metric:
\begin{equation}
\frac{L^2}{u^2} \left( \d{x_\mu}\d{x^\mu} + \d{u^2} \right)
\end{equation}
where $L$ is called the radius.\footnote{
We don't use $R$ because we will be discussing gravitational effects, and we
don't want to confuse this with the Ricci scalar.}
If this is set to $1$, the metric is just:
\begin{equation}
\d{s^2} = \frac{1}{u^2}\left( 
\d{x^\mu} \d{x_\mu} + \d{u^2}\right)
\end{equation}
This is a famous spacetime, called the anti-de Sitter (AdS) space in $4+1$ dimensions
with the Poincar\'e patch coordinates. We write this space as $\AdS_5$.
We call this $u$ coordinate the radial coordinate. 
We will see that the radial evolution is effectively the same as the RG flow.
The boundary $u = 0$ will correspond to the UV limit of a CFT. 
The IR limit is then just deep in the bulk of the geometry.
In other words, UV in CFT corresponds to the IR in gravity.

So returning to our $N = 4$ SYM in $3 +1$, we have $16$ $Q$s in our full super-algebra, 
and $16$ $S$s, meaning we have $32$ super-symmetries.
Now we get out $\SO\left( 4,2 \right)$, but we have more. 
Recall the $Q^I_\al$ all have $R$-symmetry index $I$, and each has $4$. 
So in fact, we have the bigger symmetry:
\begin{equation}
\SO\left( 4,2 \right) \tp \SU\left( 4 \right)
\end{equation}
As before, we still have this AdS space since we have the $\SO\left( 4,2 \right)$ symmetry, 
but now we have to introduce another manifold. 
Recall the accidental isomorphism $\SU\left( 4 \right) \simeqq \SO\left( 6 \right)$.
So instead of realizing $\SU\left( 4 \right)$, we can realize $\SO\left( 6 \right)$, 
which acts as rotations on a $6$ dimensional space. 
In particular, we predict that the string theory dual for this theory has
to be a SUSY theory whose ground state which looks like $\AdS_5$, has $32$ SUSY, 
and lives on $\AdS_5 \times S^5$. 

\section{Establishing the dictionary}

\subsection{Correlation functions}

Without even introducing SUSY, we
can evaluate the gravity partition function as a function of the boundary values
of the fields. We consider the bulk partition function:
\begin{equation}
\cZ_{bulk}\left[ \restr{\phi\left( \vec{x} , u \right)}{u = 0} = 
\phi_0\left( \vec{x} \right) \right] = \lr{\exp\left( \int \d{^4 x} \phi_0\left( \vec{x}\right)
\cO\left( \vec{x} \right) \right)}_{\text{CFT}}
\end{equation}
where we have specified the boundary conditions for the fields on the boundary of $\AdS$.

Now we will see that the dual string theory will always include gravitational effects. 
This means if we restrict our attention to theories of large $N$ matrices with a local energy
momentum tensor $T^{\mu\nu}$,
this implies there is a $g_{\mu\nu}$ excitation on the string side.
By this, we mean a propagating, massless, spin-$2$ particle.

\begin{exm}
In particular, we consider the example of a scalar field $\phi$ of mass $m$ in bulk:
\begin{equation}
S = \frac{1}{2}
\int \d{^4 x} \d{u} \sqrt{-G}
\left( G^{MN} \p_M \phi \p_N \phi - \frac{1}{2} m^2 \phi^2 \right)
\end{equation}
For the moment we consider solutions near the boundary of $\AdS$, which is given by
$u\approx 0$. In particular, we look for solutions of the form $\phi \sim u^\al$, where
\begin{equation}
\al\left( \al - 4 \right) - m^2 L^2 = 0
\end{equation}
then we can plug this back in, which restricts $\al$ to be:
\begin{equation}
\al_{\pm} = 2 \pm \sqrt{4 + m^2 L^2}
\end{equation}
Now we have that the $\al_-$ solution dominates near $u\fromto 0$, and the 
$\al_+$ solution always decays when $u\fromto 0$. 
This gives us the Breitenlohner-Freedman bound:
\begin{equation}
m^2 L^2 \geq -4
\end{equation}
\end{exm}

For $\phi$, the $\cO$ has conformal dimension $\Delta$, where
\begin{equation}
\Delta = 2 + \sqrt{4 + m^2 L^2}
\end{equation}

\subsection{Holographic renormalization}

We have seen this expression:
\begin{equation}
\cZ_{bulk}\left[ \restr{\phi\left( \vec{x} , u \right)}{u = 0} = 
\phi_0\left( \vec{x} \right) \right] = \lr{\exp\left( \int \d{^4 x} \phi_0\left( \vec{x}\right)
\cO\left( \vec{x} \right) \right)}_{\text{CFT}}
\end{equation}
but we have UV divergences on the RHS, meaning we need some regularization and renormalization.
Using the UV IR correspondence, this means we have an IR divergence on the LHS. 
To treat this, we take $u = \e$ for some small $\e$ and take this to zero, to see
how this blows up. The resulting procedure should reproduce the usual renormalization procedure
on the CFT side of the correspondence.

So evaluate $\phi\left( \vec{x} , u \right)$ at $u = \e$, which yields:
\begin{equation}
\restr{\phi\left( \vec{x} , u \right)}{u = \e} = 
\e^{\al_-} \phi_0^R\left( \vec{x} \right)
\end{equation}
What is the scaling (engineering) dimension of $\phi$?
Well we have the rescalings: $\vec{x}\fromto \lam \vec{x}$, 
$u\fromto \lam u$, and $\phi \fromto \phi$.
Therefore $\left[ \phi^R \right] = \al_-$.
But then we have that $\left[ \cO \right] = 4 - \al_- = \al_+$, 
which means
\begin{equation}
\Delta = 2 + \sqrt{4 + m^2 L^2}
\end{equation}
as we have already seen!
We have only shown this for the scalar field,
but this turns out to be true in general. 
Recall we have $\left[ T^{\mu\nu} \right] = 4$ which is not getting renormalized. 
This implies there is a massless source field $J_{\mu\nu}$ in the bulk.
We write this $g_{\mu\nu} \sub G_{\mu\nu}$. 
This is a massless spin-2 field, which must be embedded in a theory
local spacetime with gauge invariance. In other words, this must be a graviton.

Now let's take a limit where the LHS is approximated well by classical gravity.
In this limit, the effective action $\log \cZ_{\text{bulk}} = W_\eff$, 
will be
\begin{equation}
W_\eff = -S_{\text{on-shell}} \left[ \phi_0^R \right]_{\text{gravity}}
\end{equation}
We only keep track of the dependence of the space time metric $g_{\mu\nu}$,
which we have just argued for the existence of. So 
$W_\eff = -S_{\text{on-shell}} \left[ g_{\mu\nu} \right]_{\text{gravity}}$
and we are keeping the index $R$ implicit.
Now in order to evaluate this action, we need to figure out the evolution in the 
radial direction. 
It turns out this is governed by some Hamilton-Jacobi equation. 
There is however an algorithmic way of evaluating this on-shell action. 
First of all, it will end up being local, 
\begin{equation}
S_{\text{on-shell}}=
\frac{1}{16 \pi G_N} \int \d{^4 x} \sqrt{-g} \cL
\end{equation}
where we can write:
\begin{equation}
\cL = 
\cdots
\frac{\cL{\left( 4 \right)}}{\e^4}+
\frac{\cL{\left( 2 \right)}}{\e^2}
\cdots
\end{equation}
note these are a bunch of divergent terms.
In particular they are exactly the power-law divergences on the CFT side. 
So by holographic renormalization, we drop all the terms except the finite ones. 
So we are left with:
\begin{equation}
\cL = \cL^{\left( 0 \right)} + \cO\left( \e \right)
\end{equation}
so this is the candidate for the renormalized Lagrangian.
In even spacetime dimensions on the boundary, there is still a leftover IR divergent term. 
That is, there is enough room for a logarithmic divergence:
\begin{equation}
\cL = \cL^{\left( 0 \right)} + \cO\left( \e \right)
+ \log \e \tilde \cL^{\left( 0 \right)}
\end{equation}
Knowing that the rescaling of the field theory corresponds to the translation
in the radial direction, 
we can ask how the renormalized action would behave if we performed this
translation in the bulk. In particular, we have a translation operator
$\dd_D$ which acts as:
\begin{equation}
\dd_D \cL^{\left( 0 \right)} \propto \tilde\cL^{\left( 0 \right)}
\end{equation}
which comes from this log divergence.
So this is elementary from the bulk point of view,
but from the CFT point of view, this means that the action is no longer invariant under
the rescaling of the Minkowski spacetime $\vec{x}\fromto \lam \vec{x}$,
so there is a conformal anomaly.
And if we can iteratively solve for these particular coefficients, we can find 
how much of a quantum anomaly is present in this classical action.

\subsection{Type IIB string theory}

We now present one honest list of relations for the most famous example of the AdS CFT
correspondence. This example serves as a sort of sounding board for potential applications
of AdS CFT correspondence.
The string theory here is of
type IIB on $\AdS_5\times S^5$, 
which will turn out to be equivalent to 
$\SU\left( N \right)$,
$\cN = 2$ SYM in $3+1$ dimensions. 
\mathtabular{
\begin{tabular}{|l|r|}
\hline
IIB on $\AdS_5\times S^5$, 
&
SYM
\\
\hline\hline
string length:
$l_s = \sqrt{\al'}$
&
$N$
\\ \hline
string coupling: $g_s$
&
$g_{\ym}^2$
\\\hline
$\tau_{IIB} = c_0 + i / g_s = 
\lr{C + ie^{-\Phi}}$
&
$\tau_\ym$
\\ \hline
\end{tabular}}
Note that $g_{\ym}^2 = 2\pi g_s$ and
\begin{equation}
\tau_\ym = 
\frac{\theta}{2\pi}
+ \frac{2\pi i }{ g_\ym^2}
\end{equation}
So why is the space here $\AdS_5\times S^5$?
Well IIB has $C_4$, 
and flux $F_5 = d C_4$.
So it has $N$ units of flux through $S^5$, that is
\begin{equation}
\int_{S^5} F_5 =  N
\end{equation}
Now schematically, we have the effective gravitational action:
\begin{align}
S_\eff =
\frac{1}{G_N}\int \sqrt{g}R + 
\int F_5^2
\end{align}
Then we claim that from string theory, we get that this coefficient $1/G_N$
is the same as $1 / g_s^2$, but $g_s$ is dimensionless, 
whereas $G_n$ has dimension $8$ in $10$ dimensions. 
The only parameter we have in string theory is $\al'$, which means this is really equal to
$1 / \left( \al'^2 g_s^2 \right)$. This means
\begin{equation}
S_\eff \sim
\frac{1}{\al'^2 g_s^2} L^8 + N^2
\end{equation}
These two things need to balance out, which means:
\begin{equation}
g_s^2 \frac{L^8}{l_s^8}\sim N^2
\end{equation}
So the dictionary is now very simple:
\begin{equation}
\boxed{
\tau_{IIB} = \tau_\ym
\qquad\qquad
2 g_\ym^2 N = 
\frac{L^4}{l_s^4}}
\end{equation}
and $\lam = g_\ym^2 N$ is our 't Hooft coupling.
So even though there is no $N$ 
explicitly on the CFT time, this $\tau_\ym$ effectively goes as $N$ since $\lam$ is fixed. 
Note that on both sides, there is a $\SL\left( 2 ; \ZZ \right)$ duality.
This group acts on these $\tau$s as:
\begin{align}
\tau \fromto \frac{a \tau + b}{c\tau + b}
&&
\begin{pmatrix}
a & b \\ c & d
\end{pmatrix}\in \SL\left( 2 ; \ZZ \right)
\end{align}

\section{Strength of the equivalence}

In general, AdS/CFT is certainly an equivalence of theories, but there is some level of
uncertainty regarding the formal strength of this equivalence.
In general people believe the following in a sufficiently super-symmetric setting:
\begin{con}[Strongest]
Quantum string theory, that is a string theory with:
\begin{align}
g_s \neq 0
&&
\frac{\al'}{L^2} \neq 0
\end{align}
is equivalent to SYM for any $N$ and any $\lam$.
\end{con}
There is the slightly more conservative version:
\begin{con}[Strong]
Classical string theory, that is a string theory with:
\begin{align}
g_s\fromto 0
&&
\frac{\al'}{L^2}\neq 0
\end{align}
is equivalent to SYM for $N\fromto \infty$ and $\lam$ fixed.
\end{con}
This weak form generally receives the most attention:
\begin{con}[Weak]
Classical (super) gravity, that is a theory of the form:
\begin{align}
g_s\fromto 0
&&
\frac{\al'}{L^2} \fromto 0
\end{align}
is equivalent to SYM for $N\fromto \infty$ and $\lam$ large.
\end{con}
so in this last case, we are forced into considering a strongly coupled gauge theory.

\section{Anomalies}

\subsection{Review of anti-bracket (BRST-BV) formalism}

We will mainly be using AdS/CFT later in our discussion of anomalies.
Recall the BRST formalism considers some symmetry generators
$\dd_\al$, fields $\phi^i$, ghosts $c^{\al}$, and antighosts $b$.\footnote{
The antighosts also have and index which we omit in this brief review.}
We collectively call the fields $\phi^A$. Let the number of fields be $N$.\footnote{
This is unrelated to the $N$ from the preceding construction of large $N$ theories.}
We also have the BRST charge $Q_{\text{BRST}} \equiv s$ which squares to $0$.
This defines a cohomology theory, which allows us to identify the Hilbert space:
\begin{equation}
\cH_{\text{physical}} = 
H_Q\left( \cH_{\phi} \tp \cH_{b,c} \right)
\end{equation}
This formalism is however insufficient for certain theories, so we had
to introduce a partner for all of these original fields. 
We call these anti-fields, and write them as:
$\anti{\phi}$, $\anti{c}$, $\anti{b}$.
Just as we did with the original fields, we call these collectively $\anti{\phi^A}$
where $A$ runs from $1$ to $N$.
These anti-fields always have the opposite statistics to their original counterparts.
We might now have to introduce ghosts-for-ghosts, but as long as we only introduce such fields
a finite number of times, we are able to treat this sufficiently well. 
Now these fields $\phi^A$ and $\anti{\phi_A}$ are coordinates on a $2N$ dimensional manifold. 
Call these coordinates $\cZ\left( \phi^A ,  \anti{\phi^A}\right)$.

In the extended formalism we define this $s$ to act as the so-called anti-bracket on any operator $A$:
\begin{equation}
sA \ceqq
\left( A , S \right)
\end{equation}
We should think of the anti-bracket as a sort of analogue of the Poisson bracket.
Recall that:
\begin{equation}
s \anti{\phi_i} = \dd \anti{\phi_i} = 
-\frac{\dd S_0}{\dd \phi^i}
\end{equation}
We should think of $S$ as an improved action. This starts out as a classical action, but
gets additional corrections from the additional fields we introduce.
We define the anti-bracket to be:
\begin{equation}
\left( A , B \right) =
\frac{\dd^r A}{\dd \cZ^a}
\z^{ab} \frac{\dd^l B}{\dd \cZ^b}
\end{equation}
where $\z^{ab}$ is a generalized symplectic form:
\begin{equation}
\z^{ab} = 
\begin{pmatrix}
0 & \dd^A_B \\ 
-\dd^{A}_B & 0
\end{pmatrix}
\end{equation}
Then we have
\begin{equation}
S = \sum_{m\geq 0} S^{\left( m \right)}
\end{equation}
where $m$ is the antighost number and we take:
\begin{align}
S^{\left( 0 \right)} = S_{inv}\left( \phi^i \right)
&&
S^{\left( 1 \right)} = \anti{\phi_i}R^i_\al C^\al
&&
\cdots
\end{align}
where $R^i_\al$ is the action of the gauge transformations on the original fields.
But how do we actually find $S$?
Essentially we just have to solve the master equation:
\begin{equation}
\left( S , S \right) = 0
\end{equation}
This is sometimes called the Zinn-Justin equation, since they used this equation in \cite{ym_renorm}
to show the renormalizability of Yang-Mills in four dimensions.\footnote{
This boils down to the fact that if we write down the most general version of
Yang-Mills gauge theory in 4d, and calculate loop corrections, it turns out it is not self-contained
under normalization. So to show that the theory is still renormalizable, we need anti-fields.
For more details see \cite{zinn_justin}.}
Any solution of the master equation still has gauge invariance.
In particular, this has exactly $N$ symmetries, which we fix by choosing
gauge fixing conditions $\Om^A\left( \phi , \anti{\phi} \right) = 0$.
In order for these to be consistent with the structure of the master equation and 
$s^2 = 0$, these conditions must be ``in involution.''
This just means $\left( \Om^A , \Om^B \right) = 0$. 
\begin{thm}
Then under this condition $\left( \Om^A , \Om^B \right) = 0$
these conditions all follow from one $\Psi$.
This is the gauge-fixing fermion.
\end{thm}
\begin{proof}[``Proof'']
First imagine that the $\Om^A = 0$ conditions can be solved 
for $\anti{\phi_A}$. 
If this is not the case, we have to make a complimentary assumption. 
In any case, this means
\begin{equation}
\anti{\phi_A} - \om_A\left( \phi \right) = 0
\end{equation}
Then:
\begin{equation}
\left( \Om^A , \Om^B \right) = 
\frac{\dd\om_A}{\dd\phi^B} - 
\frac{\dd\om_B}{\dd\phi^A}  =0
\implies
\om_A = 
\frac{\dd\Phi}{\dd \phi^A}
\end{equation}
\end{proof}
So we have this solution $S$ to the master-equation, and we can simply substitute:
\begin{equation}
S_{\Phi} = S\left( \phi , \anti{\phi} = \frac{\dd\Phi}{\dd\phi} \right)
\end{equation}
so we have effectively eliminated the anti-fields from our formalism.
Then the pragmatic BRST charge we will actual use for calculations is:
\begin{equation}
\barr{s}\phi^A = \left( s \phi^A \right)\left( \phi , \anti{\phi} = \frac{\dd\Psi}{\dd \phi} \right)
\end{equation}
This means $\barr{s}S_\Psi = 0$, which means we can 
write down Feynman rules. 
In addition, we have that $\barr{s}^2\phi^A$ is proportional to the field equations, which 
means we didn't do this all for nothing.

\subsection{Anomalies}

There are two types of anomalies.
First, there are anomalies arising from global symmetries, which are okay, 
and lead to important physical effects. 
In this case we want to understand how significant the anomalies are,
and under which symmetries they arise.
The second type are anomalies arising in gauge symmetries, which invalidate quantum mechanics
since we lose unitarity. This is bad, because anomalous gauge symmetric theories need to be
abandoned. 
Mathematically,  we treat them treat them equally in the sense that
we take global ones, and artificially promote these to gauge symmetries by hand.
We do this since we can use the BRST formalism to treat these local symmetries.
This is fine because we can always reverse this at the end of the calculation.

Consider some gauge fields $A$, 
and some effective action $\Gamma\left[ A \right]$. 
Call the gauge transformations:
$\dd_\al \equiv \cT_\al\left( x \right)$.
Under the gauge transformations, 
\begin{equation}
\dd_\al \Gamma\left[ A \right] \equiv
\cT_\al \tp \Gamma\left[ A \right]
=
\begin{cases}
0 & \h = 0 \\
G_\al\left[ x , A \right]
& \text{o/w}
\end{cases}
\end{equation}
so this is zero only in the classical limit, and otherwise we get the anomaly. 
Sometimes we can add local terms to $\Gamma$ and remove this anomaly by hand. 
But what sort of forms can $G_\al\left[ A \right]$ take in general?
It turns out we can use the anti-bracket formalism to restrict this significantly.
Recall we had the following:
\begin{equation}
\left[ \dd_\al , \dd_\b \right] = \tensor{f}{_{\al\b}^\gamma}\dd_\gamma
\end{equation}
which imply a consistency condition on $G_\al$. 
These are known as the Wess-Zumino self consistency conditions:
\begin{equation}
\left[ \dd_\al , \dd_\b \right]\Gamma\left[ A \right] = 
\tensor{f}{_{\al \b}^\gamma} \dd_\gamma \Gamma\left[ A \right]
\end{equation}
where
\begin{equation}
\cT_\al\left( X \right)G_\b - \cT_\b\left( x \right) G_\al = 
\tensor{f}{_{\al\b}^\gamma} G_\gamma\left[ A \right]
\end{equation}
Now we can simply postulate:
\begin{equation}
G_\al\left[ c , A \right] = \int 
c^\al\left( x \right)G_\al\left[ x , A \right]\d{^4 x}
\end{equation}
So we know $s\Gamma\left[ A \right]\neq 0$ and 
since $s^2 = 0$, we can summarize the self-consistency condition as:
\begin{equation}
sG\left[ c , A \right] = 0
\end{equation}
Therefore if we want to suggest an anomaly, then it better satisfy this condition.

So now it is easy to classify the anomalies.
We can only have $G$s such that:
\begin{enumerate}
\item $\gnum{G} = 1$
\item $sG = 0$
\item $G + s\Gamma_1 \simeqq G$
\end{enumerate}
where $\Gamma_1$ is local.

\subsection{Schwinger terms}

Consider equal time commutation relations of conserved currents.
\begin{equation}
\left[ J^0_\al\left( \vec{x} , t \right) , J_\b^0\left( \vec{y} , t \right) \right] = 
\tensor{f}{_{\al \b}^\gamma} J^0_\gamma\left( \vec{x} , t \right) \dd^{d-1}\left( \vec{x}- \vec{y} \right)
+ S_{\al \b}\left( \vec{x} , \vec{y} , t \right)
\end{equation}
Now in analogy with the treatment of anomalies, we can define 
\begin{equation}
S = 
\int 
\d{^{d-1}\vec{x}} \d{^{d-1}\vec{y}} 
c^\al \left( x \right)c^\b\left( y \right)S_{\al \b}\left( \vec{x} , \vec{y} , t \right)
\end{equation}
This object has ghost number $2$.
Then the claim is, that this object is the second cohomology with respect to the BRST charge,
which classifies the possible Schwinger terms:
\begin{enumerate}
\item $sS = 0$ by virtue of the Jacobi identity.
\item If $S = s\cJ$, then $\cJ$ has to be of the form $\int c^\al \cJ_\al$ for some $\cJ_\al$, 
which can be used to define shifted currents:
\begin{equation}
\tilde J_\al^0 = J_\al^0 \pm \cJ_\al
\end{equation}
which satisfy the expected commutation relation without the Schwinger anomaly.
\end{enumerate}

\subsection{Examples}

We first consider Weyl anomalies in relativistic systems. 
These are associated with transformations of the form:
\begin{equation}
g_{\mu\nu}\left( x \right) \fromto \Om^2\left( x \right) g_{\mu\nu}
\end{equation}
Note that if $g_{\mu\nu} = \eta_{\mu\nu}$, we get
$\Om^2$ is just some rescaling $\lam$, so these global conformal
transformations are related to RG flows.
That is, this is some sort of local version of conformal transformations.
So if we have some action $\Gamma\left[ g_{\mu\nu} \right]$, we might have an anomaly under these
so-called Weyl transformations. 

\begin{exm}
In $1 + 1$ dimensions, the anomaly will look like:
\begin{align}
\cA \propto G_\al\left[ g_{\mu\nu} \right] c\int \d{^{1+1}x} \sqrt{-g} R
&&
G = \frac{c}{\#} \int \mathbf{c} \sqrt{-g} R
\end{align}
where $\mathbf{c}^\al$ is the ghost, $c$ is the central charge, and $R$ is the Ricci scalar. 
Recall $\sqrt{-g}R$ is a topological invariant. 

In general, any $d - 1 , 1$ theory has a conformal $\SO\left( d , 2 \right)$ symmetry.
So for $d = 2$ we get 
\begin{equation}
\SO\left( 2 , 2 \right) \simeqq \SO\left( 2 , 1 \right) \dsum \SO\left( 2 , 1 \right)
\end{equation}
In addition to this, CFTs in two dimensions satisfy the so-called Virasoro algebra:
\begin{equation}
\left[ L_m , L_n \right] = 
\left( m-n \right)L_{m+n} + \frac{c}{12}m\left( m^2 - 1 \right)\dd_{m+n , 0}
\end{equation}
\end{exm}

In a unitary CFT, one can show that $c\geq 0$, and $c = 0$ only for the trivial theory, which 
has a Hilbert space with only the vacuum state.
This $c$ is useful because it is a quantitative measure of the number of degrees of freedom.
Note that it is normalized such that $c = 1$ for
\begin{equation}
S = \frac{1}{2} \int \left( \p \phi \right)^2
\end{equation}

\begin{thm}[Zamolodchikov]
In $1+1$ dimension, there exists a function $C\left( g_i , \mu \right)$,
where $\mu$ is the RG scale, such that $C$ decreases monotonically along the RG flow 
(from the UV to the IR), 
and at any RG fixed point $g_*$ we have $C\left( g_* \right) = c$.
\end{thm}

We now redo the previous example in $2d + 1$ dimensions, and we find out $H^1_s = 0$
so there are no Weyl anomalies in any odd spacetime dimension.

\begin{exm}
In $3 + 1$ dimensions, we have
\begin{equation}
\cA \sim a \int E_4 + cI
\end{equation}
where $E_4$ is the Euler number density:
\begin{equation}
E_4 \sim \sqrt{-g} \left( R^2_{\mu\nu\rho\sigma} - 4 R^2_{\mu\nu} + R^2 \right)
\end{equation}
So we have two candidate central charges $a$ and $c$. This $a$
is a closer analogue to the central charge in $1+1$ dimensions.
In $3+1$ dimensions we can define the Weyl-tensor, which we were not able
to do in $1+1$ dimensions:
\begin{equation}
\tensor{W}{^\mu_\nu_\sigma_\rho} = \tensor{R}{^\mu_{\nu\sigma\rho}} - \text{ traces }
\end{equation}
Then we can write this $I$ factor as:
\begin{equation}
I\propto \int \sqrt{-g} W
\end{equation}
Then we have the following theorem:
\begin{thm}[$a$-theorem]
This $a$ is the measure of numbers of degrees of freedom in the field theory. 
\end{thm}
This was proposed by J. Cardy in 1988, but was ``proven'' by 
Komargodski and Schwimmer in 2011. They believe to have a nonperturbative proof \cite{a_theorem_proof}.
\begin{rmk}
For CFTs in $3+1$ dimensions which have purely gravitational dual, 
it has been proven that automatically $a = c$.
\end{rmk}
\end{exm}

\begin{exm}
In $5+1$ dimensions, we have
\begin{equation}
\cA \sim a \int E_6 + \sum_{i = 1}^3 z_i I_i\left( W \right)
\end{equation}
where $a$,$c_i$ are the $4$ ``central charges.''\footnote{
We put this in quotes, because there is no sense in which any of these are the analogue of the
actual central charge in the sense of the Virasoro algebra.}
\end{exm}

Note that we can repeat this analysis in the context of nonrelativistic 
QFTs. The analogue of the Weyl translation then becomes:
\begin{align}
t &\fromto \lam^z t 
&&
x^i &\fromto \lam x^i
\end{align}
where $z>0$ is the ``dynamical critical exponent'' which is
an important observable quantity associated to an RG fixed point.
If $z = 1$, we are back to the relativistic case, and
$z=2$ gives us Galilean invariant theories.
So this is just a generalization of the Weyl translation we saw before.
Some of the statements which we are used to in relativistic field theories will now not be true here.

\begin{exm}
Consider a theory in $2+1$ dimensions for $z = 2$. 
Define Weyl transformations by:
\begin{equation}
g_{\mu\nu}= 
\begin{pmatrix}
-N^2 + g^{ij} N_i N_j & N_i \\
N_i & g_{ij}
\end{pmatrix}
\end{equation}
this form of the metric is called the ADM decomposition.
So $g_{ij}$ is like the spatial portion of the metric, $N_i$ is like the shift vector,
and $N$ is the lapse function.

Then in this language, the anisotropic Weyl transformations are given by:
\begin{align}
g_{ij} \fromto \Om^2\left( t , \vec{x} \right) g_{ij}
&&
N_i \fromto \Om^2 N_i 
&&
N \fromto \Om^z N
\end{align}
Now we can write:
\begin{multline*}
\cA = c_K \int \d{^2 x} \d{t} \sqrt{g} N \left[ K_{ij} K^{ij} - \frac{1}{2} K^2 \right] \\
+ c_V \int \d{^3 x} \d{t}\sqrt{g}N\left[ R - \left( \frac{\nab _i N}{N} \right)^2 + 
\frac{\nab N}{N}\right]^2
\end{multline*}
So we have two ``central charges,'' a kinetic one $c_K$, and a potential one $c_V$.
Note $K_{ij}$ is the extrinsic curvature:
\begin{equation}
K_{ij} = \frac{1}{2N} \left( \dot g_{ij} - \nab_i N_j - \nab_j N_i \right)
\end{equation}

What about isometries in one higher dimension?
We will get something similar to our treatment of AdS.
We get something like:
\begin{equation}
\d{s}^2 = -\frac{\d{t}^2}{u^{2z}} + \frac{\d{\vec{x}}}{u^2} + 
\frac{\d{u}^2}{u^2}
\end{equation}
Note that for $z= 1$, this is not an AdS.
This is often times referred to as Lifshitz spacetime.

So we can start with GR (i.e. $\Lam < 0$) and add whatever $I_{\mu\nu}$. 
The problem is, very often these tensors are not self-consistent. 
So we have that for nonrelativistic CFT, having a relativistic dual, means $c_K\neq 0$
and $c_{ij} = 0$. 
Alternatively, we can start with nonrelativistic gravity with no matter.
In this case with nonrelativistic gravity duals, we have $c_K \neq 0$ and $c_V\neq 0$. 
\end{exm}

\subsection{The anti-bracket in practice}

The following claim is often made about the SM or even generic YM:

\begin{clm}
Chiral anomalies are one-loop exact.
\end{clm}

To show this, one must invoke the full structure of the 
Zinn-Justin equation. That is, write
\begin{equation}
\Gamma = S + \Gamma_1 + \Gamma_2 + \cdots
\end{equation}
then if the theory is free of anomalies, it must satisfy the master equation 
$\left( \Gamma , \Gamma \right) = 0$.
If there are only one-loop anomalies, we have $\left( S , \Gamma_1 \right) = G_1 \neq 0$. 
But then the anti-bracket has trivial BRST cohomology for this particular anomaly
at all orders $n\geq 2$.

So consider some arbitrary non-abelian gauge theory. We will now concern ourselves over
the gauge anomalies here.
We treat this using the so-called descent equations (or Stora-Zumino descent equations).
The idea is to move from our generic number of dimensions $d = 2n$ to some higher dimensional scenario
$2n + 2$:
\begin{align}
\Om_{2n+2} = 
\Tr\ubr{\left( F\ext \cdots \ext F \right)}{n+1}
\end{align}
In a formal sense, we clearly have $s\Om_{2n+2}^{\left( 0 \right)} = 0$. 
We know that since this is BRST closed, locally it will be BRST exact. 
Then by definition, it also satisfies the local condition that it is the total
derivative of some $2n + 1$ differential form:
\begin{equation}
\Om_{2n+2}^{\left( 0 \right)} = d \Om_{2n+1}^{\left( 0 \right)}
\end{equation}
We also know that $sd + ds = 0$, so now we can write:
\begin{align}
d\left( s\Om_{2n+1} \right) = -sd\left( \Om_{2n+1} \right) = 
-s^2\left( \cdots \right) = 0
\end{align}
This means that $s\Om_{2n+1}$ is the derivative of some object of ghost number $1$:
\begin{align}
s\Om_{2n+1}= d
\Om_{2n}^{\left( 1 \right)}
&&
G = \int \Om_{2n}^{\left( 1 \right)}
\end{align}
We could push this even further to get that
\begin{equation}
d\left( s\Om_{2n}^{\left( 1 \right)} \right) = 0
\end{equation}
which implies that 
\begin{equation}
s\Om_{2n}^{\left( 1 \right)} = d \Om_{2n-1}^{\left( 2 \right)}
\end{equation}
so these $\Om^{\left( n \right)}$ objects we are getting are the analogues of the 
$S_{\text{schwinger}}$ objects from before.

\chapter{Effective field theory of gravity}

\section{Preliminaries}

We know that gravity is well described by the Einstein-Hilbert action:
\begin{align}
S = \frac{1}{16 \pi G_N} \int  \d{^D x} \sqrt{-g} \left( R\right)
&&
G_N\sim \frac{1}{M_{\text{pl}}^{D-2}}
\end{align}
However, we also know that this theory is non-renormalizable, 
since $\left[ G_n\right] < 0$ for $D > 2$. 
UV completion suggests a breakdown at $E\sim M_{\text{pl}}$.
But there is also an IR breakdown 
so we can also add a nontrivial constant term:
\begin{equation}
S = \frac{1}{16 \pi G_N} \int  \d{^D x} \sqrt{-g} \left( R - 2\Lam \right)
\end{equation}
Note that $\left[ \Lam \right] = 2$, since it sits on the same footing as $R$, which has dimension $2$. 
If we take the cosmological constant problem seriously, we see that there should be a breakdown at 
$\sim 1$ eV. So if we want to embed this into a quantum field theory, 
we are facing multiple problems. 

\section{UV-IR connection?}

Phenomenologically, we see ourselves living in a universe where
$\Lam$ is much much smaller than $M_{\text{pl}}$. 
So maybe we should just interpret this as indicating a gravitational connection. 
More pragmatically, we can consider the existence of black holes, 
which implies the holographic principle.
The scaling of $S$ with the area $\cA$ 
suggests the scaling of the number of degrees of freedom.
Cosmology also suggests this connection. 
This is driven by the dramatic time dependence in the evolution of the universe. 

Note that we can have gravity with $\Lam < 0$, $\Lam > 0$, and flat spacetime 
$\Lam = 0$. 
It is known that holography displays AdS/CFT behaviour for $\Lam < 0$. 
Here we have consistency with SUSY, and string theory, so there is no 
cosmological constant problem in this highly SUSY case. 
There is a UV-IR connection in this case, but this is still an open problem 

For $\Lam > 0$, this is our real universe, we have no AdS/CFT, but we might have some sort
of de Sitter space CFT duality. There is also no SUSY consistency here. 
String theory also does a bad job of describing this, for reasons that are still
somewhat mysterious. 
Therefore we have to resort to EFT techniques. 
In this universe we have many interesting results such as
the horizon problem, and the flatness problem.
A priori, we don't know what solutions these might have, but inflation 
might provide an explanation for this.

\section{Review of EFT}

The generic idea of EFT is that we don't need to understand everything up to arbitrarily high energy
to form an understanding of low-energy physics. 
In addition, the physics of the low energy should be independent of the high energy physics. 
That is, we are essentially de-coupling the UV and IR. 
So why do we expect this to be the case?
Recall that in QM, perturbation theory suggests that we should be summing over intermediate 
states. 
As it turns out, the correct explanation for this decoupling has to do with the uncertainty
principle. Namely, in the absence of some mysterious UV-IR connection, 
short distances are probed by high energies. So unless we can change the Heisenberg 
uncertainty principal to some adjusted relationship, the contribution of high energies
will be purely local in EFT, which can be absorbed into a redefinition of some finite
number of coefficients of operators in that effective field theory. 

\section{Gravity as an EFT}

\subsection{Linear $\sigma$ model}

We first consider a linear $\sigma$ model (L$\sigma$M). This is given by:
\begin{equation}
S = \frac{1}{2} \int \d{^4 x} \left( \p_\mu \vec{\pi} \p^\mu \vec{\pi}
+\p_\mu \sigma\p^\mu \sigma + \mu^2\left( \sigma^2 + \pi^2 \right) - 
\frac{\lam}{2}\left( \sigma^2 - \vec{\pi}^2 \right)^2
\right)
\end{equation}
where $\sigma + i\vec{\tau} \vec{\pi} - \Sigma$
and we have $V_L\Sigma V_R^\dag$ where $V_L$ is $\SU\left( 2 \right)_L$
and $V_R^\dag$ is $\SU\left( 2 \right)_R$. 
We also have $\lr{\sigma} = v = \mu / \sqrt{\lam}$ and $m_0^2 = 2\mu^2$. 
This scale set by the vacuum expectation value should be though of as the plank scale, and the Goldstones
$\vec{\pi}$ should be thought of as the EH description of gravity.

EFT at $E \ll M_{pl}$ is given by
\begin{align}
S_\eff = \frac{v^2}{4} \int \d{^4 x} \Tr\left( \p_\mu U \p^\mu U^\dag \right)
&&
U = e^{i\vec{\tau}\vec{\pi}}{v}
\end{align}
Note that we have the following:
\begin{enumerate}
\item This theory is non-renormalizable
\item This theory has a two-derivative action
\item This theory looks very much like GR, so it is indeed a good proxy model.
\end{enumerate}

Consider the scattering:
\begin{equation}
\pi^+ \pi^- \fromto \pi^+ \pi^-
\end{equation}
In the microscopic theory, we will be summing two tree-level diagrams to get the
tree-level contribution:
\begin{equation}
\begin{tikzcd}
\pi^+ \arrow[dash,fermion]{rrdd}&&\, \\ \\
\pi^0 \arrow[dash,fermion]{uurr}&&\,
\end{tikzcd}
\qquad + \qquad
\begin{tikzcd}
\pi^+\arrow[fermion,dash]{rr}&
\cdot \arrow[dash]{d}{\tilde \sigma}& \,
\\
\pi^0\arrow[dash,fermion]{rr}&\cdot& \,
\end{tikzcd}
\end{equation}
with amplitudes:
\begin{align}
i\cM &= -2i \lam + \left( 2\pi \lam v \right)^2
\frac{i}{q^2 - m_\sigma^2}\\
&= i\left[ \frac{q^2}{v^2} + \frac{q^4}{v^2 m_\sigma^2} + \cdots \right]
\end{align}
In the effective theory, we take this highly non-linear Lagrangian and expand in
powers of $\pi$. Then focusing on the $\pi^4$ piece, 
we get only one diagram yielding
\begin{equation}
i\cM = \frac{iq^2}{v^2}
\end{equation}
But what about corrections at tree level?
For gravity, we don't a priori know that this makes sense for this non-renormalizable theory, 
but it turns out there are meaningful loop corrections.
In order to determine which questions are meaningful and which ones aren't we first rename the fields 
a bit:
\begin{align}
\sigma + i\vec{\tau } \cdot \vec{\pi}
\left( v + \sigma ' \right) e^{i\vec{\tau}\cdot \vec{\pi}' / v}
= v + \sigma' + i\vec{\tau} \cdot \vec{\pi}'+ \cdots
= v + \sigma + i\vec{\tau}\cdot \vec{\pi}+ \cdots
\end{align}
which should not change the physics. This gives us:
\begin{equation}
S = \frac{1}{4} \int \d{^4 x} \left( v + \sigma \right)^2
\Tr\left( \p_\mu U \p^\mu U^\dag \right) + S\left[ \sigma \right]
\end{equation}
Now we want to integrate out the heavy mode $\sigma$.
At leading order, ignoring $\sigma$ give us a good approximation, which gives us
the EFT from before: $S_\eff \equiv S_\eff^{\left( 0 \right)}$.
To consider the next order, we can no longer ignore $\sigma$.
This is just a simple Gaussian integral, which yields:
\begin{equation}
S_\eff^{\left( 0 \right)} + 
\frac{v^2}{8m_\sigma^2} \int \Tr\left( \p_\mu U\p^\mu U^\dag \right)
\end{equation}
Now how do we treat loops and renormalization?
More generally, the local Lagrangian for $S_\eff$, with unknown coefficients,
in energy expansion yields $q^2 / v^2$.
Well we know that the scale we expect things to break down is given by:
\begin{multline}
S_\eff = \frac{M_{\pl}^2}{4} \Tr\left( \p_\mu U \p^\mu U^\dag \right)
+ l_1 \left( \Tr\left( \p_\mu U \p^\mu U^\dag \right) \right)^2 \\+
l_2 \Tr\left( \p_\mu U \p^\mu U \right)\Tr\left( \p^\mu \p_\nu U \right) + 
\cO\left( \p^6 \right)
\end{multline}
Now the matching conditions predict:
\begin{align}
l_1 = \frac{v^2}{8 m_\sigma^2}
&&
l_2 = 0
\end{align}
but the renormalized one-loop corrected values will be:
\begin{align}
l_1^R = \frac{v^2}{8m_\sigma^2} + 
\frac{1}{384 \pi^2} \left[ \log \frac{m_\sigma^2}{\mu^2} - \frac{35}{6} \right]
&&
l_2^R = \frac{1}{192 \pi^2} \left[ \log
\frac{m_\sigma^2}{\mu^2} - \frac{11}{6}\right]
\end{align}

\subsection{Non-linear $\sigma$-model}

Now we can apply this to the non-linear $\sigma$ model, 
and we won't be able to do matching, but we can follow roughly the same steps.
The action here will be:
\begin{multline}
S_{gr} = \int \d{^4 x} \sqrt{-g} \bigg\{ 
\frac{1}{16\pi G_n} \left( R - 2\Lam \right) + c_1 R^2+ 
c_2 R_{\mu\nu}R^{\mu\nu} \\
+ c_3 R_{\mu\nu\sigma\rho}R^{\mu\nu\sigma\rho} + 
c_1^{\left( 3 \right)} R^3 + c_1^{\left( 4 \right)} R^4 + \cdots\bigg\}
\end{multline}

In the case of the NL$\sigma$M, we have the following:
\begin{thm}[Weinberg]
This systematic expansion in the powers of momentum makes sense.
\end{thm}

\begin{exm}
We can calculate graviton-graviton scattering with this approach. 
\begin{equation}
\cA\left( ++ , ++ \right) = 
\frac{i}{4}\frac{k^2 s^3}{tu}
\left\{ 
1 + \frac{k^2 stu}{\#} \left[ \frac{\log{-u}}{st} + \cdots \right] + \cdots \right\}
\end{equation}
We also get quantum corrections to the potential:
\begin{equation}
V\left( r \right) = -\frac{G M m }{r} \left[ 
1 + \frac{G\left( M + m \right)}{rc^2} + 
\frac{41 G\h}{10\pi r^2 c^3}+ \cdots\right]
\end{equation}
The interesting features here are more visible in momentum space. 
We get $1 / r\sim 1 / q^2$, and then the quantum corrections go like:
\begin{align}
\frac{1}{r^2} \sim \frac{1}{q^2} \sqrt{q^2}
&&
\frac{1}{r^3} \sim \frac{1}{q^2} q^2 \log q
\end{align}
so we have the expected propagator, multiplied by something non-analytic in $q$. 
This is a sign that this came from the $1$-loop correction.
On the other hand, if we had a local term such as $\dd^3\left( \vec{x} \right)$ we get:
\begin{equation}
\dd^3\left( \vec{x} \right)\sim \frac{1}{q^2} q^2
\end{equation}
so terms like this are outside of the scope of this technique.
\end{exm}

For more information on this approach, see
\cites{donoghue_1,donoghue_2}.

\subsection{Technical naturalness}

This isn't really a great EFT (yet). This is a result of technical naturalness. 
This is an idea which first came from 't Hooft. The initial concept was that microscopic 
physics implies macroscopic physics. That is, the theory flows from the UV to the IR. 
The mathematical statement here is that the parameters $\al$ in  $S$ can be small, as long as
$\al = 0$ implies enhanced symmetry.
This means,that assuming technical naturalness, $\Lam_c$ cannot be ``small'' since 
setting $\Lam_c=0$ doesn't give us any enhanced symmetry. 
Recall that for $\Lam_c < 0$, we have the AdS/CFT correspondence, and an $\SO\left( 3,2 \right)$. 
For $\Lam_c = 0$, we have the inhomogeneous $\SO\left( 3,1 \right)$. 
For $\Lam_c > 0$, we have $\SO\left( 4,1 \right)$ in de Sitter space. 
In all of these cases, $\dim G = 10$.
This means $\Lam_c = 0$ doesn't enhance any symmetries, so not only can it not be small, but in fact
$\Lam_c \sim \Lam^2$. 
So this theory doesn't break down at energies smaller or equal to the order of $M_{pl}$, but rather
$E < \sqrt{\Lam_c}$. 

\begin{exm}
Consider the following scalar field theory:
\begin{equation}
S = \frac{1}{2} \int \d{^4 x} \left( \left( \p \phi \right)^2 - m^2 \phi^2 - 
\frac{\lam}{12} \phi^2 \right)
\end{equation}
This is already self interaction without adding anything else.
But what kinds of symmetries do we have here? Well we have the trivial shift symmetry
$\phi \mapsto \phi + c$. 
Is this theory invariant? Well we see an issue with the second two terms, but when
these are $0$, this means we have an enhanced symmetry. Meaning we can self consistently make
$m^2 \sim \e M_{pl}^2$ and $\lam \sim \e$ for $\e \ll 1$ by the postulate of technical naturalness.
\end{exm}


For more on EFT of gravity see \cites{burgess_gravity_1,burgess_gravity_2}.

\section{Effective field theory of inflation}

\subsection{Cosmological setup}

We know that we live in a cosmological spacetime, and by the Copernican principal
that we aren't in any sort of special place, we will assume that we have a so-called FLRW\footnote{
This stands for Friedmann - Lema\^itre - Robertson - Walker.
This is sometimes shortened to FRW.} metric. These metrics are of the form:
\begin{equation}
\d{s^2} = -\d{t^2} + a^2\left( t \right) \left[ 
\frac{\d{r^2}}{1 - kr^2} + r^2 \d{\Om^2}\right]
\end{equation}
where
\begin{equation}
k = 
\begin{cases}
0 & \RR^3 \\
1 & S^3 \\
-1 & H^3
\end{cases}
\end{equation}
Now we can define a so-called Hubble parameter:
\begin{equation}
H\left( t \right) = \p_t \log a\left( t \right) = \frac{\dot a\left( t \right)}{a\left( t \right)}
\end{equation}
As for the dynamics, we can either take our favorite EFT of gravity, or the Einstein equations. 
We will take the latter. In this context, this gives us the Friedmann equations:
\begin{align}
H^2  = \frac{1}{3M_{pl}^2} \rho &&
\dot H + H^2 = -\frac{1}{6 M_{pl}^2 }\left( \rho + 3p \right)
\end{align}

It is often useful to introduce conformal time:
\begin{equation}
\d{\tau} = \frac{\d{t}}{a\left( t \right)}
\end{equation}
In this language the metric becomes:
\begin{equation}
\d{s^2} = a^2\left( \tau \right)\left[ 
-\d{\tau}^2 + \d{r^2} + r^2 \d{\Om^2}\right]
\end{equation}
Now in order to talk about inflation, we want to consider the causal structure on these
cosmologies. 
In conformal time, geodesics are straight lines, and at angles $\pi/4$ in the 
$\left( \tau , r \right)$ plane.
We want to calculate the maximum distance a photon can travel in a given
time-span:
\begin{align}
\Delta r = \Delta \tau = \int_{t_0}^{t_f} \frac{\d{t'}}{a\left( t' \right)}
\end{align}
Then if we take $t_i$ to be the origin of the universe, we have $\Delta r_{\max}\left( t \right)$
is the \emph{comoving particle horizon}.
We can rewrite this as:
\begin{equation}
\tau = \int \frac{\d{t}}{a\left( t \right)} = 
\int \ubr{\frac{1}{a H}}{\text{comoving Hubble radius}} \d{\log a}
\end{equation}

\begin{exm}
Consider a system with matter, and $w = p / rho$. 
We can then solve the Friedmann equations to get:
\begin{equation}
\frac{1}{aH} \propto a^{\left( 1 + 3w \right)/2}
\end{equation}
In the GR literature, the case that $1+3w > 0$ 
is known as the strong energy condition (SEC).
\end{exm}

\subsection{Inflation}

Most people who are advocates of inflation, cite that this ``solves'' the ``horizon''
and ``flatness'' problem.
The horizon problem is, roughly, that if we have the SEC, 
this disagrees with the observed homogeneity of the cosmological microwave background. 
That is, if we have some finite initial conformal time $t_i$,
when we measure background at some early conformal time $\tau_{\text{rec}}$, there isn't
enough time for photons to move in such a way that allows for the homogeneity that we observe. 
See \cref{fig:horizon_problem} for a diagram of this.
\begin{figure}
\centering
\includegraphics[width=0.5\linewidth]{horizon_problem.pdf}
\caption{Conformal diagram showing the horizon problem.}
\label{fig:horizon_problem}
\end{figure}
The solution to this is effectively letting the universe infinitely long before this recorded time, 
so this shaded region covers the full region. This means:
\begin{equation}
\frac{d}{\d{t}} \left( \frac{1}{aH} \right) < 0
\end{equation}
We can turn this apparent contraction around, and end up with a different diagram. 
So the period where this is satisfied can be called the \emph{inflationary time period}.
Then if we have $w < -1/3$, we might be overturning the consequence of the SEC, and
we might end up solving this inequality. 
For example, we might have a system with a matter source, and $\Lam_c$ with $w = -1$. 

The flatness problem is, briefly, that if we start with a somewhat curved spacetime, if we just apply
FRW without inflation, spacetime becomes very curved. With inflation, we get a spacetime which is
roughly as flat as we seem to observe it.

In order for this to be consistent, we need 
\begin{align}
\e \ceqq - \frac{\dot H}{H^2} = 
\frac{3}{2}\left( 1 + \frac{p}{\rho} \right) < 1
&&
\abs{\eta} \ceqq  \frac{\abs{\dot \e}}{H\e} \ll 1
\end{align}
the second condition assures that inflation lasts long enough. 

We now consider an example called slow roll inflation. Consider some theory:
\begin{equation}
S = \int \d{^4 x} \sqrt{-g} \left( \frac{M_{pl}}{2} R - 
\frac{1}{2} g^{\mu\nu} \p_\mu \phi \p_\nu \phi - V\left( \phi \right)\right)
\end{equation}
where the field $\phi$ is called an inflaton.
So what conditions do we have on $V\left( \phi \right)$?
Well we saw the two conditions on inflation already, so just 
looking at the EOM for this system we can just calculate:
\begin{align}
\e = \frac{\dot \phi^2}{2M_{pl}^2 H^2}
&&
\eta \ceqq 2\left( \e - \dd \right)
&&
\dd = -\frac{\ddot{\phi}}{H\dot \phi}
\end{align}
This leads to the so-called eta problem which has several different incarnations. 
Note that we are already fine tuning so the constant piece is absent, which is already somewhat
unsatisfying.
In addition, the potential $V\left( \phi \right)$,
starts as $m^2 \phi/ 2$, and then it turns out that the quantum correction of $\eta$ is given by:
\begin{equation}
\Delta \eta \approx M_{pl}^2 \frac{\Delta V''}{V} \sim \frac{\Lam}{H^2}
\end{equation}
so the punchline is, that
if we have a UV cut-off in an EFT of inflation, then 
we need to choose this cut-off, and if this cut-off is the plank scale, this is a huge
contribution, much larger than $1$.
We might then think the cut-off should only be $H$.
But we can't solve the CC problem, so maybe we shouldn't even be 
allowed to do this. In any case, this still won't really help us, because this still
just gives a correction to $\eta$ of order $1$. There is no symmetry gained by making $\eta = 0$, 
so in the least exciting possible scenario, this correction is still of order $1$.

There is also a so-called \emph{strong eta problem,} which considers
an EFT with an action of the form:
\begin{equation}
S = \cdots \int \frac{c_1}{\Lam^\dd} \phi^2 \left( \p \phi \right)^2 R^2 + \cdots
\end{equation}
then the only way out is to postulate the shift symmetry, but then coupling to gravity
breaks this symmetry. 
So this strong problem says the inflationary physics is sensitive to an infinite collection
of higher dimensional operators $c_i$. 
So an infinite sequence of these will substantially influence inflation, 
and we therefore have to complete an infinite sequence of fine-tuning.

For more information on EFT of inflation, see 
\cites{baumann_inflation_1,baumann_inflation_2,burgess_inflation}.
\chapter{Applications of QFT to finance}

Much of the content and figures from this section are from \cite{kleinert_finance}.

\section{Fluctuations of financial assets}

Denote the price of a stock, or some other asset, by $S\left( t \right)$. 
This is a complex thing to follow, but over a long period of time, we start to see some trends. 
In particular, this is relatively well approximated by exponentials as plotted in 
\cref{fig:log_dow}.
\begin{figure}
\centering
\includegraphics[width=0.5\textwidth]{log_dow_jones.png}
\caption{Logarithm of the Dow Jones industrial index over a range of 80 years.}
\label{fig:log_dow}
\end{figure}
The fluctuations of this $S\left( t \right)$
have an average width, which we will call the \emph{volatility of the market}, 
written as $\sigma$. Over long-times, that is, over a day or month,
the volatility $\sigma$ behaves stochastically as illustrated in \cref{fig:vol_sp}.
\begin{figure}
\centering
\includegraphics[width=0.6\textwidth]{vol_sp.png}
\caption{(a) Index S\&P 500 for over 13 years, recorded every minute.
(b) Volatility every 30 minutes.}
\label{fig:vol_sp}
\end{figure}
The volatilities approximately follow a Gamma distribution as illustrated in 
\cref{fig:vol_fits}.
The normalized log-normal distribution has the form:
\begin{equation}
D^{\text{log-norm}}\left( z \right) = 
\left( 2\pi \sigma^2 z^2 \right)^{-1/2}\exp\left( 
-\left( \log z - \mu \right)^2 / 2\sigma^2 \right) \ .
\label{eqn:log-norm}
\end{equation}

\begin{figure}
\centering
\includegraphics[width=0.8\textwidth]{vol_fits.png}
\caption{Comparison of best Gaussian, log-normal, and
Gamma distribution to volatilities over 300 minutes. 
The normalized log-normal distribution has the form
in \eqref{eqn:log-norm}.
We will revisit the Gamma distribution.}
\label{fig:vol_fits}
\end{figure}

\subsection{Harmonic approximations}

The fluctuations of $S$ are roughly given by the following expression:
\begin{equation}
\frac{\dot S\left( t \right)}{S\left( t \right)} = 
r_S + \eta\left( t \right)
\end{equation}
where $r_S$ is the growth rate, and $\eta\left( t \right)$ is a white noise
function given by the following:
\begin{align}
\lr{\eta\left( t \right)} = 0
&&
\lr{\eta\left( t \right)\eta\left( t' \right)} = \sigma^2 \dd\left( t - t' \right)
\end{align}
Note that we will also call this $v\ceqq \sigma^2$, the variance.

We call $\d{S\left( t \right)}/S\left( t \right)$ to be the \emph{return}
of the asset.
Financial data is given with discrete time intervals $\delta t$, so often
times this is used instead of the infinitesimal $\d{t}$.
Define the following:
\begin{equation}
x\left( t \right) \ceqq \log S\left( t \right)
\end{equation}
This implies a stochastic differential equation for linear growth:
\begin{equation}
\dot x \left( t \right) = \frac{\dot S}{S} - \frac{1}{2} \sigma^2 + 
r_x + \eta\left( t \right)
\end{equation}
where we have taken
\begin{equation}
r_x \ceqq r_s - \frac{1}{2}\sigma^2
\end{equation}
to be the \emph{drift} of the process. 
But where does this $\sigma^2 / 2$ come from?

For functions of a stochastic variable $x\left( t \right)$ we can take
the following formal expansion in $\d{t}$:
\begin{align}
\d{x} \left( t \right) &=
\frac{\d{x}}{\d{S}} \d{S}\left( t \right) + 
\frac{1}{2} \frac{\d{^2 x}}{\d{S^2}} \d{S^2}\left( t \right) + \cdots \\
&= \frac{\dot S\left( t \right)}{S\left( t \right)}\d{t} - 
\frac{1}{2} 
\left[ \frac{\dot S\left( t \right)}{S\left( t \right)} \right]^2 \d{t^2} + \cdots
\end{align}
now we can take $\dot x^2 \d{t}\mapsto \lr{\dot x^2} \d{t}= \sigma^2$
so we have
\begin{equation}
\left[ \frac{\dot S\left( t \right)}{S\left( t \right)} \right]^2 \d{t} \mapsto
\dot x^2 \left( t \right) \d{t} = \sigma^2
\end{equation}

\subsection{Approximations by distributions}

This description of the logarithms of the stock prices by Gaussian fluctuations around a linear
trend is clearly only a rough approximation. 
At small time intervals, such as every minute, the logarithms of the stock prices
are well approximated by Gaussians.
However, for rare events, there is a much higher probability than in Gaussians. 
Such distributions are observed to have \emph{heavy tails} in comparison with
the light tails of Gaussian distributions. 

\begin{exm}
In an attempt to fix this heavy-tail issue, we might 
consider the Fourier transform:
\begin{equation}
\tilde L_{\sigma^2}^\lam\left( z \right)
\ceqq \int_{-\infty}^\infty \frac{\d{p}}{2\pi}
e^{ipz} L_{\sigma^2}^\lam \left( p \right)
\end{equation}
where we take:
\begin{equation}
L_{\sigma^2}^\lam \left( p \right)\ceqq
\exp\left[ -\left( \sigma^2 p^2  \right)^{\lam / 2} / 2 \right]
\end{equation}
Phenomenologically, this $\lam$ is taken between $1.2-1.5$.
One might consider whether this $\lam$ could be a critical exponent of some RG fixed point. 
Experimentally, these are referred to as L\'evy tails. See \cref{fig:levy}
\begin{figure}
\centering
\includegraphics[width = \textwidth]{levy.png}
\caption{Left: L\'evy tails of the S\&P 500 index (1 minute log-returns) plotted
against $z / \dd$. 
Right: Double-logarithmic plot exhibiting power-like tail regions of the S\&P
500 index (1 minute log-returns).}
\label{fig:levy}
\end{figure}
\end{exm}

Before listing additional examples of distributions, note that
for any distribution $\tilde D\left( z \right)$, we can define some useful objects.
Take the Fourier decomposition:
\begin{equation}
\tilde D\left( z \right) \int \frac{\d{p}}{2\pi}
e^{ipz} D\left( p \right)
\end{equation}
and we can define the Hamiltonian to be $H\left( p \right)$ such that:
\begin{equation}
D\left( p \right) \equiv e^{-H\left( p \right)}
\end{equation}
Equivalently, we might take the following as the definition:
\begin{equation}
e^{-H\left( p \right)} \equiv
\lr{e^{-ipz}}
\end{equation}

Note that now we can identify the L\'evy distributions as having a Hamiltonian
given by 
\begin{equation}
H\left( p \right) = 
\frac{1}{2} \left( \sigma^2 p^2  \right)^{\lam / 2}
\end{equation}

\begin{exm}
We now see the Gamma distribution mentioned earlier:
\begin{equation}
\tilde D^{\text{Gamma}}_{\mu,v}\left( z \right) = 
\frac{1}{\Gamma\left( v \right)} \mu^v z^{v-1} e^{-\mu z}
\end{equation}
This has the following Hamiltonian:
\begin{equation}
H\left( p \right) = v\log\left( 1 - \frac{ip}{\mu} \right)
\end{equation}
\end{exm}

\subsection{Boltzmann distribution}

The first question a physicist might ask about all this, is if we can get
away with modelling this system with a Boltzmann distribution.
That is, does the system have a characterising temperature?
As it turns out, this point of view leads to a high-performing model.
The data for the S\&P 500 and NASDAQ 100 are displayed in \cref{exr:boltz}.

\begin{figure}
\centering
\includegraphics[width=0.8\textwidth]{boltz.png}
\caption{Boltzmann distribution of S\&P 500 and 
NASDAQ 100 high-frequency log-returns recorded by the minute.}
\label{exr:boltz}
\end{figure}

The data are fitted by the distribution:
\begin{equation}
\tilde B\left( z \right) = 
\frac{1}{2T} e^{-\abs{z} / T}
\end{equation}
It is evident that only a small portion of events
(for large $\abs{z}$) do we get a poor fit. 
These events display this heavy-tail behaviour. 
The temperature associated with the stock-market changes relatively
slowly with respect to the socio-political climate, and is high in times of crisis.
We can see in \cref{fig:temp} that crashes give us high temperatures.
\begin{figure}
\centering
\includegraphics[width=0.6\textwidth]{temp.png}
\caption{Dow Jones index over $78$ years ($1929-2006$) and the annual market temperature,
which is remarkably uniform, except in the $1930$'s, in the beginning of the great
depression. Another high extreme temperature occurred in the crash year 1987.}
\label{fig:temp}
\end{figure}

There are many other distributions which can do a relatively good job
of fitting the available economic data, but we omit any discussion of these. 
Many of these distributions seem ad hoc. One might wonder if a physicist can use
the same intuition that drives the consideration of new physical theories to
consider intuitive models rather than models with arbitrary convenient ansantz.
We now develop the path integral formalism in this context.

\section{Path integrals}

Consider the following stochastic differential equation for the logarithms of $S$:
\begin{align}
\dot x\left( t \right) = r_x + \eta'\left( t \right)
&&
\lr{\eta'\left( t \right)}
\end{align}
This $\eta$ is the noise variable, and is arbitrarily distributed.
The constant drift $r_x$ is only uniquely defined when the average $\lr{\eta} = 0$.
This isn't generally true, since plenty of the distributions used in practice have nonzero expectation.
In any case, we redefine the constant in the expansion of the Hamiltonian so we
can just write $\dot x\left( t \right) = \eta\left( t \right)$.
Now the probability distribution of the endpoints $x_b = x\left( t_b \right)$
for the paths starting at a certain initial point $x_a = x\left( t_a \right)$
can be written as a path integral:
\begin{equation}
P\left( x_b t_b \vert x_a t_a \right) = 
\int \cD \eta\left( t \right) \int_{x_1}^{x_b}
\cD x\left( t \right)
\exp\left( -\int_{t_a}^{t_b} \d{t} \tilde H\left( \eta\left( t \right) \right)
\right)
\dd\left[ \dot x - \eta \right]
\label{eqn:fin_path_int}
\end{equation}
This $\tilde H\left( \eta \right)$ is defined to be:
\begin{equation}
\tilde H\left( \eta \right) = 
-\log \tilde D\left( \eta \right)
\end{equation}
In the mathematical literature, the measure in the path integral over the noise
\begin{equation}
\cD \mu \equiv \cD \eta P\left[ \eta \right] = 
\cD \eta \exp\left( - \int_{t_a}^{t_b} \d{t} \tilde H\left( \eta\left( t \right) \right) \right)
\end{equation}
of the probability distributions is called the \emph{measure of the process}
$\dot x\left( t \right)$. 
The path integral:
\begin{equation}
\int_{x_a}^{x_b} \cD x \dd\left[ \cdot x - \eta \right]
\end{equation}
is called the \emph{filter} which determines the distribution of $x_b$
at time $t_b$ for all paths $x\left( t \right)$ starting at $x_a = x\left( t_a \right)$. 

The correlation functions of the noise variable $\eta\left( T \right)$ in
\cref{eqn:fin_path_int} are given by a straightforward functional generalization of what we
have already seen.
In particular, we express the noise distribution $P\left[ \eta \right]$ from \cref{eqn:fin_path_int}
as a Fourier path integral
\begin{align}
P\left[ \eta \right] = 
\exp\left( -\int \d{t} \tilde H \right) =
\int \frac{\cD p\left( t \right)}{2\pi}
\exp\left( \int \d{t} \left[ ip\left( t \right) \eta\left( t \right)
- H\left( p\left( t \right) \right)\right] \right)
\end{align}

\subsection{Time evolution of distribution}

The $\dd$ function in \cref{eqn:fin_path_int} may be represented by a Fourier integral leading to the
path integral
\begin{align}
P\left( b\vert a \right) &=
\int \cD \eta \int \cD x \int \cD p
\exp\left( \int i\dot x - i p\eta - \tilde H\left( \eta\left( t \right) \right) \right) \\
&= \int \cD x \int \frac{\cD p}{2\pi}
\exp\left( \d{t} \left( ip\dot x - H\left( p\left( t \right) \right) \right) \right) \\
&= \int_{-\infty}^\infty \frac{\d{p}}{\left( 2\pi \right)}
\exp\left( ip\left( x_b - x_a \right) - 
\left( t_b - t_a \right)H\left( p \right)\right)
\end{align}
where we have integrated out the noise variable, and integrated over all $x\left( t \right)$
with fixed endpoints. We simply write this as:
\begin{equation}
P\left( x,t \right) = \int_{-\infty}^\infty \frac{\d{p}}{2\pi}
\exp\left( ipx - t H\left( p \right) \right)
\end{equation}
Note that as expected, this satisfies the Markovian property:
\begin{equation}
P\left( b\vert a \right) = \int \d{x_c} P\left( b\vert c \right)P\left( c\vert a \right)
\end{equation}
as well as a
Fokker-Planck-type equation
\begin{equation}
\p_t P\left( b\vert a \right) = 
-H\left( -i\p_x \right) P\left( b\vert a \right)
\end{equation}

For more on this subject see \cites{kleinert_finance,wise_finance}.

% >>><<<
\bibliography{physics}

\end{document}

